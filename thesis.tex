% Dissertation template for the Champalimaud Neuroscience Programme
%
%   
% Adapted from Princeton University's template
% Author: Jeffrey Scott Dwoskin <jdwoskin@princeton.edu>
% Adapted from: http://www.math.princeton.edu/graduate/tex/puthesis.html
% and
% the template from Universitat Pompeu Fabra, in B5
% http://www.upf.edu/bibtic/en/guiesiajudes/eines/tesis/quart.html#template
% 
% José R Fernandes, January 2015




% ****************************************************************************************** %
% SETTINGS
% 
%%% For print copies
%% set 'singlespace' option to set entire thesis to single space, and define "\printmode" to remove all hyperlinks for printed copies of the thesis. Delete all output files before changing this mode -- it will turn hyperref package on and off
%\documentclass[12pt,lot, lof, singlespace]{puthesis}
%\newcommand{\printmode}{}

%%% For the electronic copy, use doublespacing, define "\proquestmode" to use outlined links, instead of colored links. 
%%% 
%%% ITQB's paper size is B5
%%% 
\documentclass[11pt,lot,lof,b5paper]{puthesis}
\usepackage[T1]{fontenc}
\usepackage[utf8]{inputenc}
\newcommand{\proquestmode}{}

% I prefer proquestmode to be off for electronic copies for normal use, since the colored links are less distracting. However when printed in black and white, the colored links are difficult to read. 

%%% For early drafts without some of the frontmatter
% Also see the "ifodd" command below to disable more frontmatter
%\documentclass[12pt]{puthesis}




%%%%%%%%%%%%%%%%%%%%%%%%%%%%%%%%%%%%%%%%%%%%%%%%%%%%%%%%%%%%%\
%%%% Author & title page info

\title{Characterizing Motor Cortex Function through Fine-scale Behaviour Readout}
%
\submitted{2016}  % degree conferral date (January, April, June, September, or November)
\copyrightyear{2016}  % year in which the copyright is secured by publication of the dissertation.
\author{Gonçalo C. Lopes}
\adviser{Joseph J. Paton and Adam R. Kampff}  %replace with the full name of your adviser
\departmentprefix{International Neuroscience Doctoral Programme}  % defaults to "Department of", but programs need to change this.
\department{Champalimaud Neuroscience Programme}






%%%%%%%%%%%%%%%%%%%%%%%%%%%%%%%%%%%%%%%%%%%%%%%%%%%%%%%%%%%%%\
%%%% Tweak float placements
% From: http://mintaka.sdsu.edu/GF/bibliog/latex/floats.html "Controlling LaTeX Floats"
% and based on: http://www.tex.ac.uk/cgi-bin/texfaq2html?label=floats
% LaTeX defaults listed at: http://people.cs.uu.nl/piet/floats/node1.html

% Alter some LaTeX defaults for better treatment of figures:
    % See p.105 of "TeX Unbound" for suggested values.
    % See pp. 199-200 of Lamport's "LaTeX" book for details.
    %   General parameters, for ALL pages:
    \renewcommand{\topfraction}{0.85}	% max fraction of floats at top
    \renewcommand{\bottomfraction}{0.6}	% max fraction of floats at bottom
    %   Parameters for TEXT pages (not float pages):
    \setcounter{topnumber}{2}
    \setcounter{bottomnumber}{2}
    \setcounter{totalnumber}{4}     % 2 may work better
    \setcounter{dbltopnumber}{2}    % for 2-column pages
    \renewcommand{\dbltopfraction}{0.66}	% fit big float above 2-col. text
    \renewcommand{\textfraction}{0.15}	% allow minimal text w. figs
    %   Parameters for FLOAT pages (not text pages):
    \renewcommand{\floatpagefraction}{0.66}	% require fuller float pages
	% N.B.: floatpagefraction MUST be less than topfraction !!
    \renewcommand{\dblfloatpagefraction}{0.66}	% require fuller float pages

% The documentclass already sets parameters to make a high penalty for widows and orphans. 



%%%%%%%%%%%%%%%%%%%%%%%%%%%%%%%%%%%%%%%%%%%%%%%%%%%%%%%%%%%%%\
%%% Printed vs. online formatting
\ifdefined\printmode

% Printed copy
% url package understands urls (with proper line-breaks) without hyperlinking them
\usepackage{url}

\else

\ifdefined\proquestmode
%ProQuest copy -- http://www.princeton.edu/~mudd/thesis/Submissionguide.pdf

% ProQuest requires a double spaced version (set previously). They will take an electronic copy, so we want links in the pdf, but also copies may be printed or made into microfilm in black and white, so we want outlined links instead of colored links.
\usepackage[hidelinks]{hyperref}
\hypersetup{bookmarksnumbered}

% copy the already-set title and author to use in the pdf properties
\makeatletter
\hypersetup{pdftitle=\@title,pdfauthor=\@author}
\makeatother

\else
% Online copy

% adds internal linked references, pdf bookmarks, etc

% turn all references and citations into hyperlinks:
%  -- not for printed copies
% -- automatically includes url package
% options:
%   colorlinks makes links by coloring the text instead of putting a rectangle around the text.
\usepackage[hidelinks]{hyperref}
\hypersetup{colorlinks,bookmarksnumbered}


% make the page number rather than the text be the link for ToC entries
%\hypersetup{linktocpage}
\fi % proquest or online formatting
\fi % printed or online formatting


%%%%%%%%%%%%%%%%%%%%%%%%%%%%%%%%%%%%%%%%%%%%%%%%%%%%%%%%%%%%%\
%%%% Use packages 
%%%% 
%%%% add any extra packages you need

%\usepackage{amsfonts}

%%% for inline lists
\usepackage{paralist}

%%% for comments
\usepackage{verbatim}

%%% For enumerations
\usepackage[shortlabels]{enumitem}

%%% For code
\usepackage{listings}

%%% For tables
\usepackage{multirow}
\usepackage[table]{xcolor}

%%% For figures
%\usepackage{subcaption}
\usepackage{tikz}
\usepackage{graphicx}
\usepackage[mode=buildnew]{standalone}
\DeclareGraphicsExtensions{.pdf,.png,.jpg}
\expandafter\def\csname ver@subfig.sty\endcsname{}
\usepackage{subfig,rotate}

% Longtable lets you have tables that span multiple pages.
\usepackage{longtable}
\usepackage{pdflscape}
\usepackage{afterpage}
\usepackage{geometry}

% Booktabs produces far nicer tables than the standard LaTeX tables.
%   see: http://en.wikibooks.org/wiki/LaTeX/Tables
\usepackage{pdfpages}
\usepackage{booktabs}

%set parameters for longtable:
% default caption width is 4in for longtable, but wider for normal tables
\setlength{\LTcapwidth}{\textwidth}

%hyphenation packages
\usepackage{hyphenat}

%bibliography packages
\usepackage[]{apacite}
\usepackage{etoolbox}
\usepackage{environ}
\newtoggle{bibdoi}
\newtoggle{biburl}
\makeatletter
\undef{\APACrefURL}
\undef{\endAPACrefURL}
\undef{\APACrefDOI}
\undef{\endAPACrefDOI}
\long\def\collect@url#1{\global\def\bib@url{#1}}
\long\def\collect@doi#1{\global\def\bib@doi{#1}}
\newenvironment{APACrefURL}{\global\toggletrue{biburl}\Collect@Body\collect@url}{\unskip\unskip}
\newenvironment{APACrefDOI}{\global\toggletrue{bibdoi}\Collect@Body\collect@doi}{}
\AtBeginEnvironment{thebibliography}{
 \pretocmd{\PrintBackRefs}{%
  \iftoggle{bibdoi}
    {\iftoggle{biburl}{\unskip\unskip doi:\bib@doi}{}}
    {\iftoggle{biburl}{\unskip\unskip}{}}
  \togglefalse{bibdoi}\togglefalse{biburl}%
  }{}{}
}

%\setcitestyle{authoryear,round,semicolon,aysep={},yysep={,}}

%math packages
\usepackage{amsmath, amsthm, amssymb}
\usepackage{amssymb}

%SIunit packages
\usepackage{siunitx}

%specifying table and figure packages
\usepackage{sidecap}
\usepackage{multirow}
\usepackage[labelfont=it,labelsep=period]{caption}

%epigraph packages
\usepackage{epigraph}
\let\originalepigraph\epigraph 
\renewcommand\epigraph[2]{\originalepigraph{\textit{#1}}{#2}}
\setlength{\epigraphwidth}{0.8\textwidth}

%quotation packages
\usepackage{csquotes}
\renewcommand*{\mkcitation}[1]{ #1}

%PGF packages
\usepackage{pgf}
\newcommand\inputpgf[2]{{
\let\pgfimageWithoutPath\pgfimage
\renewcommand{\pgfimage}[2][]{\pgfimageWithoutPath[##1]{chapters/figuresChBehaviour/##2}}
\input{#1/#2}
}}

%SVG packages
\usepackage{ifluatex}
\ifluatex
  \usepackage{pdftexcmds}
  \makeatletter
  \let\pdfstrcmp\pdf@strcmp
  \let\pdffilemoddate\pdf@filemoddate
  \makeatother
\fi
\usepackage{svg}
\usepackage{relsize}
\graphicspath{{chapters/figuresChTools/}{chapters/figuresChBehaviour/}}

%%%% Define commands

% Define any custom commands that you want to use.
% For example, highlight notes for future edits to the thesis
%\newcommand{\todo}[1]{\textbf{\emph{TODO:}#1}}

% create an environment that will indent text
% see: http://latex.computersci.org/Reference/ListEnvironments
% 	\raggedright makes them left aligned instead of justified
\newenvironment{indenttext}{
    \begin{list}{}{ \itemsep 0in \itemindent 0in
    \labelsep 0in \labelwidth 0in
    \listparindent 0in
    \topsep 0in \partopsep 0in \parskip 0in \parsep 0in
    \leftmargin 1em \rightmargin 0in
    \raggedright
    }
    \item
  }
  {\end{list}}

% another environment that's an indented list, with no spaces between items -- if we want multiple items/lines. Useful in tables. Use \item inside the environment.
% 	\raggedright makes them left aligned instead of justified
\newenvironment{indentlist}{
    \begin{list}{}{ \itemsep 0in \itemindent 0in
    \labelsep 0in \labelwidth 0in
    \listparindent 0in
    \topsep 0in \partopsep 0in \parskip 0in \parsep 0in
    \leftmargin 1em \rightmargin 0in
    \raggedright
    }

  }
  {\end{list}}

%set exceptions for hyphenation
\hyphenation{Gross-hirns}


%%%%%%%%%%%%%%%%%%%%%%%%%%%%%%%%%%%%%%%%%%%%%%%%%%%%%%%%%%%%%\
%%%% Front-matter

% For early drafts, you may want to disable some of the frontmatter. Simply change this to "\ifodd 1" to do so.
\ifodd 0
%% front-matter disabled while writing chapters
\renewcommand{\maketitlepage}{}
\renewcommand*{\makecopyrightpage}{}
\renewcommand*{\makeabstract}{}
%
%% you can just skip the \acknowledgements and \dedication commands to leave out these sections.
%
\else
%
%
\dedication{% !TEX root = ../Thesis_Sep_2013.tex


\label{dedication}


\emph{In loving memory of Kiba}}
\acknowledgments{% !TEX root = ../Thesis_Sep_2013.tex


\label{acknow}

An impressively large number of people have accompanied me throughout this journey, and to all of them I am deeply and profoundly thankful. All interactions have a meaning we can never recognize and their combined impact ripples into the future unabated. This is true even if I forget to publicly acknowledge some of them, as I'm sure will be the case.

The Champalimaud Neuroscience Programme has been to me a place of deep personal and intellectual transformation. I can barely recognize the person that six years ago set out on the journey from computer science to neuroscience, intrigued by the mysteries of the brain. The efforts of the institute as a whole to bring together people from widely different cultural and academic backgrounds has created a nexus in which interesting personalities cannot help but be forged and tempered. I will never forget the unique opportunity I had to interact with such a large community of scientists from all over the world, and to discuss nearly every possible topic to my utmost satisfaction. These interactions were always done freely and passionately, in the best spirit of scientific companionship, with no regard for hierarchy or rank.

My first dedication goes to my companions of the International Neuroscience Doctoral Programme, Carolina Doran, Simone Lackner, Tiago Marques, Ivo Marcelo, Bruno Miranda, Raimundo Leong, and Gustavo Moreno, who in the year of 2010 embarked with me on this adventure. Together we have shared much more than just our initial training in neuroscience. Even though we all parted to pursue our own individual projects, there were many crucial points during the journey where, even serendipitously, we were still able to support each other as pillars of sanity in the middle of uncontrollable turbulence. It was an honour and a pleasure to have met each one of you and I hope to have contributed back a small inkling of all the inspiration and admiration you have provided me.

Pavel Itskov, Elena Dreosti, Roberto Medina, Scott Rennie, and Samuel Meyler provided continued drive to collaborate on interesting projects, in addition to their friendship. The academic world would be gentler if there were even more opportunities for this kind of short, sweet and honest collaborations. Bassam Atallah and Cindy Poo have diligently introduced me to many of the subtle complications of animal research. If my transition from desk to bench was any success was mainly due to their patience and support in those early days.

I have to give thanks to Alex Gomez-Marin for sharing with me the passion to ask all the embarrassingly simple questions and to keep searching pure hearted for the truth, no matter how deep or how far back in time the answers must be sought. Also to Niccolò Bonacchi and João Frazão for all the time spent in diligent discussions about all kinds of abstract technicalities that almost no one knows or cares about, but that in the end are critical to keep everyone running; especially Frazão for once again offering me the privilege of his unrivalled sharp and constructive criticism. To João Paulo Gomes for the enduring friendship, creative input and influential discussions.

It is hard to endure long without the kind of safe haven and relaxing times which I found in the short stays with my parents, sister and extended family. Even though I was almost invariably lost to my own thoughts, I found irreplaceable solace in these moments of peace which drove some of the most important insights presented in this thesis. I hope you forgive me for all my absent mindedness and know that I love you all very much.

In the members of the Learning Lab, Gustavo Moreno, Rui Azevedo, Sofia Soares, Thiago Gouvêa and Tiago Monteiro, I always found encouraging support at the bench, as well as very critical debate of ideas in our many lab meetings. I know I was always a bit of an outsider intellectually but rest assured that I took into serious consideration each of your many comments and I thank you for the friendship and sharing spirit. 

The Intelligent Systems Lab was quite literally a second home throughout most of the years of this project. To all its members, Joana Neto, João Frazão, Danbee Kim, George Dimitriadis, Lorenza Calcaterra, Pedro Lacerda, and Atabak Dehban I also want to say thank you for the unique companionship and team spirit with which we battled through challenges of any kind. No matter what crazy project was at hand, we were always able to tackle it together, combining our expertise to overcome our individual difficulties. It is not easy to find such equally sharing and devoted spirits that will give so much while expecting so little in return. Thank you for the extraordinary opportunity. Throughout the years there were also many students interning for short projects which helped to shape research in many new weird directions. I want to thank all of them, but in particular to Tim Schröder for a particularly inspiring and fun internship collaboration, where no small measure of work assembling setups, tweaking experiments and trying to make sense of complicated results was achieved.

To the members of my thesis committee, Megan Carey and Leopoldo Petreanu, I want to thank for much needed external perspective and criticism that helped to ground the development of this project. Also to John Krakauer for unparalleled up front criticism and passionate debate about the hardest conceptual questions surrounding the project. His seemingly unlimited energy and pointers to missing literature did much to invigorate me at a time of disappointment and disillusionment.

To Joseph J. Paton for accepting me as a student and taking up a conceptually complicated project. Even though we did not always see eye-to-eye on questions of methodology and interpretation, I have always taken Joe as the highest standard of rigour against which to sharpen otherwise vague and unsupported ideas. Most, if not all, of the deeper search for answers detailed in this thesis were undertaken directly to try and address the many critically keen concerns that were raised in our meetings and discussions. I hope I have succeeded partly in providing some of the much needed justifications.

To Adam R. Kampff for being more than a mentor, a friend and companion; a fresh and inquisitive mind that never detracts from asking the critical questions. In Adam I always found the incredible ability to point out new interesting directions both for experiments and interpretation of results, even when they initially seemed disappointing or confusing. Also for the tremendous capacity to step back and let students develop their own critical thinking and ability to independently do research. It was thanks to this unique style that I was provided with many opportunities to directly defend difficult ideas in front of a large number of other scientists, and this I now see was absolutely fundamental for intellectual growth. Finally, for the unrelenting support and belief in this project, sometimes even when I had myself lost hope.

And finally, to my wife, Joana Nogueira. It is not uncommon for authors to thank their family and loved ones for shouldering the personal burden and emotional stress of the work, but in this case, the debt runs much deeper. Indeed, early on Joana took a decision that would change both our lives forever, as she joined our experimental group to directly assist in bringing this project to fruition. This work and thesis is every bit hers as it is mine, and I mean it quite literally in every respect, as we built experimental preparations together, performed surgical procedures and took care of animals together, and wrote manuscripts together, often at an exceedingly high burden to our personal life. Producing this manuscript is mostly a victory and testament to her determination and support. I don't know of any greater dedication or demonstration of love than what she has shown me throughout this endeavour. I could not have finished the journey without you by my side and I can only hope that throughout the rest of our lives together I can approximate at least a small percentage of what you have given me these last six years. I am forever yours.
}
\abstract{\label{ch:abstract}

The function of mammalian motor cortex has been a persistent mystery. There is a long history of research linking activity in this part of the brain with the control of ``voluntary'' movements but surprisingly there is an equally large body of evidence in non-human animals describing all kinds of complex behaviours that are \emph{not} impaired when motor cortex is fully removed. What is the reason behind this discrepancy? What kind of movements are actually controlled by motor cortex? This thesis attempts to reconcile the many conflicting views on the cortical control of movement and outline a strategy for investigating the teleology of this brain region.

We start out by introducing a new set of hardware and software tools for neuroscience that aim to make it easier to study in detail more naturalistic motor behaviours in rodents. These tools allow the experimenter to quickly reconfigure the physical and virtual environment of a behaviour task while simultaneously tracking in real-time fine-scale measurements of motor performance.

We then set out to investigate the behaviour of rats facing unexpected or unpredictable motor challenges while navigating dynamic obstacle courses with or without motor cortex. Surprisingly, we found that rats without motor cortex show visible impairments when dealing for the first time with an unexpected motor challenge, despite retaining the ability to skilfully adapt to the new environment with repeated trials.

This observation has led us to propose and discuss a primordial role for motor cortex in extending the robustness of sub-cortical movement systems. Specifically, we suggest that motor cortex is the structure that has helped mammals to conquer those situations that require a succession of rapid and adapted behavioural responses to unexpected environmental change; the kind of resourcefulness that is one of the defining characteristics of mammalian phylogeny.
}
\abstractport{\label{ch:abstractPT}

%\hyphenation{co-nhe-ci-men-to,par-ti-ci-pan-tes}

\begin{center}
\Large \textbf{T\'{i}tulo}
\end{center}

Viagens a v\'{a}rias na\c{c}\~{o}es remotas do mundo.

\begin{center}
\Large \textbf{Resumo}
\end{center}

Gulliver's Travels (1726, alterado em 1735), oficialmente Travels into Several Remote Nations of the World. In Four Parts. By Lemuel Gulliver, First a Surgeon, and then a Captain of Several Ships e traduzido para o português como As Viagens de Gulliver, \'{e} um romance sat\'{i}rico do escritor irland\^{e}s Jonathan Swift. \'{E}E o trabalho mais conhecido de Swift, e tamb\'{e}m um cl\'{a}ssico da literatura inglesa.

}
\contrib{\input{chapters/authorcontrib}}


%%%%%%%%%%%%%%%%%%%%%%%%%%%%%%%%%%%%%%%%%%%%%%%%%%%%%%%%%%%%%\
%%%% Hide some chapters

%%% If you want to produce a pdf that includes only certain chapters, specify them with includeonly, in addition to including all chapters below.
%\includeonly{ch-intro/chapter-intro}
%%% You can also specify multiple chapters.
%\includeonly{ch-intro/chapter-intro,ch-usage/chapter-usage}
%\includeonly{chap1,chap2,chap3}


%%%%%%%%%%%%%%%%%%%%%%%%%%%%%%%%%%%%%%%%%%%%%%%%%%%%%%%%%%%%%
%%%% Notes:

% Footnotes should be placed after punctuation.\footnote{place here.}
% Generally, place citations before the period~\cite{anotherauthor}.
% The proper usage for i.e., and e.g., include commas ``(e.g., option A, option B)''

%%%%%%%%%%%%%%%%%%%%%%%%%%%%%%%%%%%%%%%%%%%%%%%%%%%%%%%%%%%%%

\begin{document}

% % ITQB cover
% 
% Files can be edited in Illustrator
% 
\includepdf[pages={1}]{chapters/Cover.pdf}
\includepdf[pages={1}]{chapters/TitlePage.pdf}

\maketitlepage
\makefrontmatter

% If you've disabled frontmatter, you can insert the toc manually
%\tableofcontents\clearpage

%%%% %%%% Import chapters
%\include lets  us split up the document (and each include starts a new page):

\chapter{Towards a Teleology of Cortical Motor Control}
\epigraph{The infinite fertility of the organism as a field for adapted reactions has become more apparent. The purpose of a reflex seems as legitimate and urgent an object for natural inquiry as the purpose of the colouring of an insect or a blossom. And the importance to physiology is, that the reflex reaction cannot be really intelligible to the physiologist until he knows its aim.}{\textsc{Sir Charles S. Sherrington}, \textit{The Integrative Action of the Nervous System} (1906)}
				
% !TEX root = ../ThesisTemplateCNP.tex
	

% Chapter summary
				
\section{Chapter Summary}

Motor cortex has 150 years of conflicting history. It was originally defined as the part of cortex where movements can be evoked by low-current stimulation. Stimulated points across the cortical surface were found to be organized in rough somatotopy. Cytoarchitecture revealed an expansion of corticofugal layer V and a marked decrease in granular layer IV across this excitable zone. A direct monosynaptic projection system, the pyramidal tract, was found to link cortex to motor neurons in contralateral spinal cord. Lesions of the motor cortex in humans permanently disrupt the execution of fine movements of the digits and distal forelimb. Recordings of neural activity in motor cortex correlate with movement parameters. These lines of evidence support the idea that this part of the brain directly controls movement.

However, lesions of the motor cortex in non-human animals preserve most of the animal's behaviour repertoire. There are multiple parallel descending pathways to spinal centers via brainstem and other subcortical structures. In most mammals, the pyramidal tract does not target motor neurons in ventral spinal cord, but rather dorsal interneurons. Sectioning of the pyramidal tract in primates is sufficient to recapitulate the primary effects of motor cortical lesions, but there are vast projections from motor cortex to other cortical and sub-cortical areas.

This chapter is an attempt to piece together all the fragmentary and contradictory evidence on motor cortical structure and physiology.

\pagebreak




% History of the Motor Cortex
\section{A Dilemma for Cortical Motor Control}

The involvement of the brain and spinal cord in motor control has been recognized since the earliest known clinical records on head and spinal injury, dating back to ancient Egypt \cite{Louis1994,VanMiddendorp2010}. However, the role of the nervous system in generating behaviour was not fully appreciated until Galvani first reported his famous experiments on \textit{animal electricity} \cite{Galvani1791}. By isolating the sciatic nerve and gastrocnemius muscle in the frog, Galvani clearly demonstrated in a series of stimulation experiments that an electrical process, contained entirely within the biology of the frog's leg, was responsible for the spontaneous generation of muscle contractions. This would lead over the following century to the discovery and physiological characterization of the nerve impulse, the action potential, that travels across the nerve to initiate muscle movement \cite{DuBois-Reymond1843,Bernstein1868,Schuetze1983}. The success of these seminal experiments immediately raised a fundamental question regarding nerve conduction: if spontaneous muscle contraction is generated by nerve impulses transmitted throughout the nervous system, how is this transmission coordinated in order to generate the complex patterns of muscle activity observed in natural behaviour?

\subsection{Discovery of the Motor Cortex}

In search of answers to this question, many researchers looked at the brain, the seat of anatomical convergence of the nervous system, for such an integrative role. Following Galvani's footsteps, several attempts were made to stimulate the cerebral cortex electrically, but with little success \cite{Gross2007}. It wasn't until the 1870s that the first indications of a direct involvement of the cortex in the production of movement came to light, around the time when Hughlings Jackson underwent his studies on epileptic convulsions \cite{Jackson1870}. He observed that in some patients the fits would start by a deliberate spasm on one side of the body, and that different body parts would become systematically affected one after the other. He connected the orderly march of these spasms to the existence of localized lesions in the \emph{post-mortem} brain of his patients and hypothesized that the origin of these fits was uncontrolled excitation caused by local changes in cortical \emph{grey matter} \cite{Jackson1870}. In that same year, Fritsch and Hitzig published their famous study demonstrating that it is possible to elicit movements by direct stimulation of the cortex in dogs \cite{Fritsch1870}. Furthermore, stimulation of different parts of the cortex produced movement in different parts of the body \cite{Fritsch1870}. It appeared that the causal mechanism for epileptic convulsions predicted by Hughlings Jackson had been found, and with it a possible explanation for how the normal brain might control movement. The cerebral cortex was already considered at the time to be the seat of reasoning and sensation, so if activity over this so-called \emph{motor cortex} was able to exert direct control over the whole musculature of the body, then it might represent in the normal brain the area that connects volition to muscles \cite{Fritsch1870}.

\subsection{The Goltz-Ferrier Debates}

David Ferrier, a Scottish neurologist deeply impressed by the ideas of Hughlings Jackson and by the positive results of Fritsch and Hitzig's experiments, proceeded to reproduce and expand on their observations with comprehensive stimulation studies showing how activity in the motor cortex was sufficient to produce a large variety of movements across a wide range of mammalian species \cite{Ferrier1873}. Meanwhile, other researchers across Europe such as Goltz and Christiani were facing a dilemma: in many of the so-called ``lower mammals'' massive lesions of the cerebral cortex failed to demonstrate any visible long-term impairments in the motor behaviour of animals \cite{James1885,Goltz1888}.  These two lines of inquiry first clashed at the seventh International Medical Congress held in London in August 1881, where Goltz of Strassburg and Ferrier of London presented their results in a series of debates on the localization of function in the cerebral cortex \cite{Phillips1984,Tyler2000}.

Goltz assumed a clear anti-localizationist position. He advanced that it was impossible to produce a complete paresis of any muscle, or complete dysfunction of any perception, by destruction of any part of the cerebral cortex, and that he found mostly deficits of general intelligence in his dogs \cite{Tyler2000}. Following Goltz's presentation, Ferrier emphasized the danger of generalizing from the dog to animals of other orders (e.g. man and monkey). He then proceeded to exhibit his own lesion results by means of antiseptic surgery in the monkey, describing how a circumscribed unilateral lesion of the motor cortex produced complete contralateral paralysis of the leg. He also produced a striking series of microscopic sections of Wallerian degeneration \cite{Waller1850} of the ``motor path'' from the cortex to the contralateral spinal cord, the crossed descending projections forming the pyramidal corticospinal tract \cite{Tyler2000}.

The debates concluded with the public demonstration of live specimens: a dog with large lesions to the parietal and posterior lobes from Goltz; and from Ferrier, a hemiplegic monkey with a unilateral lesion to the motor cortex of the contralateral side. As predicted, Goltz's dog showed a clear ability to locomote and avoid obstacles and to make use of its other basic senses, while displaying peculiar deficits of intelligence such as failing to respond with fear to the cracking of a whip or ignoring tobacco smoke blown to its face. On the other hand, Ferrier's monkey showed up severely hemiplegic, in a condition similar to human stroke patients. After the demonstrations, the animals were killed and their brains removed. Preliminary observations revealed that the lesions in Goltz's dog were less extensive than expected, particularly on the left hemisphere. Ferrier's lesions on the other hand were precisely circumscribed to the contralateral motor cortex. These demonstrations secured the triumph of Ferrier, who went on to firmly establish the localizationist approach to neurology and the idea of a somatotopic arrangement over the motor cortex.

The Goltz-Ferrier debates had far-reaching implications throughout the entire research community of the time, and the basic dilemma that was presented has sparked controversy and confusion for over a hundred years since \cite{Phillips1984,Lashley1924,DeBarenne1933,Tyler2000,Gross2007}. In the meantime, views of motor cortex have evolved to suggest it plays a role in ``understanding'' the movements of others \cite{Rizzolatti2004}, imagining one's own movements \cite{Porro1996}, or in learning new movements \cite{Kawai2015}, but where are we today regarding its suggested primary role in directly controlling movement?

\subsubsection*{Stimulating motor cortex causes movement; motor cortex is active during movement}

Motor cortex is still broadly defined as the region of the cerebral hemispheres from which movements can be evoked by low-current stimulation, following Fritsch and Hitzig's original experiments in 1870 \cite{Fritsch1870}. Stimulating different parts of the motor cortex elicits movement in different parts of the body, and systematic stimulation surveys have revealed a topographical representation of the entire skeletal musculature across the cortical surface \cite{Leyton1917, Penfield1937, Neafsey1986}. Electrophysiological recordings in motor cortex have routinely found correlations between neural activity and many different movement parameters, such as muscle force \cite{Evarts1968}, movement direction \cite{Georgopoulos1986}, speed \cite{Schwartz1993}, or even anisotropic limb mechanics \cite{Scott2001} at the level of both single neurons \cite{Evarts1968,Churchland2007} and populations \cite{Georgopoulos1986,Churchland2012}. Determining what exactly this activity in motor cortex controls \cite{Todorov2000} has been further complicated by studies using long stimulation durations in which continuous stimulation at a single location in motor cortex evokes complex, multi-muscle movements \cite{Graziano2002,Aflalo2006}. However, as a whole, these observations all support the long standing view that activity in motor cortex is involved in the direct control of movement.

\subsubsection*{Motor cortex lesions produce different deficits in different species}

What types of movement require motor cortex? In humans, a motor cortical lesion is devastating. Permanent injury to the frontal lobes of the brain by stroke or mechanical means is often followed by weakness or paralysis of the limbs in the side of the body opposite to the lesion \cite{Louis1994}. Although the paretic symptoms have a tendency to recover partially by themselves, especially with training and rehabilitation, permanent movement deficits and loss of muscle control in the affected limbs is the common prognosis; movement is permanently and obviously impaired \cite{Laplane1977,Kwakkel2003}. In non-human primates, similar gross movement deficits are observed after lesions, albeit transiently \cite{Leyton1917,Travis1955}. The longest lasting effect of a motor cortical lesion is the decreased motility of distal forelimbs, especially in the control of individual finger movements required for precision skills \cite{Leyton1917,Darling2011}. But equally impressive is the extent to which other movements fully recover, including the ability to sit, stand, walk, climb and even reach to grasp, as long as precise finger movements are not required \cite{Leyton1917,Darling2011,Zaaimi2012}. In non-primate mammals, the absence of lasting deficits following motor cortical lesion is even more striking. Careful studies of skilled reaching in rats have revealed an impairment in paw grasping behaviours \cite{Whishaw1991,Alaverdashvili2008a}, comparable to the long lasting deficits seen in primates, but this is a limited impairment when compared to the range of movements that \emph{are} preserved \cite{Whishaw1991,Kawai2015}. In fact, even after complete decortication, rats, cats and dogs retain a shocking amount of their movement repertoire \cite{Goltz1888,Bjursten1976,Terry1989}. If we are to accept the simple hypothesis that motor cortex is the structure responsible for ``voluntary movement production'', then why is there such a blatant difference in the severity of deficits caused by motor cortical lesions in humans versus other mammals? With over a century of stimulation and electrophysiology studies clearly suggesting that motor cortex is involved in many types of movement, in all mammalian species, how can these divergent results be reconciled?


% Anatomy of the Motor Cortex
\section{Anatomy of the Motor Cortex}

\subsection{Laminar Cytoarchitecture}

\subsection{Descending Projections}

\subsection{Cortico-cortical Projections}


% Feedback control
\section{An Integrative View of the Motor System}

A different approach to the problems of motor control developed initially from studies on the integration of spinal reflexes conducted by the Sherrington school. While many researchers continued to look for the integration of complex movements in higher brain structures like the motor cortex, Sherrington turned instead to systematically characterizing anatomically and physiologically the distribution of efferent \cite{Sherrington1892} and afferent \cite{Sherrington1893a} nerve roots in the spinal cord of multiple species. His goal was to shed light on the so-called \emph{reflex arc}, the nerve pathways involved in muscular reactions like the knee-jerk whereby simple sensory stimuli elicit an immediate, automatic response from the animal, even in the absence of higher brain input \cite{Sherrington1893b}.

Sherrington and his contemporaries studied in detail a number of long and short spinal reflexes\footnote{A reflex action in which a stimulus applied to one region elicits a response in another region is termed a \emph{long spinal} reflex, whereas a reflex reaction where the muscular response happens in the same region as the stimulus is termed a \emph{short spinal} reflex.} in a variety of model organisms under different levels of anesthesia, pharmacological manipulations and spinal transsection \cite{Sherrington1903}. This systematic approach made abundantly clear a number of facts about how the nervous system organizes motor behaviour.

The first one, and perhaps the most striking, is that complex motor responses can be integrated and coordinated even in the complete absence of the brain \cite{Sherrington1906}. While the existence of automatisms and fixed action patterns had been recognized since antiquity, systematic stimulation studies in decerebrate animals quickly revealed that the reflex was far from being a rigid and fixed entity, but was rather adaptive and dynamic. In particular, reflex circuits revealed a much wider range of response characteristics than nerve fibers, which were well known since the time of Galvani to exhibit complete stereotypy in their response to a stimulus under various conditions\footnote{Some unique response characteristics of reflex arc conduction include irreversibility of the direction of conduction; fatigability and refractory period; greater variability of threshold; temporal facilitation with successive stimuli; a weaker correspondence of end-effect with intensity and frequency of the stimulus; and a greater susceptibility to metabolic and pharmacological manipulations \cite[p.14]{Sherrington1906}}.

Indeed, the motor output produced by the massively simplified spinal circuits was remarkably organized and displayed clear ethological meaning: adaptive behaviours such as reflex stepping, standing \cite{Sherrington1910, Sherrington1915}, scratching \cite{Sherrington1903}, or shaking \cite{Goltz1896, Sherrington1903} were all available to be elicited from stimulation of the isolated spinal system. Strikingly, these reflexes were also shown to be deployed and modulated appropriately to specific stimuli. The scratch reflex, for example, carries the foot roughly to the place of stimulation \cite{Sherrington1904}, and in reflex stepping the animal can maintain a rhythmic march through all phases of locomotion over unobstructed surfaces. Integration of these reflexes with input from the telereceptors is obviously entirely absent, but these observations clarified, beyond any reasonable doubt, that spinal cord circuits alone are sufficient to produce and sustain entire behaviour sequences under the right conditions. Furthermore, deafferentation experiments showed that aspects of these rhythmic network motifs persist even in the absence of sensory input \cite{GrahamBrown1911}. Many of these reflex circuits were later termed \emph{central pattern generators}, or CPGs \cite{Grillner1975, Grillner1981}, and found to be present across both vertebrate and invertebrate species \cite{Orlovsky1999,Selverston2010}.

\subsection{The Coordinative Role of Inhibition}

One of the aspects of spinal reflexes that most deeply impressed Sherrington was the general capacity of reflex circuits to initiate and switch between concurrent responses despite the existence of a \emph{final common path} from the nervous system to muscles \cite{Sherrington1904}. Motor neurons in the spinal cord send their axons through the ventral roots of spinal segments to synapse directly on muscle fibres. From his experiments, Sherrington showed that it was common to find multiple motor neurons participating synergistically or antagonistically in a single coordinated reflex response. More importantly, he revealed that the same motor neurons were actually shared among multiple, potentially conflicting, reflex arcs. Sherrington was fascinated by the fact that these antagonistic reflexes, initiated simultaneously from distinct sensory receptors, were still found to be able to coordinate their influence despite sharing this final common path to muscles. That such coordination existed was made clear by stimulation experiments where two or more reflexes were elicited at the same time, generating muscle responses to the combined stimulation that were not a simple summation or linear combination of the responses obtained by stimulation delivered in isolation. Sherrington describes the conception clearly:

\blockquote[{\protect\cite[p.461]{Sherrington1904}}]{Take the primary retinal reflex, which moves the eye so as to bring the fovea to the situation of the stimulating image. From all the receptors in each lateral retinal half rise reflex arcs with a final common path in the nerve of the opposite \emph{rectus lateralis}. Suppose simultaneous stimulation of two of these retinal points, one nearer to, one farther from, the fovea. If the arcs of both points pour their impulses into the final common path together, the effect must be a resultant of the two discharges. If these sum, the shortening of the muscle will be too great and the fovea swing too far for either point. If the resultant be a compromise between the two individual points, the fovea will come to lie between the two points of stimulation. In both cases the result obtained would be useless for the purposes of either\ldots.

When two stimuli are applied simultaneously which would evoke reflex actions that employ the same final common path in different ways, in my experience one reflex appears without the other. The result is this reflex or that reflex, but not the two together.}

In Sherrington's time the existence of such common paths was a problem for the classic view of reflex control, where the function of the nervous system was conceived in terms of nerve conduction of excitatory impulses. The existence of the final common path mediating multiple reflexes made it necessary to speak openly of the problem of how to coordinate different circuit elements and to describe mechanisms that would allow the same neurons to take on context-dependent roles in generating motor responses. It was during the hunt for such a mechanism of reflex arc coordination that Sherrington hit upon the fundamental role of inhibition in the organization of neural function. Inhibition had always been a complicated topic for physiologists, but following the demonstration of cardiac muscle inhibition by the vagus nerve \cite{Weber1846}, and Sechenov's grand proposal of a central origin of reflex inhibition \cite{Sechenov1863}, Sherrington was able to articulate and experimentally validate simple mechanisms for spinal reflex coordination, not only at the level of reciprocal inhibition of antagonistic muscles \cite{Sherrington1893b}, but also at the level of coordination and maintenance of the so-called central state.

Specifically, Sherrington described the existence of a central inhibitory state that was at least as, if not more, important than central excitatory state. He emphasized the need for discrete regulatory mechanisms in the nervous system which were capable of deciding how much converging neural impulses would influence a target cell. From physiological experiments measuring the latency and amplitude of neural responses in very short reflex arcs, he surmised that the effect could not be a simple \emph{absence} of excitation or shutting down of function, but rather the presence of a fundamentally active force in the nervous system \cite{Sherrington1965}. He emphasized that the major difficulty in studies of inhibition was the fact that it can only be actively measured in comparison to a baseline of excitation. If this baseline is not set up accurately, inhibition and its effects can be easy to miss. He pinned down the effects of inhibitory state to a theoretical gap between two neural cells where the nature of conduction changed fundamentally. He named this junction the \emph{synapse}.

Today we know from several detailed descriptions of invertebrate CPGs that different spinal circuits can be delicately super-imposed on the same neuronal elements by using a number of strategies to regulate and coordinate function \cite{Orlovsky1999,Selverston2010}. These strategies span all levels of neural organization from networks to molecules, ranging from reciprocal inhibition network motifs gated by specific neuromodulators and cell membrane receptors to overlapping distributions of voltage and ligand gated ion channels with different kinetics. The existence of such finely tuned spinal networks brings significant constraints when thinking about cortical motor control. Ultimately, any descending signals from the brain to muscles are also sharing this final common path and thus should be expected to require some form of coordination with spinal circuits if behavioural output is to remain integrated.

Another important implication of the idea of central inhibition was how it could explain the physiological and behavioural changes during phenomena of ``shock'' and ``release'', a generalized depression of activity or over-action in a given area following injury or destruction to distant but related parts of the nervous system. Sherrington described how transient changes in excitatory and inhibitory state ultimately manifest themselves depending on the particular anatomical situation of the influencing and influenced centres. In particular, he emphasized how the presence of such transient changes can be enough to establish that two areas are connected, but how these transient effects can obscure the function of either area.

% Give a detailed example of a mechanism of reciprocal innervation explaining coordination
% Talk about the difficulty of measuring inhibition: you need to compare it to a baseline of excitation
% Talk about inhibition as an "active" force in the nervous system. A "pulse" of inhibition is as much an input as a pulse of "excitation".
% Talk about the synapse as the location for the mechanism of inhibition (include stuff on the discovery of the synapse)

% The second one, the final common path and the need of neural coordination at the output
% The third one, inhibition is as active a force in the nervous system as excitation
% The fourth one, response latencies and implications for motor control (and synapses)

However, reproduction of the stimulation experiments across many mammalian species \cite{Ferrier1873,Clark1937} revealed over time a number of subtle observations. First, rather than merely activating individual muscles, prolonged stimulation at individual cortical sites reliably evoked behaviourally relevant, purposeful actions \cite{Ferrier1873,Clark1937}. Second, if the stimulation electrodes were held in the same fixed cortical point, repeated pulse trains could be made to elicit different responses if the configuration of the limbs or head was changed \cite{Ward1938}. This variability in responses to cortical stimulation was much higher than in spinal or decerebrate preparations and was thus termed the ``instability of the cortical point'' \cite{GrahamBrown1912,Leyton1917}.

However, it is important to note that apart from CPG-like network patterns in the spinal cord, there is also a general capacity of spinal circuits as a whole to initiate and switch between the different responses, as well as modulate network activity with incoming sensory input in order to produce adaptive behaviour responses \cite{Forssberg1975}.

\subsection{The Instability of Cortical Points}

One of the most distinguishing physiological characteristics of mammalian motor cortex is the ability to evoke movements in different parts of the skeletal musculature by electrical stimulation of different motor cortical areas. Following the discovery of this rough somatotopy, several researchers developed microstimulation mapping protocols, where low-amplitude, short duration pulses of electrical current are delivered systematically across many different predetermined cortical sites arranged in a grid lattice spanning the entire motor cortex. In this way it was possible to establish the precise somatotopical organization of motor cortex across many different species, including humans.

However, difficulties in interpreting the function of these stimulation fields in the normal behaving animal have been pointed out since the early stimulation experiments. First, it was noted that many of the cortical stimulation points were ``unstable''. Specifically, if the stimulating electrode is kept in the same place and the stimulation protocol repeated, the variability in evoked movements was found to be much larger than what is normally encountered with stimulation of efferent nerve roots at the level of the spinal cord. Second, while microstimulation protocols are able to evoke single muscle twitches, the application of longer stimulation trains can trigger the release of complex, integrated movement sequences which are behaviourally meaningful. In the recent experiments by Graziano, it was demonstrated that not only the evoked movements were all part of the animal's natural behaviour repertoire, but they appeared to be goal-directed and to take into account the current state of the musculature. For example, a reaching action evoked by stimulation of a single cortical point would bring the hand to grasp a specific region in space. If the stimulating electrode was kept in the same place and the arm of the animal moved to a different configuration, the stimulation would now cause different effects in the animal musculature, but which nevertheless had the result of bringing the hand to grasp the same point in space.

Critical revision of any scientific theory requires pushing against the fringe of established fact, which derives often from a certain stubbornness to explain lingering contradictions in a satisfactory manner. It is precisely the appreciation of these contradictions that provides fertile ground for seeding new perspectives of the phenomenon under study. To this end, it is best to state explicitly and concisely the basic assumptions and facts of current theory, so that contrary evidence can be seen to stand out clearly.



Old observations can be reinterpreted in the light of a new perspective which will subsequently predict and inform the results of future experiments. The capacity of the nervous system to fully integrate behaviour acts in the absence of higher brain structures is an example of one of the largest such revisions in systems neuroscience. For over a century, physiologists had explored spinal reflexes using localized stimulation of the endings of nerve fibres.

% One of the major points of contention were the differences between nerve trunk conduction and reflex arc conduction. The speed of propagation of the action potential across a nerve fiber had been known since the experiments of Helmholtz \cite{Helmholtz1850,Schmidgen2002} so it was possible to estimate the expected latency for a short spinal reflex, assuming reasonable estimates of nerve length and mechanical latency. For example, in the flexion-reflex of the dog's hind limb this latency was estimated to be around \SI{27}{\milli\second} \cite[p.19]{Sherrington1906}. However, the observed latencies in the spinal animal were, under normal conditions, found to be at least double of this number. Furthermore, reflex arc conduction had many other characteristics that were distinctive from the well studied nerve trunk conduction. Besides longer latencies, reflex arc conduction exhibited irreversibility of direction of conduction; fatigability and refractory period; greater variability of threshold; temporal facilitation with successive stimuli; a weaker correspondence of end-effect with intensity and frequency of the stimulus; and a greater susceptibility to metabolic and pharmacological manipulations \cite[p.14]{Sherrington1906}.

% These differences made Sherrington realize the necessity for a change in the physical medium of conduction at some point along the arc. The existence of such gaps clashed with the dominant idea at the time which viewed the nervous system as a continuous conductive nerve net. However, with the development of Golgi staining, Ramon y Cajal had recently been able to finally introduce convincingly the idea that the nervous system was actually composed of discrete cellular units, \emph{neurons} \cite{RamonYCajal1894}. Sherrington became a supporter of the newly established neuron doctrine and posited that the required theoretical junction between two neurons, which he termed the ``\emph{synapse}'' \cite{Foster1897}, could explain the unique physiology of reflex arc conduction.

Today we know from several detailed descriptions of invertebrate CPGs that different spinal circuits can be delicately super-imposed on the same neuronal elements by using a number of strategies to regulate and coordinate function \cite{Orlovsky1999,Selverston2010}. These strategies span all levels of neural organization from networks to molecules, ranging from reciprocal inhibition network motifs gated by specific neuromodulators and cell membrane receptors to overlapping distributions of voltage and ligand gated ion channels with different kinetics. The existence of such finely tuned spinal networks brings significant constraints when thinking about cortical motor control. Ultimately, any descending signals from the brain to muscles are also sharing this final common path and thus should be expected to require some form of coordination with spinal circuits if behavioural output is to remain integrated.

\subsubsection*{A role in modulating the movements generated by lower motor centres}

A different perspective on motor cortex emerged from studying the neural control of locomotion, suggesting that the corticospinal tract plays a role in the \emph{adjustment} of ongoing movements that are generated by lower motor systems. In this view, rather than motor cortex assuming direct control over muscle movement, it instead modulates the activity and sensory feedback in spinal circuits in order to adapt a lower movement controller to challenging conditions. This idea that the descending cortical pathways superimpose speed and precision on an existing baseline of behaviour was also suggested by lesion work in primates \cite{Lawrence1968a}, but has been investigated most thoroughly in the context of cat locomotion.

It has been known for more than a century that completely decerebrate cats are capable of sustaining the locomotor rhythms necessary for walking on a flat treadmill utilizing only spinal circuits \cite{GrahamBrown1911}. Brainstem and midbrain circuits are sufficient to initiate the activity of these spinal central pattern generators \cite{Grillner1973}, so what exactly is the contribution of motor cortex to the control of locomotion? Single-unit recordings of pyramidal tract neurons (PTNs) from cats walking on a treadmill have shown that a large proportion of these neurons are locked to the step cycle \cite{Armstrong1984a}. However, we know from the decerebrate studies that this activity is not necessary for the basic locomotor pattern. What then is its role?

Lesions of the lateral descending pathways (containing corticospinal and rubrospinal projections) produce a long term impairment in the ability of cats to step over obstacles \cite{Drew2002}. Recordings of PTN neurons during locomotion show increased activity during these visually guided modifications to the basic step cycle \cite{Drew1996}. These observations suggest that motor cortex neurons are necessary for precise stepping and adjustment of ongoing locomotion to changing conditions. However, long-term effects seem to require complete lesion of \emph{both} the corticospinal and rubrospinal tracts \cite{Drew2002}. Even in these animals, the voluntary act of stepping over an obstacle does not disappear entirely, and moreover, they can adapt to changes in the height of the obstacles \cite{Drew2002}. Specifically, even though these animals never regain the ability to gracefully clear an obstacle, when faced with a higher obstacle, they are able to adjust their stepping height in such a way that would have allowed them to comfortably clear the lower obstacle \cite{Drew2002}. Furthermore, deficits caused by lesions restricted to the pyramidal tract seem to disappear over time \cite{Liddell1944}, and are most clearly visible only the first time an animal encounters a new obstacle \cite{Liddell1944}.

The view that motor cortex in non-primate mammals is principally responsible for adjusting ongoing movement patterns generated by lower brain structures is appealing. What is this modulation good for? What does it allow an animal to achieve? How can we assay its necessity?


% A teleology for Motor Cortex
\section{A Strategy for Probing Cortical Control}

It should now be clear that the involvement of motor cortex in the direct control of all ``voluntary movement'' is human-specific. There is a role for motor cortex across mammals in the control of precise movements of the extremities, especially those requiring individual movements of the fingers, but these effects are subtle in non-primate mammals. Furthermore, what would be a devastating impairment for humans may not be so severe for mammals that do not depend on precision finger movements for survival. Therefore, generalizing this specific role of motor cortex from humans to all other mammals would be misleading. We could be missing another, more primordial role for this structure that predominates in other mammals, and by doing so, we may also be missing an important role in humans.

The proposal that motor cortex induces modifications of ongoing movement synergies, prompted by the electrophysiological studies of cat locomotion, definitely points to a role consistent with the results of various lesion studies. However, in assays used, the ability to modify ongoing movement generally recovers after a motor cortical lesion. What are the environmental situations in which motor cortical modulation is most useful?

Cortex has long been proposed to be the structure responsible for integrating a representation of the world and improving the predictive power of this representation with experience \cite{Barlow1985,Doya1999}. If motor cortex is the means by which these representations can gain influence over the body, however subtle and ``modulatory'', can we find situations (i.e. tasks) in which this cortical control is required?

The necessity of cortex for various behavioural tasks has been actively investigated in experimental psychology for over a century, including the foundational work of Karl Lashley and his students \cite{Lashley1921a,Lashley1950a}. In the rat, large cortical lesions were found to produce little to no impairment in movement control, and even deficits in learning and decision making abilities were difficult to demonstrate consistently over repeated trials. However, Lashley did notice some evidence that cortical control may be involved in postural adaptations to unexpected perturbations \cite{Lashley1921a}. These studies once again seem to recapitulate the two most consistent observations found across the entire motor cortical lesion literature in non-primate mammals since Hitzig \cite{Fritsch1870}, Goltz \cite{Goltz1888}, Sherrington \cite{Sherrington1885} and others \cite{Oakley1979,Terry1989}. One, direct voluntary control over movement is most definitely not abolished through lesion; and two, certain aspects of some movements are definitely impaired, but only under certain challenging situations. The latter are often reported only anecdotally. It was this collection of intriguing observations in animals with motor cortical lesions that prompted us to expand the scope of standard laboratory tasks to include a broader range of motor control challenges that brains encounter in their natural environments.

\subsection{Thesis Outline}

In this work, an attempt to outline a new role for motor cortex is reported. As many previous efforts, it starts with behaviour, and the realization that controlled exposure of animals to a wider range of environments is of absolute necessity to gain insight into the teleology of the system. To this end, we have developed new tools to make it easier to survey a large range of environments while recording as many fine scale measures of behaviour and physiology as possible. These technological developments and methods are described in Chapter \ref{ch:tools}.

In Chapter \ref{ch:behaviour}, a set of behaviour and lesion studies is reported in the rat. These studies had the goal of probing the limits of recovery following extensive cortical lesions by exposing animals to more challenging and dynamic environments. Detailed analysis of the moment by moment behaviour of lesioned animals revealed a number of intriguing observations, the implications of which we discuss in Chapter \ref{ch:conclusions}.



% Lesions of the Motor Cortex
% \section{Lesions of the Motor Cortex}

From the very beginning of brain research, the study of natural and artificial injury to the nervous system of animals has remained a critical means to infer neural function. The early studies of Broca localizing the function of articulate language to a specific region in the cerebral hemispheres \cite{Broca1861} established a long tradition of correlating the location of brain injury with a behavioural disorder. In this section we review the main findings derived from behavioural observation of animal and human subjects with different kinds of lesions to the motor cortex, and highlight the strengths and weaknesses of the various methods.

\subsection{Permanent Lesions}

\subsection{Transient Inactivations}

\subsection{Difficulties of Lesion Studies}


% Physiology of the Motor Cortex
% \section{Neurophysiology of the Motor Cortex}

\subsection{Muscle and Movement Representations}

\subsection{Neural Correlates of Learning}

\subsection{Difficulties of Physiological Studies}

% Theories of the Motor Cortex
% \section{Theories of the Motor Cortex}

\subsection{Optimal Feedback Control}

\subsection{Active Inference}



\chapter{Neurotechnology for Behaviour}
\epigraph{It is not true that ``the laboratory can never be like life.'' The laboratory \emph{must} be like life!}{\textsc{James J. Gibson}, \textit{The Ecological Approach to Visual Perception} (1979)}
				
% !TEX root = ../ThesisTemplateCNP.tex
	

% Chapter summary
				
\section{Chapter Summary}

The design of modern scientific experiments requires the control and monitoring of many different data streams. However, the serial execution of programming instructions in a computer makes it a challenge to develop software that can deal with the asynchronous, parallel nature of scientific data. Here we present Bonsai, a modular, high-performance, open-source visual programming framework for the acquisition and online processing of data streams. We describe Bonsai's core principles and architecture and demonstrate how it allows for the rapid and flexible prototyping of integrated experimental designs in neuroscience. We specifically highlight some applications that require the combination of many different hardware and software components, including video tracking of behavior, electrophysiology and closed-loop control of stimulation.

\pagebreak




% Modular Box
				
\section{Introduction}

The formal study of animal behaviour has a long history spanning hundreds of years across the fields of ethology, experimental psychology and neuroscience. While the ethologists mainly endeavoured to study behaviour in its natural environment, the psychologists and neurophysiologists have classically resorted, of necessity, to more controlled laboratory settings. The reason is mainly one of complexity. Behaviour is a highly multi-dimensional, multi-scale phenomenon that often allows no clear separation between relevant and irrelevant variables \cite{Gomez-Marin2014}. It is in general impossible to predict what an animal is going to do simply because some of the crucial information is not even accessible to measurement. In order to mitigate this problem, neuroscientists resort to making impoverished preparations where the number of variables that are changing at any given moment is low and very carefully controlled. The hope is that in this way the interpretation of brain signals recorded simultaneously with animal behaviour will be facilitated.

Depending on the kind of question a neuroscientist is after, an appropriate behaviour paradigm is set up. Anaesthetized and head-fixed preparations, as well as classical or operant conditioning boxes are regularly employed to drive the behaviour of the animal to oscillate between a set of repeatedly reproducible states more amenable to statistical analysis. Building such behaviour assays often requires very specialized engineering skills and long development cycles of trial and error in order to ensure all the relevant variables are controlled accordingly. Because of this, the tendency of the field has been to concentrate on a small set of ``standardized'' assays which have been shown to work for one area of research or other. Small variations to the standard tasks are gradually introduced in order to probe different aspects of the system. The complexity of behaviour studies in neuroscience has thus traditionally progressed by attrition and painstaking accumulation of small perturbations to overall design patterns.

Interestingly, however, many of the most significant conceptual advances in our understanding of brain function have in fact developed \emph{pari passu} with forays into entirely new behaviour spaces. Moving from anaesthetized to awake physiology completely changed the way we understand the neural processing of sensory stimuli \cite{Sellers2015}. Similarly, moving from head-fixed to freely moving behaviour led to the discovery of place fields in hippocampus \cite{OKeefe1971}. Single trial analysis of simultaneously recorded responses have revealed patterns of neural activity such as hippocampal ripples that are simply impossible to recover from statistical averages of repetitive behaviour episodes \cite{Foster2006,Davidson2009}. Each of these developments has required significant advances in tools used to record and control behavioural data at a fine scale. Unfortunately, the technical cost and scientific risk of trying something novel means that such advances are still much fewer and far between than would be desirable.

From the beginning of this work it was understood that revealing the teleology of cortical control over behaviour would require just this kind of foray into diverse and potentially unknown behaviour spaces. We agreed that it might be worth to try and develop a toolkit for the behavioural neuroscientist that would accelerate the exploration of this vast space. One of the first obvious targets for improvement was the behaviour box. Traditionally, when a given behaviour assay is found to produce interesting results, its design is progressively tweaked so as to exacerbate the features of the original effect. In this work, we started by breaking apart this concept of the polished behaviour box, and wondered what would happen if instead of a standard box, we could have a box of standards.

\section{The Modular Behaviour Box}

At the outset it was decided that the scale of the modular architecture would probably have to match a given animal model, given the vastly different size scales between rodents, cats and primates. Our animal model of choice is the rodent \emph{rattus norvegicus}, and all of our proposed design choices target its size scale. Small adjustments could, however, be reasonably made up to a point for other mammals of similar stature, such as mice.

The main component and interface of the modular box is the individual $1\times 1$ module (Figure \ref{fig:modules}A). This module defines a standardized footprint ($\SI{12}{\centi\meter}\times \SI{12}{\centi\meter}$), against which all other modules are measured. Every newly fabricated module is built to specification to match a multiple of this standardized footprint (e.g. it is possible to have $2\times 1$, $2\times 2$, $4\times 1$ or any other multiple combination of the standard size). Inside the module footprint the module designer places a single logical component of a behaviour box and ensures that it can operate in isolation. Figure \ref{fig:modules} shows some examples of reusable modules developed throughout the project.

\begin{figure}
\centering
\includestandalone[scale=1.00]{chapters/figuresChTools/modules}
\caption{Some examples of standardized behaviour modules. (\textbf{A}) Detail of a $1\times 1$ module mounted in support frame. Fixation is achieved by driving a screw through post-insertion nuts placed in the structural framing (see text). (\textbf{B}) Example reward port module which can be floor- or wall-mounted. All relevant electronics and water distribution circuits are assembled on the back of the module (not shown). (\textbf{C}) Wall-mounted reconfigurable obstacle course stepper module. Stepper motors mounted on the back of the module allow for dynamic reconfiguration of the orientation of each step. (\textbf{D}) Floor-mounted obstacle course step pair. Multiple of these modules can be tiled together to assemble obstacle courses of arbitrary length.}
\label{fig:modules}
\end{figure}

\begin{figure}
\centering
\includestandalone[scale=1.00]{chapters/figuresChTools/box}
\caption{Example of a linear shuttling box assembled from a $\SI{1}{\meter}\times \SI{1}{\meter}$ modular structure using reward port and obstacle course step modules.}
\label{fig:box}
\end{figure}

\begin{figure}
\centering
\includestandalone[scale=0.90]{chapters/figuresChTools/box3d}
\caption{Side view of the linear shuttling box.}
\label{fig:box3d}
\end{figure}

\begin{figure}
\centering
\includestandalone[scale=1.00]{chapters/figuresChTools/verticalBox}
\caption{Example of vertical assembly. (\textbf{A}) Detail of a $1\times 1$ wall-mounted platform module. (\textbf{B}) Example of a vertical maze configuration.}
\label{fig:verticalBox}
\end{figure}

One of the principal requirements for assembling a box is fastening all its components together. By having a standard footprint, it is possible to design a set of regularly spaced mounting points that allows the experimentalist to quickly generate an entirely new configuration by simply swapping modular components inside the box (Figure \ref{fig:box}, \ref{fig:box3d}). For this work, we took advantage of an existing aluminium structural framing system (Bosch Rexroth, DE) to build the common mounting points (Figure \ref{fig:modules}A). Modules are fastened against post-insertion nuts which are able to slide across the whole length of the aluminium rail. Each of the modules is fastened by four screws, one in each corner. In order to ensure modules can be tightly and securely fixed one next to the other, we used a system of regularly spaced double rails (Figure \ref{fig:box}). This gives the frame the flexibility to easily reposition and rearrange individual modules tiling the entire footprint of any arbitrarily large box.

If the support frame is laid out vertically, it is possible to create modular walls of arbitrary dimensions. Some of the modules can be mounted equally well on a vertical or horizontal configuration, such as reward ports (Figure \ref{fig:modules}B). The three-dimensionality of the design has even been exploited to create vertical mazes (Figure \ref{fig:verticalBox}) to great success.

Throughout the project we made the base of every module from \SI{5}{\milli\meter} acrylic pieces. While not an absolute requirement for the design, this choice of plastic material has the advantage that a laser cutter can be used to very quickly produce a large collection of custom-built modules. In addition, patterns can be engraved or cut on the base to provide additional mounting points for hardware embedded in the module. The use of such rapid prototyping fabrication tools alongside with off the shelf available electronic sensors and actuators meant we were able to completely redesign the entire behaviour box, sometimes in a matter of days.


% Bonsai
\input{./chapters/subsectChTools/bonsai}

% Acknowledgements
\section{Acknowledgements}

We thank João Bártolo Gomes for suggesting the name Bonsai; Danbee Kim for early discussions on creating virtual environments for rodents; Joana Nogueira, George Dimitriadis and all the members of the Intelligent Systems Laboratory for helpful discussions and comments on the manuscript. We also thank all the members of the Champalimaud Neuroscience Programme who used Bonsai to setup their data analysis and acquisition experiments and in so doing provided valuable feedback to improve the framework. The research leading to these results has received funding from the European Union's Seventh Framework Programme (FP7/2007-2013) under grant agreement no. 600925 and the Bial Foundation (Grant 190/12). GL is supported by the PhD Studentship SFRH/BD/51714/2011 from the Foundation for Science and Technology. The Champalimaud Neuroscience Programme is supported by the Champalimaud Foundation.



\chapter{Moving with and without Motor Cortex}
\epigraph{I first became sceptical of the supposed path of the conditioned reflex when I found that rats, trained in a differential reaction to light, showed no reduction in accuracy of performance when almost the entire motor cortex, along with the frontal poles of the brain, was removed.}{\textsc{Karl S. Lashley}, \textit{In Search of the Engram} (1950)}
				
% !TEX root = ../ThesisTemplateCNP.tex
	
	
% Chapter summary
				
\section{Chapter Summary}

The role of motor cortex in the direct control of movement remains unclear, particularly in non-primate mammals. More than a century of research using stimulation, anatomical and electrophysiological studies has implicated neural activity in this region with all kinds of movement. However, following the removal of motor cortex, or even the entire cortex, rats retain the ability to execute a surprisingly large range of adaptive behaviours, including previously learned skilled movements. In this chapter we revisit these two conflicting views of motor cortical control by asking what the primordial role of motor cortex is in non-primate mammals, and how it can be effectively assayed. In order to motivate the discussion we present a new assay of behaviour in the rat, challenging animals to produce robust responses to unexpected and unpredictable situations while navigating a dynamic obstacle course. Surprisingly, we found that rats with motor cortical lesions show clear impairments in dealing with an unexpected collapse of the obstacles, while showing virtually no impairment with repeated trials in many other motor and cognitive metrics of performance. Finally, we present the results of a preliminary investigation on the neurobiological basis of robust responses using electrocorticography and report the existence of large amplitude evoked potentials in the rat motor cortex following exposure to occasionally unstable obstacles.

\pagebreak




% Videos

\videolabel{vid:tutorial}
\videolabel{vid:videogame}


% Introduction
\section{Introduction}

In the natural world, an animal must be able to adapt locomotion to any surface, not only in anticipation of upcoming terrain, but also in response to the unexpected perturbations that often occur during movement. This allows animals to move robustly through the world, even when navigating a changing environment. Testing the ability of the motor system to generate a robust response to an unexpected change can be difficult as it requires introducing a perturbation without cueing the animal about the altered state of the world. Marple-Horvat and colleagues built a circular ladder assay for cats that was specifically designed to record from motor cortex during such conditions \cite{Marple-Horvat1993}. One of the modifications they introduced was to make one of the rungs of the ladder fall unexpectedly under the weight of the animal. When they recorded from motor cortical neurons during the rung drop, they noticed a marked increase in activity, well above the recorded baseline from normal stepping, as the animal recovered from the fall and resumed walking. However, whether this increased activity of motor cortex was necessary for the recovery response has never been assayed.


% Methods

\section{Methods}

\subsection{The Modular Behaviour Box}

At the outset it was decided that the scale of the modular architecture would probably have to match a given animal model, given the vastly different size scales between rodents, cats and primates. Our animal model of choice is the rodent \emph{rattus norvegicus}, and all of our proposed design choices target its size scale. Small adjustments could, however, be reasonably made up to a point for other mammals of similar stature, such as mice.

The main component and interface of the modular box is the individual $1\times 1$ module (Figure \ref{fig:modules}A). This module defines a standardized footprint ($\SI{12}{\centi\meter}\times \SI{12}{\centi\meter}$), against which all other modules are measured. Every newly fabricated module is built to specification to match a multiple of this standardized footprint (e.g. it is possible to have $2\times 1$, $2\times 2$, $4\times 1$ or any other multiple combination of the standard size). Inside the module footprint the module designer places a single logical component of a behaviour box and ensures that it can operate in isolation. Figure \ref{fig:modules} shows some examples of reusable modules developed throughout the project.

\begin{figure}
\centering
\includestandalone[scale=0.90]{chapters/figuresChTools/modules}
\caption{Some examples of standardized behaviour modules. (\textbf{A}) Detail of a $1\times 1$ module mounted in support frame. Fixation is achieved by driving a screw through post-insertion nuts placed in the structural framing (see text). (\textbf{B}) Example reward port module which can be floor- or wall-mounted. All relevant electronics and water distribution circuits are assembled on the back of the module (not shown). (\textbf{C}) Wall-mounted reconfigurable obstacle course stepper module. Stepper motors mounted on the back of the module allow for dynamic reconfiguration of the orientation of each step. (\textbf{D}) Floor-mounted obstacle course step pair. Multiple of these modules can be tiled together to assemble obstacle courses of arbitrary length.}
\label{fig:modules}
\end{figure}

\begin{figure}
\centering
\includestandalone[scale=1.00]{chapters/figuresChTools/box}
\caption{Example of a linear shuttling box assembled from a $\SI{1}{\meter}\times \SI{1}{\meter}$ modular structure using reward port and obstacle course step modules.}
\label{fig:box}
\end{figure}

\begin{figure}
\centering
\includestandalone[scale=0.90]{chapters/figuresChTools/box3d}
\caption{Side view of the linear shuttling box.}
\label{fig:box3d}
\end{figure}

\begin{figure}
\centering
\includestandalone[scale=1.00]{chapters/figuresChTools/verticalBox}
\caption{Example of vertical assembly. (\textbf{A}) Detail of a $1\times 1$ wall-mounted platform module. (\textbf{B}) Example of a vertical maze configuration.}
\label{fig:verticalBox}
\end{figure}

One of the principal requirements for assembling a box is fastening all its components together. By having a standard footprint, it is possible to design a set of regularly spaced mounting points that allows the experimentalist to quickly generate an entirely new configuration by simply swapping modular components inside the box (Figure \ref{fig:box}, \ref{fig:box3d}). For this work, we took advantage of an existing aluminium structural framing system (Bosch Rexroth, DE) to build the common mounting points (Figure \ref{fig:modules}A). Modules are fastened against post-insertion nuts which are able to slide across the whole length of the aluminium rail. Each of the modules is fastened by four screws, one in each corner. In order to ensure modules can be tightly and securely fixed one next to the other, we used a system of regularly spaced double rails (Figure \ref{fig:box}). This gives the frame the flexibility to easily reposition and rearrange individual modules tiling the entire footprint of any arbitrarily large box.

If the support frame is laid out vertically, it is possible to create modular walls of arbitrary dimensions. Some of the modules can be mounted equally well on a vertical or horizontal configuration, such as reward ports (Figure \ref{fig:modules}B). The three-dimensionality of the design has even been exploited to create vertical mazes (Figure \ref{fig:verticalBox}) to great success.

Throughout the project we made the base of every module from \SI{5}{\milli\meter} acrylic pieces. While not an absolute requirement for the design, this choice of plastic material has the advantage that a laser cutter can be used to very quickly produce a large collection of custom-built modules. In addition, patterns can be engraved or cut on the base to provide additional mounting points for hardware embedded in the module. The use of such rapid prototyping fabrication tools alongside with off the shelf available electronic sensors and actuators meant we were able to completely redesign the entire behaviour box, sometimes in a matter of days.

\subsection{The Bonsai Framework}

All the results concerning the Bonsai framework have been published as:
\fullcite{Lopes2015a}
\bigskip

Modern scientific experiments crucially depend on the control and monitoring of many parallel streams of data. Multiple measurement devices, from video cameras, microphones, and pressure sensors to neural electrodes, must simultaneously send their data in real-time to a recording system. General purpose digital computers have gradually replaced many of the specialized analog and digital technologies used for this kind of data acquisition and experiment control, largely due to the flexibility of programming and the exponential growth in computing power. However, the serial nature of programming instructions and shared memory makes it a challenge, even for experienced programmers, to develop software that can elegantly deal with the asynchronous, parallel nature of scientific data.

Another challenge arises from the need for software integration. Each hardware vendor provides their own set of drivers and programming interfaces for configuring and acquiring data from their devices. In addition, the growth of the open-source movement has greatly increased the number of freely available technologies for different data processing domains. Integration of these diverse software and hardware components remains a major challenge for researchers.

These difficulties lead to increased development times when setting up an experiment. Moreover, it requires experimenters to pursue specialized training outside their domain of research. This limits the ability to rapidly prototype and try out new designs and can quickly become the factor limiting the kinds of questions that are amenable to scientific investigation.

Here we describe Bonsai, an open-source visual programming framework for processing data streams. The main goal of Bonsai is to simplify and accelerate the development of software for acquiring and processing the many heterogeneous data sources commonly used in (neuro) scientific research. We aim to facilitate the fast implementation of state-of-the-art experimental designs and to encourage the exploration of new paradigms. The framework has already been successfully used for many applications. In the following we will specifically highlight Bonsai's utility in neuroscience for monitoring and controlling a diverse range of behaviour and physiology experiments.

\subsubsection{Architecture}

Scientific data, like the world we live in, is inherently parallel. To monitor this complexity, modern experimenters are often forced to use multiple electronic instruments simultaneously, each with their own independent sampling rates. As data arrives at the acquisition computer, there are two main approaches to log and process these asynchronous data streams. The first approach is to use a polling strategy: a single sequential process in the computer runs a processing loop that goes through each device in sequence and gathers the available data. In this case, data from only one device is being collected and manipulated at any point in time. The second approach is to use an event-driven (reactive) architecture: processes are setup in parallel to collect data from all the devices simultaneously. Whenever new data is available, notifications are sent to the appropriate software routines that collect and process the data as soon as possible. When only a single processor is available, the difference between these two strategies is negligible: only one instruction at a time can be executed by the computer. However, with modern multi-processor cores and dedicated data transfer circuits, the performance difference between the two approaches will significantly influence the throughput of a data acquisition and processing system. Unfortunately, software tools to support and facilitate the “reactive” approach to data stream processing are only just now starting to be adopted and most software systems are still built from the sequential composition of simple program routines. Many of the assumptions of the sequential processing scenario do not scale to handle parallel execution, especially when shared memory and resources are involved.

In recent years, a number of advances in programming languages and software frameworks have tried to make it easier to create complex software applications by composition of asynchronous computing elements \cite{Bainomugisha2013}. Bonsai builds upon these new efforts and aims to extend these developments to the rapid-prototyping domain by introducing a visual programming language for composing and processing asynchronous data streams. Bonsai was developed on top of the Reactive Extensions for the.NET framework (Rx) \cite{MicrosoftOpenTechnologies2014}. Rx provides many built-in operators that transparently deal with the concurrency challenges that inevitably surface when multiple data streams need to be processed and integrated together in a single program. It has become an increasingly popular framework to develop reactive interfaces for next generation mobile and desktop computing platforms, where it is used to handle the growing number of sensors and network communications required by business logic and consumer applications.

Bonsai (via Rx) represents asynchronous data streams using the notion of an observable sequence. An observable sequence represents a data stream where elements follow one after the other. An example would be a sequence of frames being captured by a camera, or a sequence of key presses logged by the keyboard. The name observable simply specifies that the way we access elements in the data stream is by listening to (i.e., observing) the data as it arrives, in contrast with the static database model, in which the desired data is enumerated.

In Bonsai, observable sequences are created and manipulated graphically using a dataflow \cite{Mosconi2000, Johnston2004} representation (Figures \ref{fig:bonsaiInterface}, \ref{fig:bonsaiExamples}A, Supplementary Video 1). Each node in the dataflow represents an observable sequence. Nodes can be classified as either observable sources of data or combinators (Table \ref{tab:bonsaiCategories}). Sources deliver access to raw data streams, such as images from a video camera or signal waveforms from a microphone or electrophysiology amplifier. Combinators represent any observable operator that handles one or more of these sequences. This category can be further specialized into transforms, sinks and other operator types depending on how they manipulate their inputs (Table \ref{tab:bonsaiCategories}). Transforms modify the incoming data elements of a single input sequence. An example would be taking a sequence of numbers and generating another sequence of numbers containing the original elements multiplied by two. Sinks, on the other hand, simply introduce processing side-effects without modifying the original sequence at all. One example would be printing each number in the sequence to a text file. The act of printing in itself changes nothing about the sequence, which continues to output every number, but the side-effect will generate some useful action. Combinators that change, filter or merge the flow of data streams are neither transforms nor sinks, and they are simply referred to by the more general term combinator. The Sample combinator illustrated in Figure \ref{fig:bonsaiExamples}A takes two data sequences and produces a new sequence where elements are sampled from the first sequence whenever the second sequence produces a new value. In this example, we use Sample to extract and save single images from a video stream whenever a key is pressed.

\begin{figure}
\begin{center}
\scalebox{0.7}{\includegraphics[width=\linewidth]{chapters/figuresChTools/bonsaiInterface.png}}
\end{center}
\vspace{-5mm}
\caption{Screenshot of the Bonsai user interface running a video processing pipeline. An example dataflow for color segmentation and tracking of a moving pendulum is shown. Data sources are colored in violet; transform operators in white; sinks in dark gray. The currently selected node (HsvThreshold) is colored in black and its configuration parameters are displayed in the properties panel on the right. Overlaid windows and graphs represent Bonsai data visualizers for the output of individual nodes.}
\label{fig:bonsaiInterface}
\end{figure}

\begin{figure}
\begin{center}
\scalebox{0.5}{\includesvg{chapters/figuresChTools/bonsaiExamples}}
\end{center}
\vspace{-5mm}
\caption{Examples of dataflow processing pipelines using Bonsai. \textbf{(A)} Taking grayscale snapshots from a camera whenever a key is pressed. Top: graphical representation of the Bonsai dataflow for camera and keyboard processing. Data sources are colored in violet; transform operators in white; combinators in light blue; sinks in dark gray. Bottom: marble diagram showing an example execution of the dataflow. Colored tokens represent frames arriving from the camera. Black circles represent key press events from the keyboard. Asterisks indicate saving of images to permanent storage. \textbf{(B)} Dynamic modulation of an image processing threshold using the mouse. The x-coordinate of mouse movements is used to directly set the externalized ThresholdValue property (orange). The updated threshold value will be used to process any new incoming images. \textbf{(C)} Grouping a set of complex transformations into a single node. In the nested dataflow, the source represents incoming connections to the group and the sink represents the group output.}
\label{fig:bonsaiExamples}
\end{figure}

\begin{table}
\begin{center}
\scalebox{0.7}{\includesvg{chapters/figuresChTools/bonsaiCategories}}
\end{center}
\vspace{-5mm}
\caption{List of Bonsai node categories. The color of each Bonsai node serves as a visual aid to identify their role in dataflow processing pipelines. Most of these categories are actually specializations of the very general combinator and are meant to visually depict their specific data processing semantics.}
\label{tab:bonsaiCategories}
\end{table}

A common requirement when designing and manipulating dataflows is the ability to visualize the state of the data at different stages of processing. We have therefore included a set of visualizers to assist debugging and inspection of data elements, including images and signal waveforms (Figure \ref{fig:bonsaiInterface}). These visualizers are automatically associated with the output data type of each node and can be launched at any time in parallel with the execution of the dataflow. Furthermore, it is often desirable to be able to manipulate processing parameters online for calibration purposes. Each node has a set of properties which parameterize the operation of that particular source or combinator (Figure \ref{fig:bonsaiInterface}). This allows, for example, changing the cutoff frequency of a signal processing filter, or setting the name of the output file in the case of data recording sinks. We have also included the possibility of externalizing node properties into the dataflow (Figure \ref{fig:bonsaiExamples}B). Externalizing a property means extracting one of the parameters into its own node in the dataflow, making it possible to connect the output of another node to the exposed property. This allows for the dynamic control of node parameters.

Finally, we have built into Bonsai the ability to group nodes hierarchically. In its simplest form, this feature can be used to encapsulate a set of operations into a single node which can be reused elsewhere (Figure \ref{fig:bonsaiExamples}C). This is similar to defining a function in a programming language and is one of the ways to create new reactive operators in Bonsai. Any named externalized properties placed inside an encapsulated dataflow will also show up as properties of the group node itself. This allows for the parameterization of nested dataflows and increases their reuse possibilities. In addition, encapsulated dataflows are used to specify more complicated, yet powerful, operators such as iteration constructs that allow for the compact description of complex data processing scenarios that can be cumbersome to specify in pure dataflow visual languages \cite{Mosconi2000} (see below).

Bonsai was designed to be a modular framework, which means it is possible to extend its functionality by installing additional packages containing sources and combinators developed for specific purposes. New packages can be written using C\# or any of the.NET programming languages. Python scripts [via IronPython \cite{IronPython2014}] can be embedded in the dataflow as transforms and sinks, allowing for rapid integration of custom code. All functionality included in Bonsai was designed using these modular principles, and we hope to encourage other researchers to contribute their own packages and thereby extend the framework to other application domains. At present, the available packages include computer vision and signal processing modules based on the OpenCV library \cite{Itseez2014}. Drivers for several cameras and interfaces to other imaging and signal acquisition hardware were integrated as Bonsai sources and sinks, including support for Arduino microcontrollers \cite{Banzi2014}, serial port devices and basic networking using the OSC protocol \cite{Wright2003}. Given the specific applications in the domain of neuroscience, we also integrated a number of neuroscience technology packages. The Ephys package, for example, builds on the Open Ephys initiative for the sharing of electrophysiology acquisition hardware \cite{Voigts2013} by providing support for the Rhythm open-source USB/FPGA interface (Intan Technologies, US). Therefore, the next generation tools for electrophysiology can already be used inside Bonsai, the acquired physiology data implicitly integrated with other available data streams and thus easily assembled into a powerful and flexible experimental neuroscience platform.

\subsubsection{Advanced Operators}

The most common application of Bonsai is the acquisition and processing of simple, independent data streams. However, for many modern experiments, basic acquisition and storage of data is often not sufficient. For example, it can be convenient to only record the data aligned on events of interest, such as the onset of specific stimuli. Furthermore, neuroscience experiments often progress through several stages, especially for behavioral assays, where controlled conditions vary systematically across different sessions or trials. In order to enforce these conditions, experiments need to keep track of which stage is active and use that information to update the state of control variables and sensory processing. These requirements often cannot be described by a simple linear pipeline of data, and require custom code to handle the complicated logic and bookkeeping of experimental states. Below we describe a set of advanced Bonsai operators that can be used to flexibly reconfigure data processing logic to cover a larger number of scenarios. These operators and their applications are all built on the single idea of slicing a data stream into sub-sequences, called windows, which are then processed independently and, potentially, in parallel (Figure \ref{fig:bonsaiAdvanced}).

\begin{figure}
\begin{center}
\scalebox{0.5}{\includesvg{chapters/figuresChTools/bonsaiAdvanced}}
\end{center}
\vspace{-5mm}
\caption{Using slicing and window processing combinators in Bonsai.}
\label{fig:bonsaiAdvanced}
\end{figure}

Bonsai provides different combinators that allow the creation of these sub-sequences from any observable data stream, using element count information, timing, or external triggers (Figures \ref{fig:bonsaiAdvanced}A–C). The specific set of operations to apply on each window is described by encapsulating a dataflow inside a SelectMany group, as detailed in the signal processing example of Figure \ref{fig:bonsaiAdvanced}D. The input source in this group represents each of the window sub-sequences, i.e., it is as if each of the windows is a new data source, containing only the elements that are a part of that window. These elements will be processed as soon as they are available by the encapsulated dataflow. Windows can have overlapping common elements, in which case their processing will happen concurrently. The processing outputs from each window are merged together to produce the final result. In the case of Figure \ref{fig:bonsaiAdvanced}D, past and future samples are grouped in windows to compute a running average of the signal through time, necessarily time-shifted by the number of future samples that are considered in the average.

The processing of the elements of each window happens independently, as if there was a new isolated dataflow running for each of the sequences. We can exploit this independence in order to dynamically turn dataflows on and off during an experiment. In the video splitting example of Figure \ref{fig:bonsaiAdvanced}E, we use an external trigger source to chop a continuous video stream into many small video sequences, aligned when the trigger fired. We then nest a VideoWriter sink into the SelectMany group. The VideoWriter sink is used to encode video frames into a continuous movie file. It starts by creating the video file upon arrival of the first frame, and then encoding every frame in the sequence as they arrive. When the data stream is completed, the file is closed. By nesting the VideoWriter inside the SelectMany group, what we have effectively done is to create a new video file for each of the created windows. Whenever a new trigger arrives, a new clip is created and saving proceeds, implicitly parallelized, for that video file.

More generally, we can use this idea to implement discrete transitions between different processing modes, and chain these states together to design complex control structures such as finite state machines (FSMs). FSMs are widely used to model environments and behavioral assays in systems and cognitive neuroscience. One example is illustrated in Figure \ref{fig:bonsaiAdvanced}F, where we depict the control scheme of a stimulus-response apparatus for a simple reaction time task. In this task, there are only two states: Ready and Go. In the Ready state, no stimulus is presented and a timer is armed. Whenever the timer fires, the task transitions into the Go state, and a stimulus is presented. The subject is instructed to press a key as fast as possible upon presentation of the stimulus. As soon as the key is pressed, the system goes back to the Ready state to start another trial. In a FSM, nodes represent states, e.g., stimulus availability or reward delivery, and edges represent transitions between states that are caused by events in the assay, e.g., a key press. In each state, a number of output variables and control parameters are set (e.g., turning on a light) which represent the behaviour of the machine in that state.

In the Bonsai dataflow model, dataflows encapsulated in a SelectMany group can be used to represent states in a FSM (Figure \ref{fig:bonsaiAdvanced}F, bottom). Specifically, a state is activated whenever it receives an input event, i.e., the dataflow nested inside the state will be turned on. The dynamics of the nested dataflow determine the dynamics of the state. In the Go state presented in Figure \ref{fig:bonsaiAdvanced}F, the activation event is used to trigger stimulus onset. In parallel, we start listening for the key press which will terminate the state. Conversely, for the Ready state we would trigger stimulus offset and arm the timer for presenting the next stimulus. An important difference between Bonsai dataflows and pure state machine models is that a dataflow is specified as a directed acyclic graph, i.e., the data stream cannot loop back on itself. However, by taking advantage of the Repeat combinator, we can restart a dataflow once it is completed, allowing us to reset the state machine for the next trial.

Many of the control tasks in experiments have this sequential trial-based structure, which has allowed us to rapidly prototype complex behaviour assays, such as closed-loop rodent decision making tasks, simply by leveraging the flexibility of the data stream slicing operators.

\subsubsection{Alternatives to Bonsai}

Although graphical user interfaces have played a crucial role in the widespread proliferation of computing technology throughout various scientific fields, the majority of these interfaces tend to be applied to relatively narrow domains, such as the operation of a specific instrument. Their goal is often to provide access to all the various configuration parameters of the hardware and to provide basic data acquisition functionality. There is often no opportunity to parameterize or condition the behaviour of the instrument beyond the possibilities presented by the interface, and interconnections with other devices are often limited to simple hardware triggers. The alternative, when available, is to access low-level application programming interfaces (APIs), and program the desired behaviour from scratch.

In the more flexible domains of data analysis, behaviour control and software simulations, the use of more versatile graphical interfaces has become increasingly prevalent. In these scenarios, it is not uncommon to encounter the development of domain-specific languages (DSLs), where graphical building blocks related to the domain of application can be combined together by the user to generate new behaviors, such as the sequence of steps in a psychophysics experiment or a state-machine diagram used to control stimuli and rewards in operant conditioning. While providing more flexibility to the end user, such DSLs are usually not conceived, at their core, to be applied to wildly different domains (e.g., an operant conditioning state machine is not expected to be able to filter continuous electrophysiology signals). In fact, most DSLs will not even allow the user to extend the set of built-in operations. In those that do, the developer may find a customization pit \cite{Cook2007}, where concepts and operations that are within the range of what the DSL can express are easy to develop, whereas tasks that are a little bit outside of the boundaries of the language quickly become impossible or too cumbersome to implement.

As the level of flexibility of a graphical user interface increases, we start to approach the space occupied by general purpose visual programming languages (GPVPL). These are languages that are designed from the outset to be capable of solving problems across a wide variety of domains using a general set of operations. Ideally, the core building blocks of the language will themselves be domain-independent, so that the user can easily apply the same set of operations to the widest possible class of inputs. In order to better illustrate the feel and expressive power of GPVPLs, and to clarify where Bonsai itself is positioned, we will give two examples of popular languages that have succeeded in this niche: LabVIEW \cite{Instruments2014} and Simulink \cite{MathWorks2014}.

LabVIEW is one of the best examples of a GPVPL applied to the design and control of experiments \cite{Elliott2007}. In LabVIEW, users create virtual instruments (VIs) which are composed of a graphical front-panel containing an assortment of buttons, dials, charts and other objects; as well as a back-panel where a flowchart-like block diagram can be used to specify the behaviour of the VI. In this back-panel, nodes and terminal elements can represent hardware components, numerical operations or front-panel objects, which are connected together using virtual wires that specify the flow of data between them. The popularity of LabVIEW grew initially from its support for state-of-the-art data acquisition cards and hardware as well as its data visualization capabilities. The modularity of its architecture also allowed users to quickly develop and implement new nodes within the language itself by using VIs themselves as nodes.

Although the LabVIEW back-panel is a dataflow visual programming language, its execution model tends to follow a polling, rather than event-driven, strategy for dealing with multiple data streams. In order to properly scale this model to the increasing number of available processor cores, LabVIEW has implemented sophisticated code analysis tools that attempt to identify parallelizable portions of block diagrams automatically (Elliott et al., 2007). Once these sections are identified, LabVIEW will automatically generate parallel processes depending on the number of available cores and will manage the bottlenecks in the code accordingly. Although this mitigates the limitations of the sequential polling programming model, it is important to realize that the goal of such automatic parallelization is still to provide the user with a logically synchronized programming model.

Simulink is a popular dataflow visual programming language for modeling, simulating and analyzing multi-domain dynamic systems. It has become extremely popular for modeling response characteristics of control systems, allowing not only for the rapid prototyping of algorithms, but also the automatic generation of microcontroller code for embedded systems. Again, the success of the language stemmed primarily from the flexibility and ease of use of the block diagrams, as well as the number of prebuilt operations and data visualization tools which quickly took care of many crucial but tedious aspects of control systems modeling.

Like LabVIEW, the execution model for Simulink generated code is still based on polling strategies, where ready to execute dataflow nodes are updated in turn as inputs become available. Again, strategies to scale the output of Simulink to multiple cores have been proposed based on analyzing and segmenting the model into parallelizable sections which can be converted into equivalent parallel execution code for microcontrollers \cite{Kumura2012}.

Similar to LabVIEW and Simulink, Bonsai was designed as a general purpose modular language. The core architecture of Bonsai is domain-independent and provides a general framework to compose asynchronous data streams. A general set of composition operators, or combinators, provides support for iteration, segmentation and merging of parallel data streams, as well as other common manipulations on observable sequences. Both the sources of data and available processing operations can be extended within the language itself using nesting of dataflows. Data visualizers and a growing library of data stream acquisition, processing and logging modules are provided to allow rapid prototyping of a large number of different applications.

However, in contrast to LabVIEW or Simulink, Bonsai adopts a very different strategy to implement dataflow execution. Rather than trying to derive a global sequential execution order of dataflow nodes based on the number of active inputs, Bonsai nodes simply react to incoming inputs immediately, without the need to wait for all of them to be active. When multiple observable sequences are present, this allows for a choice of different concurrency composition strategies. Nevertheless, as the result of the composition is an observable sequence itself, such concurrency management can remain functionally isolated from the combinator that is handling the composition. From the point of view of downstream operators, they are simply receiving an observable sequence. There is a tradeoff, of course, that more responsibility for managing the flow of data is passed to the end user, but it also allows for a finer grained control of concurrency that is critical to the specification of parallel applications.

One important caveat of developing asynchronous systems is that debugging can be more difficult in situations where the precise timing and ordering of events is required to reproduce an offending behaviour. In synchronized and sequential execution environments, one can easily go step by step through the precise cascade of transformations that resulted in a problem. In contrast, when multiple processes are executing concurrently, it can be harder to analyze the program flow in a similarly reproducible, deterministic manner. However, it should be noted that this issue is not unique to reactive environments with real asynchronous devices. A sequential polling strategy will be equally deficient in reproducing a particular execution sequence when data from parallel input devices is being accessed.

Another important caveat is that Bonsai currently runs exclusively in Windows operating systems. However, Microsoft has recently open-sourced the execution engine of the.NET framework and will pursue implementations for all the major operating systems (Linux/Mac). This raises the interesting possibility of eventually extending the Bonsai user base into these important platforms.

% Results
\section{Results}

The author permitted to see the grand academy of Lagado.  The academy largely described.  The arts wherein the professors employ themselves.

\begin{figure}
\begin{subfigure}{.5\linewidth}
\centering\includegraphics[width=\columnwidth]{chapters/figuresChBehaviour/noseTrajectoryStable}
\label{fig:noseTrajectoryStable}
\end{subfigure}%
\begin{subfigure}{.5\linewidth}
\centering\includegraphics[width=\columnwidth]{chapters/figuresChBehaviour/noseTrajectoryUnstable}
\label{fig:noseTrajectoryUnstable}
\end{subfigure}\\[1ex]
\begin{subfigure}{\linewidth}
\centering\includegraphics[width=.5\columnwidth]{chapters/figuresChBehaviour/noseTrajectoryRestable}
\label{fig:noseTrajectoryRestable}
\end{subfigure}
\caption{\textbf{Average nose trajectories when crossing the obstacles under different conditions of the shuttling protocol.} Each line represents the average nose trajectory for each individual animal. Line thickness indicates standard error of the mean. Leftward trials were mirrored so that progression is always from left to right. Classification of an animal into jumper or non-jumper was done on the basis of their trajectory in the \emph{unstable} condition and retained throughout.}
\label{fig:noseTrajectory}
\end{figure}

\begin{figure}
\begin{center}
\includegraphics[width=\columnwidth]{chapters/figuresChBehaviour/correlationJumperWeight}
\end{center}
\vspace{-5mm}
\caption{\textbf{Probability of skipping both middle steps in unstable sessions correlated with body weight.} Each dot represents an individual animal. Body weights were taken from the first habituation session before starting the water deprivation protocol. A trial was marked as skipped if neither of the regions of interest in the middle steps were activated by any body part of the animal in that trial.}
\label{fig:correlationJumperWeight}
\end{figure}

\begin{figure}
\begin{subfigure}{.3\linewidth}
\centering\scalebox{0.5}{\inputpgf{chapters/figuresChBehaviour}{averagePostureCa.pgf}}
\label{fig:averagePostureCa}
\end{subfigure}%
\begin{subfigure}{.3\linewidth}
\centering\scalebox{0.5}{\inputpgf{chapters/figuresChBehaviour}{averagePostureCb.pgf}}
\label{fig:averagePostureCb}
\end{subfigure}%
\begin{subfigure}{.3\linewidth}
\centering\scalebox{0.5}{\inputpgf{chapters/figuresChBehaviour}{averagePostureCc.pgf}}
\label{fig:averagePostureCc}
\end{subfigure}
\vspace{-5mm}\\
\begin{subfigure}{.3\linewidth}
\centering\scalebox{0.5}{\inputpgf{chapters/figuresChBehaviour}{averagePostureCd.pgf}}
\label{fig:averagePostureCd}
\end{subfigure}%
\begin{subfigure}{.3\linewidth}
\centering\scalebox{0.5}{\inputpgf{chapters/figuresChBehaviour}{averagePostureCe.pgf}}
\label{fig:averagePostureCe}
\end{subfigure}%
\begin{subfigure}{.3\linewidth}
\centering\scalebox{0.5}{\inputpgf{chapters/figuresChBehaviour}{averagePostureCg.pgf}}
\label{fig:averagePostureCf}
\end{subfigure}
\vspace{-5mm}\\
\begin{subfigure}{.3\linewidth}
\centering\scalebox{0.5}{\inputpgf{chapters/figuresChBehaviour}{averagePostureCk.pgf}}
\label{fig:averagePostureCd}
\end{subfigure}%
\begin{subfigure}{.3\linewidth}
\centering\scalebox{0.5}{\inputpgf{chapters/figuresChBehaviour}{averagePostureLb.pgf}}
\label{fig:averagePostureCe}
\end{subfigure}%
\begin{subfigure}{.3\linewidth}
\centering\scalebox{0.5}{\inputpgf{chapters/figuresChBehaviour}{averagePostureLd.pgf}}
\label{fig:averagePostureCf}
\end{subfigure}
\vspace{-5mm}\\
\begin{subfigure}{.3\linewidth}
\centering\scalebox{0.5}{\inputpgf{chapters/figuresChBehaviour}{averagePostureLe.pgf}}
\label{fig:averagePostureCd}
\end{subfigure}%
\begin{subfigure}{.3\linewidth}
\centering\scalebox{0.5}{\inputpgf{chapters/figuresChBehaviour}{averagePostureLg.pgf}}
\label{fig:averagePostureCe}
\end{subfigure}%
\begin{subfigure}{.3\linewidth}
\centering\scalebox{0.5}{\inputpgf{chapters/figuresChBehaviour}{averagePostureLk.pgf}}
\label{fig:averagePostureCf}
\end{subfigure}
\vspace{-5mm}
\caption{\textbf{Average posture on approach to the manipulated step under stable (green) and unstable (red) conditions.}}
\label{fig:averagePosture}
\end{figure}

\begin{figure}
\begin{subfigure}{.3\linewidth}
\centering\scalebox{0.5}{\inputpgf{chapters/figuresChBehaviour}{averagePostureCh.pgf}}
\label{fig:averagePostureCa}
\end{subfigure}%
\begin{subfigure}{.3\linewidth}
\centering\scalebox{0.5}{\inputpgf{chapters/figuresChBehaviour}{averagePostureCj.pgf}}
\label{fig:averagePostureCb}
\end{subfigure}%
\begin{subfigure}{.3\linewidth}
\centering\scalebox{0.5}{\inputpgf{chapters/figuresChBehaviour}{averagePostureCf.pgf}}
\label{fig:averagePostureCc}
\end{subfigure}
\vspace{-5mm}\\
\begin{subfigure}{.3\linewidth}
\centering\scalebox{0.5}{\inputpgf{chapters/figuresChBehaviour}{averagePostureCi.pgf}}
\label{fig:averagePostureCd}
\end{subfigure}%
\begin{subfigure}{.3\linewidth}
\centering\scalebox{0.5}{\inputpgf{chapters/figuresChBehaviour}{averagePostureLc.pgf}}
\label{fig:averagePostureCe}
\end{subfigure}%
\begin{subfigure}{.3\linewidth}
\centering\scalebox{0.5}{\inputpgf{chapters/figuresChBehaviour}{averagePostureLh.pgf}}
\label{fig:averagePostureCf}
\end{subfigure}
\vspace{-5mm}\\
\begin{subfigure}{.3\linewidth}
\centering\scalebox{0.5}{\inputpgf{chapters/figuresChBehaviour}{averagePostureLj.pgf}}
\label{fig:averagePostureCd}
\end{subfigure}%
\begin{subfigure}{.3\linewidth}
\centering\scalebox{0.5}{\inputpgf{chapters/figuresChBehaviour}{averagePostureLf.pgf}}
\label{fig:averagePostureCe}
\end{subfigure}%
\begin{subfigure}{.3\linewidth}
\centering\scalebox{0.5}{\inputpgf{chapters/figuresChBehaviour}{averagePostureLi.pgf}}
\label{fig:averagePostureCf}
\end{subfigure}
\vspace{-5mm}
\caption{\textbf{Average posture on approach to the manipulated step for the jumper group under stable (green) and unstable (red) conditions.}}
\label{fig:averagePosture}
\end{figure}

% Discussion
\section{Discussion}

In these experiments, we assessed the role of motor cortical structures by making targeted lesions to areas responsible for forelimb control \cite{Kawai2015,Otchy2015}. Consistent with previous studies, we did not observe any conspicuous deficits in movement execution for rats with bilateral motor cortex lesions when negotiating a stable environment. Even when exposed to a sequence of unstable obstacles, animals were able to learn an efficient strategy for crossing these more challenging environments, with or without motor cortex. These movement strategies also include a preparatory component that might reflect the state of the world an animal expected to encounter. Surprisingly, these preparatory responses also did not require the presence of motor cortex.

It was only when the environment did not conform to expectation, and demanded a rapid adjustment, that a difference between the lesion and control groups was obvious. Animals with extensive damage to the motor cortex did not deploy a change in strategy. Rather, they halted their progression for several seconds, unable to robustly respond to the new motor challenge. In an ecological setting, such hesitation could easily prove fatal. Control animals, on the other hand, were able to rapidly and flexibly reorganize their motor response to an entirely unexpected change in the environment.

Our preliminary investigations of the neurophysiological basis of these robust responses with ECoG have revealed the presence of large amplitude evoked potentials in the motor cortex arising specifically in response to an unexpected collapse of the steps during locomotion. Compared with evoked responses obtained from normal stepping under stable conditions (\SI{-100}{\micro\volt} peak at \SI{10}{\milli\second}), these potentials are both much larger (\SI{-300}{\micro\volt}) and delayed in time (peak at \SI{70}{\milli\second}). Still, they preceded any overt behaviour corrections from the animal following the perturbation, as observed in the high-speed video recordings. The onset of these evoked potentials is in the range of the long-latency stretch reflex, which has been suggested to involve a transcortical loop through the motor cortex \cite{Phillips1969,Matthews1990,Capaday1991}. However, the simultaneous complexity and rapidity of adaptive motor responses we observed in control animals is striking, as they appear to go beyond simple corrective responses to reach a predetermined goal and include a fast switch to entirely different investigatory or compensatory motor strategies adapted to the novel situation. What is the nature of these robust responses that animals without motor cortex seem unable to deploy? What do they allow an animal to achieve? Why are cortical structures necessary for their successful and rapid deployment?



\chapter{Conclusions}
				
% !TEX root = ../ThesisTemplateCNP.tex
	

% Chapter summary
				
\section{Chapter Summary}

We propose a new role for motor cortex: extending the robustness of sub-cortical movement systems, specifically to unexpected situations demanding rapid motor responses adapted to environmental context. The implications of this idea for current and future research are discussed.

\pagebreak




% Extended discussion
\section{Discussion}

In these experiments, we assessed the role of motor cortical structures by making targeted lesions to areas responsible for forelimb control \cite{Kawai2015,Otchy2015}. Consistent with previous studies, we did not observe any conspicuous deficits in movement execution for rats with bilateral motor cortex lesions when negotiating a stable environment. Even when exposed to a sequence of unstable obstacles, animals were able to learn an efficient strategy for crossing these more challenging environments, with or without motor cortex. These movement strategies also include a preparatory component that might reflect the state of the world an animal expected to encounter. Surprisingly, these preparatory responses also did not require the presence of motor cortex.

It was only when the environment did not conform to expectation, and demanded a rapid adjustment, that a difference between the lesion and control groups was obvious. Animals with extensive damage to the motor cortex did not deploy a change in strategy. Rather, they halted their progression for several seconds, unable to robustly respond to the new motor challenge. In an ecological setting, such hesitation could easily prove fatal. Control animals, on the other hand, were able to rapidly and flexibly reorganize their motor response to an entirely unexpected change in the environment.

Our preliminary investigations of the neurophysiological basis of these robust responses with ECoG have revealed the presence of large amplitude evoked potentials in the motor cortex arising specifically in response to an unexpected collapse of the steps during locomotion. Compared with evoked responses obtained from normal stepping under stable conditions (\SI{-100}{\micro\volt} peak at \SI{10}{\milli\second}), these potentials are both much larger (\SI{-300}{\micro\volt}) and delayed in time (peak at \SI{70}{\milli\second}). Still, they preceded any overt behaviour corrections from the animal following the perturbation, as observed in the high-speed video recordings. The onset of these evoked potentials is in the range of the long-latency stretch reflex, which has been suggested to involve a transcortical loop through the motor cortex \cite{Phillips1969,Matthews1990,Capaday1991}. However, the simultaneous complexity and rapidity of adaptive motor responses we observed in control animals is striking, as they appear to go beyond simple corrective responses to reach a predetermined goal and include a fast switch to entirely different investigatory or compensatory motor strategies adapted to the novel situation. What is the nature of these robust responses that animals without motor cortex seem unable to deploy? What do they allow an animal to achieve? Why are cortical structures necessary for their successful and rapid deployment?





%\appendix % all chapters following will be labeled as appendices
%\include{ch-appendicies/implementation}
%\include{ch-appendicies/printing}


% Make the bibliography single spaced
\singlespacing


% add the Bibliography to the Table of Contents
\cleardoublepage
\ifdefined\phantomsection
  \phantomsection  % makes hyperref recognize this section properly for pdf link
\else
% include your .bib file

\fi

\bibliographystyle{SupportFilesTemplates/myapacite}
\bibliography{./bibliographyFile}

% % ITQB cover
% 
% Uncomment line and insert cover file as pdf 
% 
\includepdf[pages={1}]{chapters/BackCover.pdf}

\end{document}

