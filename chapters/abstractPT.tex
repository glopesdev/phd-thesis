\label{ch:abstractPT}

%\hyphenation{co-nhe-ci-men-to,par-ti-ci-pan-tes}

\begin{center}
\Large \textbf{T\'{i}tulo}
\end{center}

Uma Função Robusta para o Córtex Motor

\begin{center}
\Large \textbf{Resumo}
\end{center}

Determinar qual a função do córtex motor existente no cérebro dos mamíferos tem sido um mistério que persistiu ao longo do tempo. Existe uma longa história de estudos que ligam a actividade desta parte do cérebro ao controlo de movimentos ``voluntários'' mas, curiosamente, existe uma história igualmente longa de estudos em animais descrevendo uma ampla variedade de movimentos complexos que \emph{não são} afectados com a remoção total do córtex motor. Qual a razão por detrás desta discrepância? Que tipo de movimentos serão realmente controlados pelo córtex motor? Esta tese procura reconciliar as muitas perspectivas existentes sobre o controlo do córtex motor sobre os movimentos e sugerir uma estratégia para investigar a teleologia desta região do cérebro.

Começamos por introduzir um conjunto de ferramentas de \emph{hardware} e \emph{software} para facilitar o estudo detalhado de comportamentos motores em situações naturalistas em roedores. Estas ferramentas permitem ao cientista reconfigurar rapidamente o contexto físico e lógico de uma tarefa comportamental em simultâneo com a medição precisa e em tempo-real de vários parâmetros de performance motora.

De seguida investigamos o comportamento de ratos, com e sem o córtex motor, durante a travessia de um percurso de obstáculos em que eram apresentados novos desafios motores inesperados. Surpreendentemente, observámos que os ratos em que o córtex motor havia sido removido demonstraram dificuldade em lidar pela primeira vez com um desafio motor inesperado, apesar de preservarem a sua capacidade de se adaptar com eficácia ao novo ambiente após sucessivas tentativas.

Esta observação levou-nos a propor e discutir uma possível função primordial para o córtex motor: estender a robustez dos sistemas sub-corticais responsáveis pelo controlo dos movimentos. Especificamente, sugerimos que o córtex motor é a estrutura que permite aos mamíferos ultrapassar situações que requerem uma sucessão de respostas comportamentais rápidas e adaptadas a um novo contexto; uma das capacidades que reconhecemos como característica no reino mamífero.
