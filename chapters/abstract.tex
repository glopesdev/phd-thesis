\label{ch:abstract}

The function of mammalian motor cortex has remained a persistent mystery. There is a long history of research linking activity in this part of the brain with the control of ``voluntary'' movements but surprisingly there is an equally large body of evidence in non-human animals describing all kinds of complex behaviours that are \emph{not} impaired when motor cortex is fully removed. What is the reason behind this discrepancy? What kind of movements are actually controlled by motor cortex? This thesis attempts to reconcile the many conflicting views on the cortical control of movement and outline a strategy for investigating the teleology of this brain region.

We start out by introducing a new set of hardware and software tools for neuroscience that aim to make it easier to study in detail more naturalistic motor behaviours in rodents. These tools allow the experimenter to quickly reconfigure the physical and virtual environment of a behaviour task while simultaneously tracking in real-time fine-scale measurements of motor performance.

We then set out to investigate the behaviour of rats facing unexpected or unpredictable motor challenges while navigating dynamic obstacle courses with or without motor cortex. Surprisingly, we found that rats without motor cortex show visible impairments when dealing for the first time with an unexpected motor challenge, despite retaining the ability to skilfully adapt to the new environment with repeated trials.

This observation has led us to propose and discuss a primordial role for motor cortex in extending the robustness of sub-cortical movement systems. Specifically, we suggest that motor cortex is the structure that has helped mammals to conquer those situations that require a succession of rapid and adapted behavioural responses to unexpected environmental change; the kind of resourcefulness that is one of the defining characteristics of mammalian phylogeny.
