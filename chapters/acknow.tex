% !TEX root = ../Thesis_Sep_2013.tex


\label{acknow}

An impressively large number of people have accompanied me throughout this journey, and to all of them I am deeply and profoundly thankful. All interactions have a meaning we can never recognize and their combined impact ripples into the future unabated. This is true even if I forget to publicly acknowledge some of them, as I'm sure will be the case.

The Champalimaud Neuroscience Programme has been to me a place of deep personal and intellectual transformation. I can barely recognize the person that six years ago set out on the journey from computer science to neuroscience, intrigued by the mysteries of the brain. The efforts of the institute as a whole to bring together people from widely different cultural and academic backgrounds has created a nexus in which interesting personalities cannot help but be forged and tempered. I will never forget the unique opportunity I had to interact with such a large community of scientists from all over the world, and to discuss nearly every possible topic to my utmost satisfaction. These interactions were always done freely and passionately, in the best spirit of scientific companionship, with no regard for hierarchy or rank.

My first dedication goes to my companions of the International Neuroscience Doctoral Programme, Carolina Doran, Simone Lackner, Tiago Marques, Ivo Marcelo, Bruno Miranda, Raimundo Leong, and Gustavo Moreno, who in the year of 2010 embarked with me on this adventure. Together we have shared much more than just our initial training in neuroscience. Even though we all parted to pursue our own individual projects, there were many crucial points during the journey where, even serendipitously, we were still able to support each other as pillars of sanity in the middle of uncontrollable turbulence. It was an honour and a pleasure to have met each one of you and I hope to have contributed back a small inkling of all the inspiration and admiration you have provided me.

Pavel Itskov, Elena Dreosti, Roberto Medina, Scott Rennie, and Samuel Meyler provided continued drive to collaborate on interesting projects, in addition to their friendship. The academic world would be gentler if there were even more opportunities for this kind of short, sweet and honest collaborations. Bassam Atallah and Cindy Poo have diligently introduced me to many of the subtle complications of animal research. If my transition from desk to bench was any success was mainly due to their patience and support in those early days.

I have to give thanks to Alex Gomez-Marin for sharing with me the passion to ask all the embarrassingly simple questions and to keep searching pure hearted for the truth, no matter how deep or how far back in time the answers must be sought. Also to Niccolò Bonacchi and João Frazão for all the time spent in diligent discussions about all kinds of abstract technicalities that almost no one knows or cares about, but that in the end are critical to keep everyone running; especially Frazão for once again offering me the privilege of his unrivalled sharp and constructive criticism. To João Paulo Gomes for the enduring friendship, creative input and influential discussions.

It is hard to endure long without the kind of safe haven and relaxing times which I found in the short stays with my parents, sister and extended family. Even though I was almost invariably lost to my own thoughts, I found irreplaceable solace in these moments of peace which drove some of the most important insights presented in this thesis. I hope you forgive me for all my absent mindedness and know that I love you all very much.

In the members of the Learning Lab, Gustavo Moreno, Rui Azevedo, Sofia Soares, Thiago Gouvêa and Tiago Monteiro, I always found encouraging support at the bench, as well as very critical debate of ideas in our many lab meetings. I know I was always a bit of an outsider intellectually but rest assured that I took into serious consideration each of your many comments and I thank you for the friendship and sharing spirit. 

The Intelligent Systems Lab was quite literally a second home throughout most of the years of this project. To all its members, Joana Neto, João Frazão, Danbee Kim, George Dimitriadis, Lorenza Calcaterra, Pedro Lacerda, and Atabak Dehban I also want to say thank you for the unique companionship and team spirit with which we battled through challenges of any kind. No matter what crazy project was at hand, we were always able to tackle it together, combining our expertise to overcome our individual difficulties. It is not easy to find such equally sharing and devoted spirits that will give so much while expecting so little in return. Thank you for the extraordinary opportunity. Throughout the years there were also many students interning for short projects which helped to shape research in many new weird directions. I want to thank all of them, but in particular to Tim Schröder for a particularly inspiring and fun internship collaboration, where no small measure of work assembling setups, tweaking experiments and trying to make sense of complicated results was achieved.

To the members of my thesis committee, Megan Carey and Leopoldo Petreanu, I want to thank for much needed external perspective and criticism that helped to ground the development of this project. Also to John Krakauer for unparalleled up front criticism and passionate debate about the hardest conceptual questions surrounding the project. His seemingly unlimited energy and pointers to missing literature did much to invigorate me at a time of disappointment and disillusionment.

To Joseph J. Paton for accepting me as a student and taking up a conceptually complicated project. Even though we did not always see eye-to-eye on questions of methodology and interpretation, I have always taken Joe as the highest standard of rigour against which to sharpen otherwise vague and unsupported ideas. Most, if not all, of the deeper search for answers detailed in this thesis were undertaken directly to try and address the many critically keen concerns that were raised in our meetings and discussions. I hope I have succeeded partly in providing some of the much needed justifications.

To Adam R. Kampff for being more than a mentor, a friend and companion; a fresh and inquisitive mind that never detracts from asking the critical questions. In Adam I always found the incredible ability to point out new interesting directions both for experiments and interpretation of results, even when they initially seemed disappointing or confusing. Also for the tremendous capacity to step back and let students develop their own critical thinking and ability to independently do research. It was thanks to this unique style that I was provided with many opportunities to directly defend difficult ideas in front of a large number of other scientists, and this I now see was absolutely fundamental for intellectual growth. Finally, for the unrelenting support and belief in this project, sometimes even when I had myself lost hope.

And finally, to my wife, Joana Nogueira. It is not uncommon for authors to thank their family and loved ones for shouldering the personal burden and emotional stress of the work, but in this case, the debt runs much deeper. Indeed, early on Joana took a decision that would change both our lives forever, as she joined our experimental group to directly assist in bringing this project to fruition. This work and thesis is every bit hers as it is mine, and I mean it quite literally in every respect, as we built experimental preparations together, performed surgical procedures and took care of animals together, and wrote manuscripts together, often at an exceedingly high burden to our personal life. Producing this manuscript is mostly a victory and testament to her determination and support. I don't know of any greater dedication or demonstration of love than what she has shown me throughout this endeavour. I could not have finished the journey without you by my side and I can only hope that throughout the rest of our lives together I can approximate at least a small percentage of what you have given me these last six years. I am forever yours.
