\section{Theories of the Motor Cortex}

The increasing number of experiments involving the motor cortex has exposed a very complex, heterogeneous set of observations and data that demand explanation. In response to this growing complexity regarding motor cortical circuits and function, dominant theories of motor cortical function were forced to evolve from the simple ``keyboard of the mind'' model of the late 19th century to more dynamical models that are able to account for the results of physiological, anatomical and behavioural studies.

\subsection{Optimal Feedback Control}

Optimal feedback control takes ideas from control theory and interprets the function of motor cortex as a goal-directed servomechanism. In this view, high-level goals computed in pre-motor areas establish reference signals specifying desired states. Motor cortical structures would then prescribe optimal control policies mapping the current state to actions that bring that state closer to the target reference state.

One difficulty with models of simple servomechanisms is their response latency due to delays in feedback and effector output. In order to get over these limitations, optimal feedback control posits the existence of forward models which are able to compute the sensory consequences of motor commands. These models allow the prediction of expected sensory input ahead of time, thereby avoiding sensory delays in the process of minimizing control error signals. Actual sensory signals can be used for applying online corrections to the forward model in order to ensure that it remains valid in the face of changing environmental conditions.

\subsection{Equilibrium-Point Control}



\subsection{Active Inference}

Active inference extends the ideas of predictive coding to motor control. Predictive coding is a hierarchical theory of cortical neural responses inspired on principles of compressed sensing and bayesian inference. According to predictive coding, spiking activity in higher cortical areas represents predictions about the activity in lower areas. These predictions are conveyed through diffuse feedback projections to the top layers of the lower areas and compared to the incoming responses from lower levels. Errors in the prediction pass through to the higher areas via feedforward projections and used to update subsequent predictions.

\subsection{Motor Learning}

\subsection{Dexterous Control}
