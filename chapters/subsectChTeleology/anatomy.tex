\section{Comparative Anatomy of the Descending Pathways}

Anatomy has for a long time served as a powerful source of inspiration for drawing speculative statements on the function of biological systems, and the case of cortical motor control is no different. Indeed, it did not go unnoticed for a day that if electrical stimulation of the cortical surface produced muscle responses, these must be due to the huge pyramidal tract, the same way as Galvani stimulated the frog's legs.

The significant anatomical groundwork on the long fiber tracts descending from the brain to the spinal cord, made at the turn of the century by Flechsig, von Monakow and many others \cite{Nathan1955}, contributed much to the interpretation of the functional results presented at the Goltz-Ferrier debates. Questions such as why are motor movements evoked by surface stimulation of the precentral gyrus, or why do cats and dogs retain most of their behaviour repertoire following large cortical lesions were interpreted in the light of the anatomy of the projection fibers running from the cerebrum to the spinal gray matter, their termination patterns in the cord, and distribution of the cells of origin.

Charles Sherrington himself started out his work by tracing spinal cord degeneration over large periods of time (up to 11 months) following cortical lesions in Goltz's dogs \cite{Langley1884,Sherrington1885}. From these examinations, he became the first to observe in the dog the presence of a degenerated ``re-crossed'' pyramidal tract that travels down the cord ipsilateral to the side of the lesion \cite{Sherrington1885}. These fibers would later come to be called the ipsilateral, ventral corticospinal tract, and have since been found and described in most mammalian species \cite{Kuypers1981,Brosamle2000,Lacroix2004}. He also had the chance during this time to observe first hand the negative effects of cortical lesions reported by Goltz in a variety of specimens. In his own words:

\blockquote[{\protect\cite[p.189]{Sherrington1885}}]{That the pyramidal tracts are in the dog requisite for volitional~impulses to reach limbs and body seems negatived by the fact that the animal can run, leap, turn to either side, use neck and jaws, \&c. with ease and success after nearly, if not wholly, complete degeneration of these tracts on both sides. Further, after complete degeneration of one pyramid, there is in the dog no obvious difference between the movements of the right and left sides.}

Interestingly, he does note that \enquote{defect of motion is observable only as a clumsiness in execution of fine movements} \cite{Sherrington1885}, hinting at ideas that are today still part of broadly accepted theories for the role of the corticospinal tract in motor control.

With the development of Golgi staining, Ramon y Cajal was finally able to introduce convincingly the idea that the nervous system was actually composed of discrete cellular units, \emph{neurons} \cite{RamonYCajal1894}. Sherrington became a supporter of the newly established neuron doctrine and posited that the required theoretical junction between two neurons, which he termed the ``\emph{synapse}'' \cite{Foster1897}, could explain the unique physiology of reflex arc conduction.

Since the early observations of its laminar organization \cite{Baillarger1840}, the intricate anatomy of the cortex has always fascinated neuroscience researchers. Following the first detailed analyses of cellular composition across the entire mammalian cortex \cite{Campbell1905,Brodmann1909}, it became clear that a number of recurring patterns were present in its morphology. Nevertheless, the consistency of the pattern also made it very easy to identify certain systematic differences in the structure of the cortical fields, specifically in the motor cortex.

\begin{figure}
\begin{center}
\includegraphics[width=\columnwidth]{chapters/figuresChTeleology/descendingTaxa}
\end{center}
\vspace{-5mm}
\caption{Forebrain motor control pathways across different vertebrate taxa. The molecular divergence times between human (primate), rodent and lamprey groups \protect\cite{Kumar1998} are noted above a schematic view of the major divisions in the vertebrate brain. Arrows indicate the descending monosynaptic projections identified in each group from motor regions of the forebrain pallium to lower motor centres. Note the specialized monosynaptic projection directly targeting spinal motor neurons in human. MLR, Mesencephalic Locomotor Region; M, Motor Neurons.}
\label{fig:descendingTaxa}
\end{figure}

\subsection{Laminar Cytoarchitecture}

Anatomically, there is a clear histological signature of mammalian motor cortex: the loss of granular layer IV. In the canonical layered arrangement of cortical tissue there is a dense layer of cell bodies known as the granular layer that separates the superficial and deep layers. In primary motor cortex this layer is dramatically reduced, earning it the name of agranular cortex. This loss of granular layer seems to be tightly correlated with expansion of the deep layers, especially layer V.

Functionally, layer IV is thought to be the target of feed-forward projections from lower areas in hierarchical models of cortical organization. This view is based on extrapolation from anatomical studies in primary sensory areas where layer IV receives most of the projections from the thalamus, making it the first layer of the cortex to receive direct sensory input. Conversely, layer V is considered the main source of efferents from cortex to subcortical structures.

\subsection{Descending Projections}

Another anatomical landmark of mammalian motor cortex is the presence of a large number of myelinated fibers that descend monosynaptically to the spinal cord. These form the so-called corticospinal, or pyramidal, tract. There is cross over, or decussation, of the tract at the level of the medulla oblongata, which brings the nerve impulses generated by motor cortex mostly to contralateral spinal cord. There is also a smaller bundle of fibres that descend ipsilaterally.

Apart from these direct projections from motor cortex to the spinal cord, there are also many other privileged routes from motor cortex to movement controlling centers in the brainstem. These are also thought to modulate many behaviours via direct brainstem projections to the spinal cord, such as the rubrospinal, tectospinal, vestibulospinal and reticulospinal tracts.

\subsection{Cortico-cortical Projections}

Despite the predominance of descending projections from motor cortex, in recent years a number of studies have revealed the full extent of direct motor cortical outputs, many of them targeting primary sensory areas.

\subsubsection*{There are anatomical differences in corticospinal projections between primates and other mammals}

In primates, the conspicuous effects of motor cortical lesion can also be induced by sectioning the corticospinal tract, the direct monosynaptic projection that connects motor cortex, and other cortical regions, to the spinal cord \cite{Tower1940,Lawrence1968}. In monkeys, and similarly in humans, this pathway has been found to directly terminate on spinal motor neurons responsible for the control of distal muscles \cite{Leyton1917,Bernhard1954} and is also thought to support the low-current movement responses evoked by electrical stimulation of the cortex, as evidenced by the increased difficulty in obtaining a stimulation response following section at the level of the medulla \cite{Woolsey1972}.

However, the corticospinal tract is by no means the only pathway from cortex to movement (Figure \ref{fig:descendingTaxa}). Motor cortex targets many other brain regions that can themselves generate movement. In fact, this specialized connection from telencephalon to spinal cord appeared only recently in vertebrate evolution \cite{TenDonkelaar2009}, and was further elaborated to include a direct connection from cortex to motor neurons only in some primate species and other highly manipulative mammals such as raccoons \cite{Heffner1983}. In all other mammals, including cats and rats, the termination pattern of the corticospinal tract largely avoids the ventral motor neuron pools in the spinal cord and concentrates instead on intermediate zone interneurons and dorsal sensory neurons \cite{Kuypers1981,Yang2003}. Why then is there such a large dependency on this tract for human motor control? One possibility is that the rubrospinal tract---a descending pathway originating in the brainstem and terminating in the intermediate zone---is degenerated in humans compared to other primates and mammals \cite{Nathan1955,Nathan1982}, and is thought to play a role in compensating for the loss of the corticospinal tract in non-human species \cite{Lawrence1968a,Zaaimi2012}.

It thus seems likely that most mammals rely on ``indirect'' pathways to convey cortical motor commands to muscles. These differences in anatomy might explain the lack of conspicuous, lasting movement deficits following motor cortical lesion in non-primates, but leaves behind a significant question: what is the motor cortex actually controlling in all these other mammals?

\subsubsection*{What is the role of motor cortex in non-primate mammals?}

In the rat, a large portion of cortex is considered ``motor'' based on anatomical \cite{Donoghue1982}, stimulation \cite{Donoghue1982,Neafsey1986} and electrophysiological evidence \cite{Hyland1998}. However, the most consistently observed long-term deficit following motor cortical lesion has been an impairment in supination of the wrist and individuation of digits during grasping, which in turn impairs reaching for food pellets through a narrow vertical slit \cite{Whishaw1991,Alaverdashvili2008a}. Despite the fact that activity in rodent motor cortex has been correlated with movements in every part of the body (not just distal limbs) \cite{Hill2011,Erlich2011}, it would appear we are led to conclude that this large high-level motor structure, with dense efferent projections to motor areas in the spinal cord \cite{Kuypers1981}, basal ganglia \cite{Turner2000,Wu2009}, thalamus \cite{Lee2008}, cerebellum \cite{Baker2001} and brainstem \cite{Jarratt1999}, as well as to most primary sensory areas \cite{Petreanu2012,Schneider2014}, evolved simply to facilitate more precise wrist rotations and grasping gestures. Maybe we are missing something. Might there be other problems in movement control that motor cortex is solving, but that we may be overlooking with our current assays?

