\section{Comparative Anatomy of the Descending Pathways}

It must have felt uncanny to those early researchers to find that surface stimulation of the cortex produces discrete muscle responses, in a way so similar to what Galvani did with the frog's leg. Indeed, Sherrington himself conveys the feeling clearly in the opening of his seminal lecture on the motor cortex \cite[p.271]{Sherrington1906}, confessing ``that although it is not surprising that such territorial subdivision of function should exist in the cerebral cortex, it is surprising that by our relatively imperfect artifices for stimulation we should be able to obtain clear evidence thereof.''

Of course, it did not go unnoticed that this fact might be due to the massive projection from cortex to the spinal cord, which had been fully traced by Türck only twenty years before Fritsch and Hitzig's experiment \cite{Nathan1955}. This so-called ``pyramidal'' tract\footnote{The name ``pyramidal'' is derived from the fact that the tract passes through the medullary pyramids, a pair of white matter structures in the brainstem with a roughly pyramidal shape.} was found to originate in the anterior regions of the cerebral cortex and terminate directly in the lateral columns of the spinal cord after decussating (i.e. crossing over) at the level of the brainstem's \emph{medulla oblongata}. The presence of this corticospinal tract presented compelling visual evidence of the means by which the motor cortex might be able to exert such a direct influence on movement by electrical conduction of nerve impulses, but the underlying biological mechanism remained elusive.

\subsection{A Functional Theory for the Motor Cortex}

Only four years after the discovery of the motor cortex, the Ukrainian anatomist and histologist Vladimir Betz connected for the first time the macroscopic cerebral organization and function proposed by Hitzig and Ferrier with unique, detailed histological evidence of cells found in the motor region, in his remarkably insightful 1874 publication:

\blockquote[{\protect\cite{Betz1874, Kushchayev2012}}]{Such consistency in the region where these cells can be found, manifested as a very definitive cortical layer, as well as in a specific cerebral convolution, prompted me to devote my attention to that particular part of the animal brain, mainly the dog’s, in which Fritsch and Hitzig achieved such brilliant physiological results, i.e. the lobe which borders the cruciate sulcus. I now found such cells of the same shape and in exactly the same position in nests in the dog, precisely in the lobe just mentioned. So in the dog, as well as in man, they are imbedded in the fourth cortical layer and occur only in this lobe and in the anterior half of the posterior (postcentral) convolution bordering it. In the dog, they are somewhat smaller, but nevertheless are the largest in its entire nervous system. They also possess two large and many small processes, and the inner process runs into a genuine nerve filament. In the area where they are found there are also many axis cylinders visible in the white substance, which run in the same direction as in the human. Undoubtedly these cells have all the attributes of the so-called ‘motor cells’ and very definitely continue as cerebral nerve fibres.}

Furthermore, in the same article he distinguishes between sensory and motor poles in the brain, placing the division in the central sulcus: ``The sulcus of Rolando divides the cerebral surface into two parts; an \emph{anterior} in which the large pyramidal nerve cells predominate, and a \emph{posterior}---including the temporal lobes---in which the cell layers are the same'' \cite{Betz1874,Clarke1996}.

In this way, Betz founded the hypothesis that these cells, which he called ‘giant pyramids’ were the cells of origin of the corticospinal tract, and that it were their impulses propagating down to the spinal cord that initiated the muscle responses evoked by electrical stimulation of the cortical surface. His early assignment of these giant cells to cortical layer four was essentially correct, although his layer four would today be considered layer five due to refinement of the total number of layers in cortical tissue.

When the landmark works of Campbell and Brodmann elucidated in detail the cytoarchitectural features of the mammalian cortex, they both included extensive treatments of the pre-central ‘motor’ region and its unique anatomical arrangement \cite{Campbell1905,Brodmann1909}. In regard to the cells of Betz, Campbell in particular provided great clarification on their pattern of distribution and possible physiological function. After extensive histological examination of cortical tissue in the brains of the anthropoid ape and normal human subject, as well as examination of pathological material from cases of Amyotrophic Lateral Sclerosis and from patients that underwent amputation of a body part, he proposed in his monograph a histological basis for assigning the influence of different areas of the pre-central region to different muscle groups, suggesting that ``the largest (motor) cells are exactly those whose impulses have to travel down to the muscles of the lower extremity, those for the arm being quite a third smaller'' \cite[p.33]{Campbell1905}. He argued on the basis of similarity in anatomical configuration and retrograde degeneration studies that most likely even smaller pyramidal cells can form part of the corticospinal tract, becoming one of the first to suggest that a classification of Betz cells based purely on size was probably erroneous.

In addition to the characterization of the ‘motor’ cortex by its prominent layer V containing giant pyramidal cells, both Campbell and Brodmann noted clearly that the pre-central region showed a dramatic reduction in granular layer IV\footnote{For this reason, the frontal ‘motor’ cortices are also sometimes referred to as \emph{agranular} cortex.}. In the canonical layered arrangement of cortical tissue, layer IV is a dense layer of cell bodies known as the granular layer that separates superficial from deep layers. Functionally, layer IV is described as the target of feed-forward projections from sensory thalamus, making it the first layer of the cortex to receive direct sensory input. The observation that this layer was dramatically reduced in the pre-central motor region provided further suggestive evidence that the frontal cortices were more concerned with ‘output’ than with ‘input’.

The last remaining question was to explain how surface stimulation of the cortex was so effective if the subcortically projecting cells were located in the deep layers. This last piece of the puzzle was finally resolved with the development of the Golgi stain and the characterization of the full structure of the cortical pyramidal cell presented in Ramon y Cajal's anatomical masterpiece \cite{RamonYCajal1894,RamonYCajal1909}. In his work, Cajal astutely hypothesized that ‘centrifugal’ excitation originated in the tufts of the long apical dendrite, which extends vertically from the soma of deep pyramidal cells all the way into the superficial layers \cite{RamonYCajal1909}. A complete theory for the direct cortical control of movement thus came into being (Figure \ref{fig:cajalPathway}).

\begin{figure}
\begin{center}
\includegraphics[width=0.5\columnwidth]{chapters/figuresChTeleology/cajalPathway}
\end{center}
\vspace{-5mm}
\caption{Schematic of the double motor pathway. A, motor area of the cerebral cortex; B, pontine protuberance and its collaterals; C, Purkinje cells of the cerebellum; D, crossed pyramidal pathway; E, cerebellomedullary pathway of Marchi; F, ventral or motor roots; G, pontocerebellar pathway or middle cerebellar peduncle. The arrows indicate the direction of impulses \protect\cite{RamonYCajal1909}.}
\label{fig:cajalPathway}
\end{figure}

\subsection{The Effects of Lesions in the Corticospinal Tract}

In the wake of the Goltz-Ferrier debates, investigations of the role of the direct corticospinal descending pathway were conducted in multiple animal species. Sherrington himself started out his work by tracing spinal cord degeneration over large periods of time (up to 11 months) following cortical lesions in Goltz's dogs \cite{Langley1884,Sherrington1885}. He confirmed that many of the properties of the corticospinal tract in the primate held for the dog, and furthermore became one of the first to observe the presence of a degenerated ``re-crossed'' pyramidal tract that travels down the cord ipsilateral to the side of the lesion \cite{Sherrington1885}. These fibers would later come to be called the ipsilateral, ventral corticospinal tract, and have since been found and described in most mammalian species as forming roughly 10\% of the entire corticospinal projections \cite{Kuypers1981,Brosamle2000,Lacroix2004}. However, he also had the chance during this time to observe first hand the negative effects of corticospinal degeneration following lesion, which had been previously reported by Goltz and others in a variety of non-primate specimens. In his own words:

\blockquote[{\protect\cite[p.189]{Sherrington1885}}]{That the pyramidal tracts are in the dog requisite for volitional~impulses to reach limbs and body seems negatived by the fact that the animal can run, leap, turn to either side, use neck and jaws, \&c. with ease and success after nearly, if not wholly, complete degeneration of these tracts on both sides. Further, after complete degeneration of one pyramid, there is in the dog no obvious difference between the movements of the right and left sides.}

Interestingly, he does note that \enquote{defect of motion is observable only as a clumsiness in execution of fine movements} \cite{Sherrington1885}. These observations once again stood out in stark contrast with lesion experiments reported by Ferrier in the monkey, where cauterization of specific motor cortical areas produced complete and persistent paralysis of the corresponding body parts \cite{Ferrier1884}. Years later, Sherrington would revisit these experiments and reported what were apparently contradictory observations of dramatic recovery following similar lesions to the arm area \cite{GrahamBrown1913,Leyton1917}. Furthermore, not only was a near complete recovery reported, but the focus of such recovery could not be obviously traced to other brain areas either in the neighbourhood of the lesion or in the corresponding cortex of the opposite hemisphere. This was verified both by stimulation of the perilesional and contralesional cortex, which failed to evoke movements in the affected limb \cite{Leyton1917}, and by subsequent ablation of these areas. Although the lesion of the intact hemisphere did produce a temporary impairment in the use of the other limb, both of these new lesions failed to reinstate the original deficit \cite{GrahamBrown1913,Leyton1917}. 

One of the limitations of early studies was the lack of quantitative assessment of motor function. Later studies introduced new behavioural assays such as the matchbox or dexterity board as a way to more formally test motor recovery \cite{Glees1950,Cole1952}. For example in the dexterity board, monkeys are tasked to pick morsels of food from differently sized round holes placed in a grid as a way to test for prehensile strength and fine digit control. Using such measures it was found that circumscribed lesions of the thumb and index areas produced permanent deficits of weakness and loss of fine movements despite an early recovery period \cite{Glees1950}. The stimulation experiments of Sherrington \cite{Leyton1917} were also revisited and it was found that motor cortical areas adjacent to the lesion were now able to evoke movements of the impacted digits \cite{Glees1950}. Subsequent removal of these areas reinstated the original motor deficit \cite{Glees1950}. An important difference to emphasize between these two experiments, however, is that only relatively circumscribed motor cortical legions were removed in every surgery, whereas in the original Sherrington study the entire elbow, wrist, index, thumb and remaining digit areas of M1 were excised at once \cite{Leyton1917}.

In the hopes of clarifying the confusion of which exact movements were controlled by cortex, later studies focused on lesions restricted to the corticospinal tract, using both unilateral and bilateral section at the level of the medullary pyramids \cite{Tower1940,Lawrence1968,Lawrence1968a}. The goal was to isolate the effects of all the individual descending pathways to the spinal cord and resolve once and for all the question of whether the corticospinal tract of the motor cortex was the source of all ``voluntary'' movements. Sarah Tower was the first to describe in detail the results of unilateral and bilateral pyramidotomy in primates, with and without lesion of the motor cortex \cite{Tower1940}. She summarized the condition as ``hypotonic paresis'', characterized by a loss of skeletal muscle tone and depression of the vasomotor system, along with general weakening of the reflexes involving the affected limb segments. Although all discrete usage of the hand and digits was eliminated, she did emphasize the clear presence of voluntary movements in the various purposeful compensations produced by the animals to deal with the affliction. Tower attributed these compensations to the preserved capacities of brainstem circuits. A more definitive study to dissociate the effects of direct corticospinal and indirect brainstem descending pathways was conducted by Lawrence and Kuypers, and presented in their now classical publications \cite{Lawrence1968,Lawrence1968a}.

Using the dexterity board, they observed that while normal monkeys routinely pick up the food by pinching individual bits with their fingers, monkeys with bilateral corticospinal lesions were mostly unable to perform this precise pincer movement, and instead employed coarser compensatory clasping strategies to retrieve the food. In addition, lesioned monkeys were consistently reported to be somewhat slower and less agile than normal animals. However, most of their overall movement repertoire was surprisingly preserved. Their final conclusions fit remarkably well with the initial observations of Sherrington in the dog, suggesting that the corticospinal pathways superimpose speed and agility on subcortical mechanisms, and provide the capacity for fractionation of movements such as independent finger movements \cite{Lawrence1968}. These observations also recapitulate the effects of motor cortical lesions reported by Sherrington, but remain at odds with the primary stated role involving motor cortex, and the direct corticospinal tract, with the control of all voluntary movements.

\subsubsection*{There are anatomical differences in corticospinal projections between primates and other mammals}

In primates, the conspicuous effects of motor cortical lesion can also be induced by sectioning the corticospinal tract, the direct monosynaptic projection that connects motor cortex, and other cortical regions, to the spinal cord \cite{Tower1940,Lawrence1968}. In monkeys, and similarly in humans, this pathway has been found to directly terminate on spinal motor neurons responsible for the control of distal muscles \cite{Leyton1917,Bernhard1954} and is also thought to support the low-current movement responses evoked by electrical stimulation of the cortex, as evidenced by the increased difficulty in obtaining a stimulation response following section at the level of the medulla \cite{Woolsey1972}.

However, the corticospinal tract is by no means the only pathway from cortex to movement (Figure \ref{fig:descendingTaxa}). Motor cortex targets many other brain regions that can themselves generate movement. In fact, this specialized connection from telencephalon to spinal cord appeared only recently in vertebrate evolution \cite{TenDonkelaar2009}, and was further elaborated to include a direct connection from cortex to motor neurons only in some primate species and other highly manipulative mammals such as raccoons \cite{Heffner1983}. In all other mammals, including cats and rats, the termination pattern of the corticospinal tract largely avoids the motor neuron pools in ventral spinal cord and concentrates instead on intermediate zone interneurons and dorsal sensory neurons \cite{Kuypers1981,Yang2003}. Why then is there such a large dependency on this tract for human motor control? One possibility is that the rubrospinal tract---a descending pathway originating in the brainstem and terminating in the intermediate zone---is degenerated in humans compared to other primates and mammals \cite{Nathan1955,Nathan1982}, and is thought to play a role in compensating for the loss of the corticospinal tract in non-human species \cite{Lawrence1968a,Zaaimi2012}.

It thus seems likely that most mammals rely on ``indirect'' pathways to convey cortical motor commands to muscles. These differences in anatomy might explain the lack of conspicuous, lasting movement deficits following motor cortical lesion in non-primates, but leaves behind a significant question: what is the motor cortex actually controlling in all these other mammals?

\begin{figure}
\begin{center}
\includegraphics[width=\columnwidth]{chapters/figuresChTeleology/descendingTaxa}
\end{center}
\vspace{-5mm}
\caption{Forebrain motor control pathways across different vertebrate taxa. The molecular divergence times between human (primate), rodent and lamprey groups \protect\cite{Kumar1998} are noted above a schematic view of the major divisions in the vertebrate brain. Arrows indicate the descending monosynaptic projections identified in each group from motor regions of the forebrain pallium to lower motor centres. Note the specialized monosynaptic projection directly targeting spinal motor neurons in human. MLR, Mesencephalic Locomotor Region; M, Motor Neurons.}
\label{fig:descendingTaxa}
\end{figure}

\subsubsection*{What is the role of motor cortex in non-primate mammals?}

In the rat, a large portion of cortex is considered ``motor'' based on anatomical \cite{Donoghue1982}, stimulation \cite{Donoghue1982,Neafsey1986} and electrophysiological evidence \cite{Hyland1998}. However, the most consistently observed long-term motor control deficit following motor cortical lesion has been an impairment in supination of the wrist and individuation of digits during grasping, which in turn impairs reaching for food pellets through a narrow vertical slit \cite{Whishaw1991,Alaverdashvili2008a}. Despite the fact that activity in rodent motor cortex has been correlated with movements in every part of the body (not just distal limbs) \cite{Hill2011,Erlich2011}, it would appear we are led to conclude that this large high-level motor structure, with dense efferent projections to motor areas in the spinal cord \cite{Kuypers1981}, basal ganglia \cite{Turner2000,Wu2009}, thalamus \cite{Lee2008}, cerebellum \cite{Baker2001} and brainstem \cite{Jarratt1999}, as well as to most primary sensory areas \cite{Petreanu2012,Schneider2014}, evolved simply to facilitate more precise wrist rotations and grasping gestures. Maybe we are missing something. Might there be other problems in movement control that motor cortex is solving, but that we may be overlooking with our current assays?

