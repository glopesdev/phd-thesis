\section{Anatomy of the Motor Cortex}

Since the early observations of its laminar organization \cite{Baillarger1840}, the intricate anatomy of the cortex has always fascinated neuroscience researchers.

\subsection{Laminar Cytoarchitecture}

Anatomically, there is a clear histological signature of mammalian motor cortex: the loss of granular layer IV. In the canonical layered arrangement of cortical tissue there is a dense layer of cell bodies known as the granular layer that separates the superficial and deep layers. In primary motor cortex this layer is dramatically reduced, earning it the name of agranular cortex. This loss of granular layer seems to be tightly correlated with expansion of the deep layers, especially layer V.

Functionally, layer IV is thought to be the target of feed-forward projections from lower areas in hierarchical models of cortical organization. This view is based on extrapolation from anatomical studies in primary sensory areas where layer IV receives most of the projections from the thalamus, making it the first layer of the cortex to receive direct sensory input.

\subsection{Descending Projections}

Another anatomical landmark of mammalian motor cortex is the presence of a large number of myelinated fibers that descend monosynaptically to the spinal cord. These form the so-called corticospinal, or pyramidal, tract. There is cross over, or decussation, of the tract at the level of the medulla oblongata, which brings the nerve impulses generated by motor cortex mostly to contralateral spinal cord. There is also a smaller bundle of fibres that descend ipsilaterally.

Apart from these direct projections from motor cortex to the spinal cord, there are also many other privileged routes from motor cortex to movement controlling centers in the brainstem. These are also thought to modulate many behaviours via direct brainstem projections to the spinal cord, such as the rubrospinal, tectospinal, vestibulospinal and reticulospinal tracts.

\subsection{Cortico-cortical Projections}

Despite the predominance of descending projections from motor cortex, in recent years a number of studies have revealed the full extent of direct motor cortical outputs, many of them targeting primary sensory areas.