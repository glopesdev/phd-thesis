				
\section{Chapter Summary}

Motor cortex has 150 years of conflicting history. It was originally defined as the part of cortex where movements can be evoked by low-current stimulation. Stimulated points across the cortical surface were found to be organized in rough somatotopy. A monosynaptic projection system, the pyramidal tract, was found to directly link cortex to neurons in contralateral spinal cord. Lesions of the motor cortex in humans were found to permanently disrupt the execution of fine movements of the digits and distal forelimb. Large-scale cytoarchitectural studies revealed an expansion of corticofugal layer V and a marked decrease in granular layer IV across this excitable zone. Recordings of neural activity in motor cortex correlate with various movement parameters. These lines of evidence were in support of the idea that this part of the brain directly controls movement.

However, lesions of the motor cortex in non-human animals preserve most of the animal's behaviour repertoire. Sectioning of the pyramidal tract in primates is sufficient to reinstate the primary effects of motor cortical lesions, but there are also vast projections from motor cortex to other cortical and sub-cortical areas, including multiple disynaptic parallel descending pathways to spinal centers via brainstem. In most mammals, the pyramidal tract does not target motor neurons in ventral spinal cord, as it does in primates, but rather spinal interneurons.

It is clear that this part of the brain is somehow involved in movement, but a large number of questions remain strangely unanswered. If motor cortex is a controller, what kind of movements does it control? How does it interact with other existing brain structures to generate behaviour? Why do motor cortical lesions produce such an apparently incomplete effect on movement? This chapter is an attempt to piece together all the fragmentary and contradictory evidence on motor cortical structure and physiology in order to derive a unified functional picture of cortical motor control.

\pagebreak


