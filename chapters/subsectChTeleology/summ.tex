				
\section{Chapter Summary}

Motor cortex has 150 years of conflicting history. It was originally defined as the part of cortex where movements can be evoked by low-current stimulation. Stimulated points across the cortical surface were found to be organized in rough somatotopy. Cytoarchitecture revealed an expansion of corticofugal layer V and a marked decrease in granular layer IV across this excitable zone. A direct monosynaptic projection system, the pyramidal tract, was found to link cortex to motor neurons in contralateral spinal cord. Lesions of the motor cortex in humans permanently disrupt the execution of fine movements of the digits and distal forelimb. Recordings of neural activity in motor cortex correlate with movement parameters. These lines of evidence support the idea that this part of the brain directly controls movement.

However, lesions of the motor cortex in non-human animals preserve most of the animal's behaviour repertoire. There are multiple parallel descending pathways to spinal centers via brainstem and other subcortical structures. In most mammals, the pyramidal tract does not target motor neurons in ventral spinal cord, but rather dorsal interneurons. Sectioning of the pyramidal tract in primates is sufficient to recapitulate the primary effects of motor cortical lesions, but there are vast projections from motor cortex to other cortical and sub-cortical areas.

This chapter is an attempt to piece together all the fragmentary and contradictory evidence on motor cortical structure and physiology.

\pagebreak


