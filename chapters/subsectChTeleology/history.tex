\section{A Dilemma for Cortical Motor Control}

The involvement of the brain and spinal cord in motor control has been recognized since the earliest known clinical records on head and spinal injury, dating back to ancient Egypt \cite{Louis1994,VanMiddendorp2010}. However, the role of the nervous system in generating behaviour was not fully appreciated until Galvani first reported his famous experiments on \textit{animal electricity} \cite{Galvani1791}. By isolating the sciatic nerve and gastrocnemius muscle in the frog, Galvani clearly demonstrated in a series of stimulation experiments that an electrical process, contained entirely within the biology of the frog's leg, was responsible for the spontaneous generation of muscle contractions. This would lead over the following century to the discovery and physiological characterization of the nerve impulse, the action potential, that travels across the nerve to initiate muscle movement \cite{DuBois-Reymond1843,Bernstein1868,Schuetze1983}. The success of these seminal experiments notwithstanding, they immediately raised a fundamental question regarding nerve conduction: if spontaneous muscle contraction is generated by nerve impulses transmitted throughout the nervous system, how is this transmission coordinated in order to generate the complex patterns of muscle activity observed in natural behaviour?

\subsection{Discovery of the Motor Cortex}

In search of answers to this question, many researchers looked at the brain, the seat of anatomical convergence of the nervous system, for such an integrative role. Following Galvani's footsteps, several attempts were made to stimulate the cerebral cortex electrically, but with little success \cite{Gross2007}. It wasn't until the 1870s that the first indications of a direct involvement of the cortex in the production of movement came to light, around the time when Hughlings Jackson underwent his studies on epileptic convulsions \cite{Jackson1870}. He observed that in some patients the fits would start by a deliberate spasm on one side of the body, and that different body parts would become systematically affected one after the other. He connected the orderly march of these spasms to the existence of localized lesions in the \emph{post-mortem} brain of his patients and hypothesized that the origin of these fits was uncontrolled excitation caused by local changes in cortical \emph{grey matter} \cite{Jackson1870}. In that same year, Fritsch and Hitzig published their famous study demonstrating that it is possible to elicit movements by direct stimulation of the cortex in dogs \cite{Fritsch1870}. Furthermore, stimulation of different parts of the cortex produced movement in different parts of the body \cite{Fritsch1870}. It appeared that the causal mechanism for epileptic convulsions predicted by Hughlings Jackson had been found, and with it a possible explanation for how the normal brain might control movement. The cerebral cortex was already considered at the time to be the seat of reasoning and sensation, so if activity over this so-called \emph{motor cortex} was able to exert direct control over the whole musculature of the body, then it might represent in the normal brain the area that connects volition to muscles \cite{Fritsch1870}.

\subsection{The Goltz-Ferrier Debates}

David Ferrier, a Scottish neurologist deeply impressed by the ideas of Hughlings Jackson and by the positive results of Fritsch and Hitzig's experiments, proceeded to reproduce and expand on their observations with comprehensive stimulation studies showing how activity in the motor cortex was sufficient to produce a large variety of movements across a wide range of mammalian species \cite{Ferrier1873}. Meanwhile, other researchers across Europe such as Goltz and Christiani were facing a dilemma: in many of the so-called ``lower mammals'' massive lesions of the cerebral cortex failed to demonstrate any visible long-term impairments in the motor behaviour of animals \cite{James1885,Goltz1888}.  These two lines of inquiry first clashed at the seventh International Medical Congress held in London in August 1881, where Goltz of Strassburg and Ferrier of London presented their results in a series of debates on the localization of function in the cerebral cortex \cite{Phillips1984,Tyler2000}.

Goltz assumed a clear anti-localizationist position. He advanced that it was impossible to produce a complete paresis of any muscle, or complete dysfunction of any perception, by destruction of any part of the cerebral cortex, and that he found mostly deficits of general intelligence in his dogs \cite{Tyler2000}. Following Goltz's presentation, Ferrier emphasized the danger of generalizing from the dog to animals of other orders (e.g. man and monkey). He then proceeded to exhibit his own lesion results by means of antiseptic surgery in the monkey, describing how a circumscribed unilateral lesion of the motor cortex produced complete contralateral paralysis of the leg. He also produced a striking series of microscopic sections of Wallerian degeneration \cite{Waller1850} of the ``motor path'' from the cortex to the contralateral spinal cord, the crossed descending projections forming the pyramidal corticospinal tract \cite{Tyler2000}.

The debates concluded with the public demonstration of live specimens: a dog with large lesions to the parietal and posterior lobes from Goltz; and from Ferrier, a hemiplegic monkey with a unilateral lesion to the motor cortex of the contralateral side. As predicted, Goltz's dog showed a clear ability to locomote and avoid obstacles and to make use of its other basic senses, while displaying peculiar deficits of intelligence such as failing to respond with fear to the cracking of a whip or ignoring tobacco smoke blown to its face. On the other hand, Ferrier's monkey showed up severely hemiplegic, in a condition similar to human stroke patients. After the demonstrations, the animals were killed and their brains removed. Preliminary observations revealed that the lesions in Goltz's dog were less extensive than expected, particularly on the left hemisphere. Ferrier's lesions on the other hand were precisely circumscribed to the contralateral motor cortex. These demonstrations secured the triumph of Ferrier, who went on to firmly establish the localizationist approach to neurology and the idea of a somatotopic arrangement over the motor cortex.

The Goltz-Ferrier debates had far-reaching implications throughout the entire research community of the time, and the basic dilemma that was presented has sparked controversy and confusion for over a hundred years since \cite{Phillips1984,Lashley1924,DeBarenne1933,Tyler2000,Gross2007}. In the meantime, views of motor cortex have evolved to suggest it plays a role in ``understanding'' the movements of others \cite{Rizzolatti2004}, imagining one's own movements \cite{Porro1996}, or in learning new movements \cite{Kawai2015}, but where are we today regarding its suggested primary role in directly controlling movement?

\subsubsection*{Stimulating motor cortex causes movement; motor cortex is active during movement}

Motor cortex is still broadly defined as the region of the cerebral hemispheres from which movements can be evoked by low-current stimulation, following Fritsch and Hitzig's original experiments in 1870 \cite{Fritsch1870}. Stimulating different parts of the motor cortex elicits movement in different parts of the body, and systematic stimulation surveys have revealed a topographical representation of the entire skeletal musculature across the cortical surface \cite{Leyton1917, Penfield1937, Neafsey1986}. Electrophysiological recordings in motor cortex have routinely found correlations between neural activity and many different movement parameters, such as muscle force \cite{Evarts1968}, movement direction \cite{Georgopoulos1986}, speed \cite{Schwartz1993}, or even anisotropic limb mechanics \cite{Scott2001} at the level of both single neurons \cite{Evarts1968,Churchland2007} and populations \cite{Georgopoulos1986,Churchland2012}. Determining what exactly this activity in motor cortex controls \cite{Todorov2000} has been further complicated by studies using long stimulation durations in which continuous stimulation at a single location in motor cortex evokes complex, multi-muscle movements \cite{Graziano2002,Aflalo2006}. However, as a whole, these observations all support the long standing view that activity in motor cortex is involved in the direct control of movement.

\subsubsection*{Motor cortex lesions produce different deficits in different species}

What types of movement require motor cortex? In humans, a motor cortical lesion is devastating, resulting in the loss of muscle control or even paralysis; movement is permanently and obviously impaired \cite{Laplane1977}. In non-human primates, similar gross movement deficits are observed after lesions, albeit transiently \cite{Leyton1917,Travis1955}. The longest lasting effect of a motor cortical lesion is the decreased motility of distal forelimbs, especially in the control of individual finger movements required for precision skills \cite{Leyton1917,Darling2011}. But equally impressive is the extent to which other movements fully recover, including the ability to sit, stand, walk, climb and even reach to grasp, as long as precise finger movements are not required \cite{Leyton1917,Darling2011,Zaaimi2012}. In non-primate mammals, the absence of lasting deficits following motor cortical lesion is even more striking. Careful studies of skilled reaching in rats have revealed an impairment in paw grasping behaviours \cite{Whishaw1991,Alaverdashvili2008a}, comparable to the long lasting deficits seen in primates, but this is a limited impairment when compared to the range of movements that \emph{are} preserved \cite{Whishaw1991,Kawai2015}. In fact, even after complete decortication, rats, cats and dogs retain a shocking amount of their movement repertoire \cite{Goltz1888,Bjursten1976,Terry1989}. If we are to accept the simple hypothesis that motor cortex is the structure responsible for ``voluntary movement production'', then why is there such a blatant difference in the severity of deficits caused by motor cortical lesions in humans versus other mammals? With over a century of stimulation and electrophysiology studies clearly suggesting that motor cortex is involved in many types of movement, in all mammalian species, how can these divergent results be reconciled?

\section{Hierarchy of Feedback Controllers}

\subsection{The Reflex Arc and the Principle of the Final Common Path}

After his return to Cambridge, Sherrington mounted his attack on the problems of nerve conduction from the periphery. While many researchers continued to look for the integration of complex movements in higher brain structures like the motor cortex, Sherrington turned instead to systematically characterizing anatomically and physiologically the distribution of efferent \cite{Sherrington1892} and afferent \cite{Sherrington1893a} nerve roots in the spinal cord of multiple species. His goal was to shed light on the so-called \emph{reflex arc}, the nerve pathways involved in muscular reactions like the knee-jerk whereby simple sensory stimuli elicit an immediate, automatic response from the animal, even in the absence of higher brain input \cite{Sherrington1893b}.

Sherrington and his contemporaries studied in detail a number of long and short spinal reflexes\footnote{A reflex action in which a stimulus applied to one region elicits a response in another region is termed a \emph{long spinal} reflex, whereas a reflex reaction where the muscular response happens in the same region as the stimulus is termed a \emph{short spinal} reflex.} in a variety of model organisms under different levels of anesthesia, pharmacological manipulations and spinal transsection \cite{Sherrington1903}. This systematic approach made abundantly clear a number of facts about how the nervous system organizes motor control.

The first one, and perhaps the most striking, is that complex motor responses can be integrated and coordinated even in the complete absence of the brain \cite{Sherrington1906}. From early on it became clear that the most basic unit of integration, the reflex, was far from being a rigid and fixed entity, but was rather adaptive and dynamic. All reflex circuits revealed a much broader range of response characteristics than nerve fibers, which were well known since the time of Galvani to exhibit complete stereotypy in their response to a stimulus under various conditions\footnote{Some unique response characteristics of reflex arc conduction include irreversibility of the direction of conduction; fatigability and refractory period; greater variability of threshold; temporal facilitation with successive stimuli; a weaker correspondence of end-effect with intensity and frequency of the stimulus; and a greater susceptibility to metabolic and pharmacological manipulations \cite[p.14]{Sherrington1906}}. Furthermore, the displayed muscular responses could be interpreted as having ethological meaning: reflex stepping and standing \cite{Sherrington1910, Sherrington1915}, scratching \cite{Sherrington1903}, or shaking \cite{Goltz1896, Sherrington1903} are just some examples of the adaptive behaviour repertoire found in spinal animals. These reflexes were also shown to be deployed and modulated appropriately to specific stimuli. The execution of the scratch reflex, for example, depends on where in the skin the stimulus is delivered, and in reflex stepping the animal can maintain a rhythmic march through all phases of locomotion over unobstructed surfaces. Integration of these reflexes with input from the telereceptors is obviously entirely absent, but these observations even today should raise awareness to the fact that spinal cord circuits are sufficient to produce and sustain entire behaviour sequences under the right conditions.

Deafferentation experiments showed that aspects of these rhythmic network motifs persist even in the absence of sensory input \cite{GrahamBrown1911}. Many of these reflex circuits were later termed \emph{central pattern generators}, or CPGs \cite{Grillner1975, Grillner1981}, and found to be present across both vertebrate and invertebrate species \cite{Orlovsky1999,Selverston2010}. However, it is important to note that apart from CPG-like network patterns in the spinal cord, there is also a general capacity of spinal circuits as a whole to initiate and switch between the different responses, as well as modulate network activity with incoming sensory input in order to produce adaptive behaviour responses \cite{Forssberg1975}.

Another aspect of the neural control of behaviour emphasized by Sherrington was the existence of a \emph{final common path} from the nervous system to muscles \cite{Sherrington1904}. Motor neurons in the spinal cord send their axons through the ventral roots of spinal segments to synapse directly on muscle fibres. From his experiments, Sherrington emphasized that these motor neurons are actually shared by reflex arcs initiated by distinct sensory receptors. In Sherrington's time the existence of such common paths was a problem for reflex control, especially if the function of the nervous system was conceived in terms of nerve conduction. The idea of the final common path made it necessary to speak openly of coordination between different circuit elements and to describe mechanisms that would allow the same neurons to take on context-dependent roles in generating motor behaviour.

\subsection{The Coordinative Role of Inhibition}

In search of a mechanism for reflex arc coordination, Sherrington hit upon the fundamental role of inhibition in the organization of neural function. Inhibition had always been a complicated topic for physiologists, but following up on the demonstration of cardiac muscle inhibition by the vagus nerve \cite{Weber1846}, and Sechenov's grand proposal of a central origin of reflex inhibition \cite{Sechenov1863}, Sherrington was able to hypothesize and experimentally validate simple mechanisms for spinal reflex coordination, such as the law of reciprocal inhibition of antagonistic muscles \cite{Sherrington1893b}.

However, Sherrington's experiments went much beyond the simple antagonism of opponent muscles and also probed the central role of inhibition in the coordination of conflicting reflex responses initiated by distinct sensory receptors. These antagonistic reflexes were found to be able to coordinate their influence despite sharing a final common path to muscles. That such coordination existed was made clear by stimulation experiments where two or more reflexes were elicited at the same time. In these experiments, Sherrington showed that the muscle response to combined stimulation was not a simple summation or linear combination of the responses obtained by stimulation delivered in isolation. Rather, as he describes:

\blockquote[{\protect\cite[p.461]{Sherrington1904}}]{When two stimuli are applied simultaneously which would evoke reflex actions that employ the same final common path in different ways, in my experience one reflex appears without the other. The result is this reflex or that reflex, but not the two together.}

Sherrington's investigations of inhibition in the coordination of reflexes again produced a number of invaluable insights for thinking about cortical motor control. The first one is the realization of inhibition as a powerful active force in the nervous system \cite{Sherrington1965}. Rather than simply being the \emph{absence} of excitation or shutting down of function, inhibition was found to play a fundamental role in the coordination and maintenance of central state. A major difficulty that physiologists found with inhibition is that it can only be actively measured in comparison to a baseline of excitation. If this baseline is not set up accurately, inhibition and its effects can be easy to miss.

Another important realization is how transient changes in inhibitory and excitatory state can explain phenomena of ``shock'' and ``release'', a generalized depression of activity or over-action in a given area following injury or destruction to distant but related parts of the nervous system. How these transient changes ultimately manifest themselves is dependent on the particular anatomical situation of the influencing and influenced centres. The presence of such transient changes can be enough to establish that two areas are connected, in a way similar to a stimulation experiment, but the generalized transient change in central excitatory and inhibitory state can obscure the function of either area.

% Give a detailed example of a mechanism of reciprocal innervation explaining coordination
% Talk about the difficulty of measuring inhibition: you need to compare it to a baseline of excitation
% Talk about inhibition as an "active" force in the nervous system. A "pulse" of inhibition is as much an input as a pulse of "excitation".
% Talk about the synapse as the location for the mechanism of inhibition (include stuff on the discovery of the synapse)

% The second one, the final common path and the need of neural coordination at the output
% The third one, inhibition is as active a force in the nervous system as excitation
% The fourth one, response latencies and implications for motor control (and synapses)

% One of the major points of contention were the differences between nerve trunk conduction and reflex arc conduction. The speed of propagation of the action potential across a nerve fiber had been known since the experiments of Helmholtz \cite{Helmholtz1850,Schmidgen2002} so it was possible to estimate the expected latency for a short spinal reflex, assuming reasonable estimates of nerve length and mechanical latency. For example, in the flexion-reflex of the dog's hind limb this latency was estimated to be around \SI{27}{\milli\second} \cite[p.19]{Sherrington1906}. However, the observed latencies in the spinal animal were, under normal conditions, found to be at least double of this number. Furthermore, reflex arc conduction had many other characteristics that were distinctive from the well studied nerve trunk conduction. Besides longer latencies, reflex arc conduction exhibited irreversibility of direction of conduction; fatigability and refractory period; greater variability of threshold; temporal facilitation with successive stimuli; a weaker correspondence of end-effect with intensity and frequency of the stimulus; and a greater susceptibility to metabolic and pharmacological manipulations \cite[p.14]{Sherrington1906}.

% These differences made Sherrington realize the necessity for a change in the physical medium of conduction at some point along the arc. The existence of such gaps clashed with the dominant idea at the time which viewed the nervous system as a continuous conductive nerve net. However, with the development of Golgi staining, Ramon y Cajal had recently been able to finally introduce convincingly the idea that the nervous system was actually composed of discrete cellular units, \emph{neurons} \cite{RamonYCajal1894}. Sherrington became a supporter of the newly established neuron doctrine and posited that the required theoretical junction between two neurons, which he termed the ``\emph{synapse}'' \cite{Foster1897}, could explain the unique physiology of reflex arc conduction.

Today we know from several detailed descriptions of invertebrate CPGs that different spinal circuits can be delicately super-imposed on the same neuronal elements by using a number of strategies to regulate and coordinate function \cite{Orlovsky1999,Selverston2010}. These strategies span all levels of neural organization from networks to molecules, ranging from reciprocal inhibition network motifs gated by specific neuromodulators and cell membrane receptors to overlapping distributions of voltage and ligand gated ion channels with different kinetics. The existence of such finely tuned spinal networks brings significant constraints when thinking about cortical motor control. Ultimately, any descending signals from the brain to muscles are also sharing this final common path and thus should be expected to require some form of coordination with spinal circuits if behavioural output is to remain integrated.

\subsubsection*{A role in modulating the movements generated by lower motor centres}

A different perspective on motor cortex emerged from studying the neural control of locomotion, suggesting that the corticospinal tract plays a role in the \emph{adjustment} of ongoing movements that are generated by lower motor systems. In this view, rather than motor cortex assuming direct control over muscle movement, it instead modulates the activity and sensory feedback in spinal circuits in order to adapt a lower movement controller to challenging conditions. This idea that the descending cortical pathways superimpose speed and precision on an existing baseline of behaviour was also suggested by lesion work in primates \cite{Lawrence1968a}, but has been investigated most thoroughly in the context of cat locomotion.

It has been known for more than a century that completely decerebrate cats are capable of sustaining the locomotor rhythms necessary for walking on a flat treadmill utilizing only spinal circuits \cite{GrahamBrown1911}. Brainstem and midbrain circuits are sufficient to initiate the activity of these spinal central pattern generators \cite{Grillner1973}, so what exactly is the contribution of motor cortex to the control of locomotion? Single-unit recordings of pyramidal tract neurons (PTNs) from cats walking on a treadmill have shown that a large proportion of these neurons are locked to the step cycle \cite{Armstrong1984a}. However, we know from the decerebrate studies that this activity is not necessary for the basic locomotor pattern. What then is its role?

Lesions of the lateral descending pathways (containing corticospinal and rubrospinal projections) produce a long term impairment in the ability of cats to step over obstacles \cite{Drew2002}. Recordings of PTN neurons during locomotion show increased activity during these visually guided modifications to the basic step cycle \cite{Drew1996}. These observations suggest that motor cortex neurons are necessary for precise stepping and adjustment of ongoing locomotion to changing conditions. However, long-term effects seem to require complete lesion of \emph{both} the corticospinal and rubrospinal tracts \cite{Drew2002}. Even in these animals, the voluntary act of stepping over an obstacle does not disappear entirely, and moreover, they can adapt to changes in the height of the obstacles \cite{Drew2002}. Specifically, even though these animals never regain the ability to gracefully clear an obstacle, when faced with a higher obstacle, they are able to adjust their stepping height in such a way that would have allowed them to comfortably clear the lower obstacle \cite{Drew2002}. Furthermore, deficits caused by lesions restricted to the pyramidal tract seem to disappear over time \cite{Liddell1944}, and are most clearly visible only the first time an animal encounters a new obstacle \cite{Liddell1944}.

The view that motor cortex in non-primate mammals is principally responsible for adjusting ongoing movement patterns generated by lower brain structures is appealing. What is this modulation good for? What does it allow an animal to achieve? How can we assay its necessity?

\section{A Teleology for Motor Cortex - Bernstein}

\subsection{The Search for a Teleology of Motor Cortex}

Despite his success in laying the foundations for the coordination of reflexes in the spinal cord, Sherrington continued to show interest in following up his original work on cortical localization and behaviour. In a series of seminal publications together with Gr\"unbaum \cite{Grunbaum1903,Leyton1917}, Sherrington presented a number of experiments attempting to clarify the role of excitable cortex.

In these experiments covering different primate species, Sherrington combined surface stimulation mapping with studies of behaviour after lesions to excitable areas in the same animal. Through these systematic studies he hoped to better understand two previously conflicting observations of motor cortical physiology: the instability and complexity of responses evoked by stimulating different cortical points \cite{GrahamBrown1912}, and the dramatic recovery of motor function which was often observed following circumscribed cortical lesions \cite{GrahamBrown1913}.

To do this, Sherrington and his co-workers would carefully map the forearm region of motor cortex in one hemisphere of individual animals, and then proceed to precisely excise the entire area, using the stimulation maps as a guide. In one of their reported cases, after the animal recovered from surgery, his use of the contralateral forearm was visibly impaired, showing pronounced weakness and inability to grasp with the right hand. However, one month following the surgery, the animal had recovered most of the functions in the affected arm, apart from deficits in individuated finger movements and slight weakening of thumb and index. In an attempt to trace out the parts of the brain responsible for the recovery, they re-exposed the lesioned hemisphere and again mapped out movement responses to stimulation. They observed that they were unable to evoke responses in the affected arm by stimulating the site of the lesion, as well as any adjacent cortex in front or behind \cite{Leyton1917}.

\subsubsection*{Towards a new teleology; new experiments required}

It should now be clear that the involvement of motor cortex in the direct control of all ``voluntary movement'' is human-specific. There is a role for motor cortex across mammals in the control of precise movements of the extremities, especially those requiring individual movements of the fingers, but these effects are subtle in non-primate mammals. Furthermore, what would be a devastating impairment for humans may not be so severe for mammals that do not depend on precision finger movements for survival. Therefore, generalizing this specific role of motor cortex from humans to all other mammals would be misleading. We could be missing another, more primordial role for this structure that predominates in other mammals, and by doing so, we may also be missing an important role in humans.

The proposal that motor cortex induces modifications of ongoing movement synergies, prompted by the electrophysiological studies of cat locomotion, definitely points to a role consistent with the results of various lesion studies. However, in assays used, the ability to modify ongoing movement generally recovers after a motor cortical lesion. What are the environmental siutations in which motor cortical modulation is most useful?

Cortex has long been proposed to be the structure responsible for integrating a representation of the world and improving the predictive power of this representation with experience \cite{Barlow1985,Doya1999}. If motor cortex is the means by which these representations can gain influence over the body, however subtle and ``modulatory'', can we find situations (i.e. tasks) in which this cortical control is required?

The necessity of cortex for various behavioural tasks has been actively investigated in experimental psychology for over a century, including the foundational work of Karl Lashley and his students \cite{Lashley1921a,Lashley1950a}. In the rat, large cortical lesions were found to produce little to no impairment in movement control, and even deficits in learning and decision making abilities were difficult to demonstrate consistently over repeated trials. However, Lashley did notice some evidence that cortical control may be involved in postural adaptations to unexpected perturbations \cite{Lashley1921a}. These studies once again seem to recapitulate the two most consistent observations found across the entire motor cortical lesion literature in non-primate mammals since Hitzig \cite{Fritsch1870}, Goltz \cite{Goltz1888}, Sherrington \cite{Sherrington1885} and others \cite{Oakley1979,Terry1989}. One, direct voluntary control over movement is most definitely not abolished through lesion; and two, certain aspects of some movements are definitely impaired, but only under certain challenging situations. The latter are often reported only anecdotally. It was this collection of intriguing observations in animals with motor cortical lesions that prompted us to expand the scope of standard laboratory tasks to include a broader range of motor control challenges that brains encounter in their natural environments.

In the following, we report an experiment that was designed to provide controlled exposure of animals to more naturally challenging environments. The results of this experiment have led us to formulate a new teleology for cortical motor control that we will present in the discussion.



