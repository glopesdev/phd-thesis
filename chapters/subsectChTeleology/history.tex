\section{Brief History of Cortical Motor Control}

The involvement of the brain and spinal cord in motor control has been recognized since the earliest known clinical records on head and spinal injury \cite{Louis1994,VanMiddendorp2010}. However, the role of the nervous system in generating behaviour was not fully appreciated until Galvani first reported his famous experiments on \textit{animal electricity} \cite{Galvani1791}. By isolating the sciatic nerve and gastrocnemius muscle in the frog, Galvani clearly demonstrated in a series of stimulation experiments that an electrical process, contained entirely within the biology of the frog's leg, was responsible for the spontaneous generation of muscle contractions. This would lead over the following century to the discovery and physiological characterization of the nerve impulse, the action potential, that travelled across the nerve to initiate muscle movement \cite{DuBois-Reymond1843,Bernstein1868,Schuetze1983}. The success of these experiments immediately raised further questions of nerve conduction: if spontaneous muscle contraction is generated by nerve impulses transmitted throughout the nervous system, how is this transmission coordinated to generate the patterns of natural movement we see in the behaving animal?

\subsection{Discovery of the Motor Cortex}

In search of answers to this question, many researchers looked at the brain, the seat of anatomical convergence of the nervous system, for such an integrative role. Following Galvani's footsteps, several attempts were made to stimulate the cerebral cortex electrically, but with little success \cite{Gross2007}. It wasn't until the 1870s that the first indications of a direct involvement of the cortex in the production of movement came to light, around the time when Hughlings Jackson underwent his studies on epileptic convulsions \cite{Jackson1870}. He realized that in some patients the fits would start by a deliberate spasm on one side of the body, and that different body parts would become systematically affected one after the other. He connected the orderly march of these spasms to the existence of localized lesions in the brain of his patients and hypothesized that the origin of these fits was uncontrolled excitation caused by local changes in cortical \emph{grey matter} \cite{Jackson1870}. In that same year, Fritsch and Hitzig published their famous study demonstrating that it is possible to elicit movements by direct stimulation of the cortex in dogs \cite{Fritsch1870}. Furthermore, stimulation of different parts of the cortex produced movement in different parts of the body \cite{Fritsch1870}. It appeared that the causal mechanism for epileptic convulsions predicted by Hughlings Jackson had been found, and with it a possible explanation for how the normal brain might control movement. By this time the cerebral cortex was already considered to be the seat of reasoning and sensations, so if activity over this so-called \emph{motor cortex} could exert direct control over the whole musculature of the body, then it might represent in the normal brain the area that connects volition to muscles \cite{Fritsch1870}.

\subsection{The Goltz-Ferrier Debates}

David Ferrier, a Scottish neurologist deeply impressed by the ideas of Hughlings Jackson and by the positive results of Fritsch and Hitzig's experiments, proceeded to reproduce and expand on their observations across a wide range of mammalian species \cite{Ferrier1873}. However, while Ferrier's comprehensive stimulation studies were showing how activity in the motor cortex was sufficient to produce a large variety of movements, other researchers like Goltz were reporting how massive cortical lesions failed to demonstrate any visible long-term impairments in the motor behaviour of animals \cite{Goltz1888}. These two lines of inquiry first clashed at the seventh International Medical Congress held in London in August 1881, where both Goltz and Ferrier presented their results in a series of debates on the localization of function in the cerebral cortex \cite{Tyler2000}.

Goltz assumed a clear anti-localizationist position. He advanced that it was impossible to produce a complete paresis of any muscle, or complete dysfunction of any perception, by destruction of any part of the cerebral cortex, and that he found mostly deficits of general intelligence in his dogs \cite{Tyler2000}. Following Goltz's presentation, Ferrier emphasized the danger of generalizing from the dog to animals of other orders (e.g. man and monkey). He then proceeded to exhibit his own lesion results by means of antiseptic surgery in the monkey, describing how a circumscribed unilateral lesion of the motor cortex produced complete contralateral paralysis of the leg. He also produced a striking series of microscopic sections of Wallerian degeneration \cite{Waller1850} of the ``motor path'' from the cortex to the contralateral spinal cord, the crossed descending projections forming the pyramidal corticospinal tract \cite{Tyler2000}.

The debates concluded with the public demonstration of live specimens: a dog with large lesions to the parietal and posterior lobes from Goltz; and a hemiplegic monkey with a unilateral lesion to the motor cortex of the contralateral side from Ferrier. As predicted, Goltz's dog showed a clear ability to locomote and avoid obstacles and to make use of its other basic senses, while displaying peculiar deficits of intelligence such as failing to respond with fear to the cracking of a whip or ignoring tobacco smoke blown to its face. On the other hand, Ferrier's monkey showed up severely hemiplegic, in a condition similar to human stroke patients. After the demonstrations, the animals were killed and their brains removed. Preliminary observations revealed that the lesions in Goltz's dog were less extensive than expected, particularly on the left hemisphere. Ferrier's lesions on the other hand were precisely circumscribed to the contralateral motor cortex. These demonstrations secured the triumph of Ferrier, who went on to firmly establish the localizationist approach to neurology and the idea of a somatotopic arrangement over the motor cortex.

Nevertheless, the Goltz-Ferrier debates had far-reaching implications throughout the entire research community of the time, and the complexities of motor cortical anatomy and physiology would not be forgotten for too long. Charles Sherrington himself started out his work by tracing spinal cord degeneration over large periods of time (up to 11 months) following cortical lesions in Goltz's dogs \cite{Langley1884,Sherrington1885}. From these examinations, he observed for the first time in the dog the presence of a degenerated ``re-crossed'' pyramidal tract that travels down the cord ipsilateral to the side of the lesion \cite{Sherrington1885}. These fibers would later come to be called the ipsilateral, ventral corticospinal tract, and have since been found and described in most mammalian species \cite{Kuypers1981,Brosamle2000,Lacroix2004}. He also had the chance during this time to observe first hand the negative effects of cortical lesions reported by Goltz in a variety of specimens. In his own words:

\blockquote[{\protect\cite[p.189]{Sherrington1885}}]{That the pyramidal tracts are in the dog requisite for volitional~impulses to reach limbs and body seems negatived by the fact that the animal can run, leap, turn to either side, use neck and jaws, \&c. with ease and success after nearly, if not wholly, complete degeneration of these tracts on both sides. Further, after complete degeneration of one pyramid, there is in the dog no obvious difference between the movements of the right and left sides.}

Interestingly, he does note that \enquote{defect of motion is observable only as a clumsiness in execution of fine movements} \cite{Sherrington1885}, already hinting at ideas that are today still part of broadly accepted theories for the role of the pyramidal tract in motor control.

\subsection{The Reflex Arc and the Principle of the Final Common Path}

After his return to Cambridge, Sherrington mounted his attack on the problems of nerve conduction from the periphery. While many researchers continued to look for the integration of complex movements in higher brain structures like the motor cortex, Sherrington turned instead to systematically characterizing anatomically and physiologically the distribution of efferent \cite{Sherrington1892} and afferent \cite{Sherrington1893a} nerve roots in the spinal cord of multiple species. His goal was to shed light on the so-called \emph{reflex arc}, the nerve pathways involved in muscular reactions like the knee-jerk whereby simple sensory stimuli elicit an immediate, automatic response from the animal, even in the absence of higher brain input \cite{Sherrington1893b}.

Sherrington and his contemporaries studied in detail a number of long and short spinal reflexes\footnote{A reflex action in which a stimulus applied to one region elicits a response in another region is termed a \emph{long spinal} reflex, whereas a reflex reaction where the muscular response happens in the same region as the stimulus is termed a \emph{short spinal} reflex.} in a variety of model organisms under different levels of anesthesia, pharmacological manipulations and spinal transsection \cite{Sherrington1903}. This systematic approach made abundantly clear a number of facts about how the nervous system organizes motor control.

The first one, and perhaps the most striking, is that complex motor responses can be integrated and coordinated even in the complete absence of the brain \cite{Sherrington1906}. From early on it became clear that the unit of integration, the reflex, was far from being a rigid and fixed entity, but was rather adaptive and dynamic. All reflex circuits revealed a much broader range of response characteristics than nerve fibers, which were well known since the time of Galvani to exhibit complete stereotypy in their response to a stimulus under various conditions\footnote{Some unique response characteristics of reflex arc conduction include irreversibility of the direction of conduction; fatigability and refractory period; greater variability of threshold; temporal facilitation with successive stimuli; a weaker correspondence of end-effect with intensity and frequency of the stimulus; and a greater susceptibility to metabolic and pharmacological manipulations \cite[p.14]{Sherrington1906}}. Furthermore, the displayed muscular responses could be interpreted as having ethological meaning: reflex stepping and standing \cite{Sherrington1910, Sherrington1915}, scratching \cite{Sherrington1903}, or shaking \cite{Goltz1896, Sherrington1903} are just some examples of the adaptive behaviour repertoire found in spinal mammals. These reflexes were also shown to be deployed and modulated appropriately to specific stimuli. The execution of the scratch reflex, for example, depends on where in the skin the stimulus is delivered, and in reflex stepping the animal can maintain a rhythmic march through all phases of locomotion over unobstructed surfaces. Integration of these reflexes with input from the telereceptors is obviously entirely absent, but these observations even today should raise awareness to the fact that spinal cord circuits are sufficient to produce and sustain entire behaviour sequences under the right conditions.

Following deafferentation experiments showing that aspects of these rhythmic network motifs persist even in the absence of sensory input \cite{GrahamBrown1911}, many of these reflex circuits were termed \emph{central pattern generators}, or CPGs \cite{Grillner1975, Grillner1981}, and were found to be present across both vertebrate and invertebrate species \cite{Orlovsky1999,Selverston2010}. However, it is important to keep in mind that even though we find CPG-like network patterns in the spinal cord, there is also a general capacity of spinal circuits as a whole to initiate and switch between these different patterns, as well as modulate network activity with incoming sensory input in order to produce adaptive behaviour responses \cite{Forssberg1975}.

Another aspect of the neural control of behaviour emphasized by Sherrington was the existence of a \emph{final common path} from the nervous system to muscles \cite{Sherrington1904}. Motor neurons in the spinal cord send their axons through the ventral roots of spinal segments to synapse directly on muscle fibres. From his experiments, Sherrington emphasized that these motor neurons are actually shared by reflex arcs initiated by distinct sensory receptors. In Sherrington's time the existence of such common paths was a problem for reflex control, especially if the function of the nervous system was conceived in terms of nerve conduction. The idea of the final common path made it necessary to speak openly of coordination between different circuit elements and to describe mechanisms that would allow the same neurons to take on context-dependent roles in generating motor behaviour.

\subsection{The Coordinative Role of Inhibition}

In search of a mechanism for reflex arc coordination, Sherrington hit upon the fundamental role of inhibition in the organization of neural function. Inhibition had always been a complicated topic for physiologists, but following up on the demonstration of cardiac muscle inhibition by the vagus nerve \cite{Weber1846}, and Sechenov's grand proposal of a central origin of reflex inhibition \cite{Sechenov1863}, Sherrington was able to hypothesize and experimentally validate simple mechanisms for spinal reflex coordination, such as the law of reciprocal innervation of antagonistic muscles \cite{Sherrington1893b}.

These principles of inhibition provided a mechanistic explanation of how reflex responses initiated by distinct sensory receptors were able to coordinate their influence despite sharing a final common path to muscles. That such coordination existed was made clear by stimulation experiments where two or more reflexes were elicited at the same time. In these experiments, Sherrington showed that the muscle response to combined stimulation was not a simple summation or linear combination of the responses obtained by stimulation delivered in isolation. Rather, as he describes:

\blockquote[{\protect\cite[p.461]{Sherrington1904}}]{When two stimuli are applied simultaneously which would evoke reflex actions that employ the same final common path in different ways, in my experience one reflex appears without the other. The result is this reflex or that reflex, but not the two together.}

% Give a detailed example of a mechanism of reciprocal innervation explaining coordination
% Talk about the difficulty of measuring inhibition: you need to compare it to a baseline of excitation
% Talk about inhibition as an "active" force in the nervous system. A "pulse" of inhibition is as much an input as a pulse of "excitation".
% Talk about the synapse as the location for the mechanism of inhibition (include stuff on the discovery of the synapse)

Sherrington went on to characterize reciprocal inhibition and the reflex arc as the basic unit of integration in the spinal cord \cite{Sherrington1906}.

Yet for Sherrington himself, his greatest contributions were the discovery of the concept of reciprocal inhibition

This posed a significant problem for spinal motor control, in the cases where multiple such reflex responses had to be initiated simultaneously. Sherrington went on to demonstrate that when two or more reflexes were elicited at the same time, the observed response was not a simple summation or linear combination of both responses. synapse onto muscle fibers and  Specifically, he showed how reflex responses initiated by distinct sensory receptors would often impinge on the same motor neurons. These shared neurons represented thus a \emph{final common path} \cite{Sherrington1904} throughout the nervous system.

The second one, the final common path and the need of neural coordination at the output

The third one, inhibition is as active a force in the nervous system as excitation

The fourth one, response latencies and implications for motor control (and synapses)

 revealed a number of difficulties with the conceptual arrangement of the reflex arc in terms of the simple nerve conduction ideas of the time, and had dramatic implications for cortical motor control theories.

One of the major points of contention were the differences between nerve trunk conduction and reflex arc conduction. The speed of propagation of the action potential across a nerve fiber had been known since the experiments of Helmholtz \cite{Helmholtz1850,Schmidgen2002} so it was possible to estimate the expected latency for a short spinal reflex, assuming reasonable estimates of nerve length and mechanical latency. For example, in the flexion-reflex of the dog's hind limb this latency was estimated to be around \SI{27}{\milli\second} \cite[p.19]{Sherrington1906}. However, the observed latencies in the spinal animal were, under normal conditions, found to be at least double of this number. Furthermore, reflex arc conduction had many other characteristics that were distinctive from the well studied nerve trunk conduction. Besides longer latencies, reflex arc conduction exhibited irreversibility of direction of conduction; fatigability and refractory period; greater variability of threshold; temporal facilitation with successive stimuli; a weaker correspondence of end-effect with intensity and frequency of the stimulus; and a greater susceptibility to metabolic and pharmacological manipulations \cite[p.14]{Sherrington1906}.

These differences made Sherrington realize the necessity for a change in the physical medium of conduction at some point along the arc. The existence of such gaps clashed with the dominant idea at the time which viewed the nervous system as a continuous conductive nerve net. However, with the development of Golgi staining, Ramon y Cajal had recently been able to finally introduce convincingly the idea that the nervous system was actually composed of discrete cellular units, \emph{neurons} \cite{RamonYCajal1894}. Sherrington became a supporter of the newly established neuron doctrine and posited that the required theoretical junction between two neurons, which he termed the ``\emph{synapse}'' \cite{Foster1897}, could explain the unique physiology of reflex arc conduction.

Today we know from several detailed descriptions of invertebrate CPGs that different spinal circuits can be delicately super-imposed on the same neuronal elements by using a number of strategies to regulate and coordinate function \cite{Orlovsky1999,Selverston2010}. These strategies span all levels of neural organization from networks to molecules, ranging from reciprocal inhibition network motifs gated by specific neuromodulators and cell membrane receptors to overlapping distributions of voltage and ligand gated ion channels with different kinetics. The existence of such finely tuned spinal networks brings significant constraints when thinking about cortical motor control. Ultimately, any descending signals from the brain to muscles are also sharing this final common path and thus should be expected to require some form of coordination with spinal circuits if behavioural output is to remain integrated.

\subsection{The Search for a Teleology of Motor Cortex}

Despite his success in laying the foundations for the coordination of reflexes in the spinal cord, Sherrington continued to show interest in following up his original work on cortical localization and behaviour.