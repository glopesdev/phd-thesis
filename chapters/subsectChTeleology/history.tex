\section{A Dilemma for Cortical Motor Control}

The involvement of the brain and spinal cord in motor control has been recognized since the earliest known clinical records on head and spinal injury, dating back to ancient Egypt \cite{Louis1994,VanMiddendorp2010}. However, the role of the nervous system in generating behaviour was not fully appreciated until Galvani first reported his famous experiments on \textit{animal electricity} \cite{Galvani1791}. By isolating the sciatic nerve and gastrocnemius muscle in the frog, Galvani clearly demonstrated in a series of stimulation experiments that an electrical process, contained entirely within the biology of the frog's leg, was responsible for the spontaneous generation of muscle contractions. This would lead over the following century to the discovery and physiological characterization of the nerve impulse, the action potential, that travels across the nerve to initiate muscle movement \cite{DuBois-Reymond1843,Bernstein1868,Schuetze1983}. The success of these seminal experiments immediately raised a fundamental question regarding nerve conduction: if spontaneous muscle contraction is generated by nerve impulses transmitted throughout the nervous system, how is this transmission coordinated in order to generate the complex patterns of muscle activity observed in natural behaviour?

\subsection{Discovery of the Motor Cortex}

In search of answers to this question, many researchers looked at the brain, the seat of anatomical convergence of the nervous system, for such an integrative role. Following Galvani's footsteps, several attempts were made to stimulate the cerebral cortex electrically, but with little success \cite{Gross2007}. It wasn't until the 1870s that the first indications of a direct involvement of the cortex in the production of movement came to light, around the time when Hughlings Jackson underwent his studies on epileptic convulsions \cite{Jackson1870}. He observed that in some patients the fits would start by a deliberate spasm on one side of the body, and that different body parts would become systematically affected one after the other. He connected the orderly march of these spasms to the existence of localized lesions in the \emph{post-mortem} brain of his patients and hypothesized that the origin of these fits was uncontrolled excitation caused by local changes in cortical \emph{grey matter} \cite{Jackson1870}. In that same year, Fritsch and Hitzig published their famous study demonstrating that it is possible to elicit movements by direct stimulation of the cortex in dogs \cite{Fritsch1870}. Furthermore, stimulation of different parts of the cortex produced movement in different parts of the body \cite{Fritsch1870}. It appeared that the causal mechanism for epileptic convulsions predicted by Hughlings Jackson had been found, and with it a possible explanation for how the normal brain might control movement. The cerebral cortex was already considered at the time to be the seat of reasoning and sensation, so if activity over this so-called \emph{motor cortex} was able to exert direct control over the whole musculature of the body, then it might represent in the normal brain the area that connects volition to muscles \cite{Fritsch1870}.

\subsection{The Goltz-Ferrier Debates}

David Ferrier, a Scottish neurologist deeply impressed by the ideas of Hughlings Jackson and by the positive results of Fritsch and Hitzig's experiments, proceeded to reproduce and expand on their observations with comprehensive stimulation studies showing how activity in the motor cortex was sufficient to produce a large variety of movements across a wide range of mammalian species \cite{Ferrier1873}. Meanwhile, other researchers across Europe such as Goltz and Christiani were facing a dilemma: in many of the so-called ``lower mammals'' massive lesions of the cerebral cortex failed to demonstrate any visible long-term impairments in the motor behaviour of animals \cite{James1885,Goltz1888}.  These two lines of inquiry first clashed at the seventh International Medical Congress held in London in August 1881, where Goltz of Strassburg and Ferrier of London presented their results in a series of debates on the localization of function in the cerebral cortex \cite{Phillips1984,Tyler2000}.

Goltz assumed a clear anti-localizationist position. He advanced that it was impossible to produce a complete paresis of any muscle, or complete dysfunction of any perception, by destruction of any part of the cerebral cortex, and that he found mostly deficits of general intelligence in his dogs \cite{Tyler2000}. Following Goltz's presentation, Ferrier emphasized the danger of generalizing from the dog to animals of other orders (e.g. man and monkey). He then proceeded to exhibit his own lesion results by means of antiseptic surgery in the monkey, describing how a circumscribed unilateral lesion of the motor cortex produced complete contralateral paralysis of the leg. He also produced a striking series of microscopic sections of Wallerian degeneration \cite{Waller1850} of the ``motor path'' from the cortex to the contralateral spinal cord, the crossed descending projections forming the pyramidal corticospinal tract \cite{Tyler2000}.

The debates concluded with the public demonstration of live specimens: a dog with large lesions to the parietal and posterior lobes from Goltz; and from Ferrier, a hemiplegic monkey with a unilateral lesion to the motor cortex of the contralateral side. As predicted, Goltz's dog showed a clear ability to locomote and avoid obstacles and to make use of its other basic senses, while displaying peculiar deficits of intelligence such as failing to respond with fear to the cracking of a whip or ignoring tobacco smoke blown to its face. On the other hand, Ferrier's monkey showed up severely hemiplegic, in a condition similar to human stroke patients. After the demonstrations, the animals were killed and their brains removed. Preliminary observations revealed that the lesions in Goltz's dog were less extensive than expected, particularly on the left hemisphere. Ferrier's lesions on the other hand were precisely circumscribed to the contralateral motor cortex. These demonstrations secured the triumph of Ferrier, who went on to firmly establish the localizationist approach to neurology and the idea of a somatotopic arrangement over the motor cortex.

The Goltz-Ferrier debates had far-reaching implications throughout the entire research community of the time, and the basic dilemma that was presented has sparked controversy and confusion for over a hundred years since \cite{Phillips1984,Lashley1924,DeBarenne1933,Tyler2000,Gross2007}. In the meantime, views of motor cortex have evolved to suggest it plays a role in ``understanding'' the movements of others \cite{Rizzolatti2004}, imagining one's own movements \cite{Porro1996}, or in learning new movements \cite{Kawai2015}, but where are we today regarding its suggested primary role in directly controlling movement?

\subsubsection*{Stimulating motor cortex causes movement; motor cortex is active during movement}

Motor cortex is still broadly defined as the region of the cerebral hemispheres from which movements can be evoked by low-current stimulation, following Fritsch and Hitzig's original experiments in 1870 \cite{Fritsch1870}. Stimulating different parts of the motor cortex elicits movement in different parts of the body, and systematic stimulation surveys have revealed a topographical representation of the entire skeletal musculature across the cortical surface \cite{Leyton1917, Penfield1937, Neafsey1986}. Electrophysiological recordings in motor cortex have routinely found correlations between neural activity and many different movement parameters, such as muscle force \cite{Evarts1968}, movement direction \cite{Georgopoulos1986}, speed \cite{Schwartz1993}, or even anisotropic limb mechanics \cite{Scott2001} at the level of both single neurons \cite{Evarts1968,Churchland2007} and populations \cite{Georgopoulos1986,Churchland2012}. Determining what exactly this activity in motor cortex controls \cite{Todorov2000} has been further complicated by studies using long stimulation durations in which continuous stimulation at a single location in motor cortex evokes complex, multi-muscle movements \cite{Graziano2002,Aflalo2006}. However, as a whole, these observations all support the long standing view that activity in motor cortex is involved in the direct control of movement.

\subsubsection*{Motor cortex lesions produce different deficits in different species}

What types of movement require motor cortex? In humans, a motor cortical lesion is devastating. Permanent injury to the frontal lobes of the brain by stroke or mechanical means is often followed by weakness or paralysis of the limbs in the side of the body opposite to the lesion \cite{Louis1994}. Although the paretic symptoms have a tendency to recover partially by themselves, especially with training and rehabilitation, permanent movement deficits and loss of muscle control in the affected limbs is the common prognosis; movement is permanently and obviously impaired \cite{Laplane1977,Kwakkel2003}. In non-human primates, similar gross movement deficits are observed after lesions, albeit transiently \cite{Leyton1917,Travis1955}. The longest lasting effect of a motor cortical lesion is the decreased motility of distal forelimbs, especially in the control of individual finger movements required for precision skills \cite{Leyton1917,Darling2011}. But equally impressive is the extent to which other movements fully recover, including the ability to sit, stand, walk, climb and even reach to grasp, as long as precise finger movements are not required \cite{Leyton1917,Darling2011,Zaaimi2012}. In non-primate mammals, the absence of lasting deficits following motor cortical lesion is even more striking. Careful studies of skilled reaching in rats have revealed an impairment in paw grasping behaviours \cite{Whishaw1991,Alaverdashvili2008a}, comparable to the long lasting deficits seen in primates, but this is a limited impairment when compared to the range of movements that \emph{are} preserved \cite{Whishaw1991,Kawai2015}. In fact, even after complete decortication, rats, cats and dogs retain a shocking amount of their movement repertoire \cite{Goltz1888,Bjursten1976,Terry1989}. If we are to accept the simple hypothesis that motor cortex is the structure responsible for ``voluntary movement production'', then why is there such a blatant difference in the severity of deficits caused by motor cortical lesions in humans versus other mammals? With over a century of stimulation and electrophysiology studies clearly suggesting that motor cortex is involved in many types of movement, in all mammalian species, how can these divergent results be reconciled?
