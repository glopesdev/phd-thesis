\section{An Integrative View of the Motor System}

A different approach to the problems of motor control developed initially from studies on the integration of spinal reflexes conducted by the Sherrington school. While many researchers continued to look for the integration of complex movements in higher brain structures like the motor cortex, Sherrington turned instead to systematically characterizing anatomically and physiologically the distribution of efferent \cite{Sherrington1892} and afferent \cite{Sherrington1893a} nerve roots in the spinal cord of multiple species. His goal was to shed light on the so-called \emph{reflex arc}, the nerve pathways involved in muscular reactions like the knee-jerk whereby simple sensory stimuli elicit an immediate, automatic response from the animal, even in the absence of higher brain input \cite{Sherrington1893b}.

Sherrington and his contemporaries studied in detail a number of long and short spinal reflexes\footnote{A reflex action in which a stimulus applied to one region elicits a response in another region is termed a \emph{long spinal} reflex, whereas a reflex reaction where the muscular response happens in the same region as the stimulus is termed a \emph{short spinal} reflex.} in a variety of model organisms under different levels of anesthesia, pharmacological manipulations and spinal transsection \cite{Sherrington1903}. This systematic approach made abundantly clear a number of facts about how the nervous system organizes motor behaviour.

The first one, and perhaps the most striking, is that complex motor responses can be integrated and coordinated even in the complete absence of the brain \cite{Sherrington1906}. While the existence of automatisms and fixed action patterns had been recognized since antiquity, systematic stimulation studies in decerebrate animals quickly revealed that the reflex was far from being a rigid and fixed entity, but was rather adaptive and dynamic. In particular, reflex circuits revealed a much wider range of response characteristics than nerve fibers, which were well known since the time of Galvani to exhibit complete stereotypy in their response to a stimulus under various conditions\footnote{Some unique response characteristics of reflex arc conduction include irreversibility of the direction of conduction; fatigability and refractory period; greater variability of threshold; temporal facilitation with successive stimuli; a weaker correspondence of end-effect with intensity and frequency of the stimulus; and a greater susceptibility to metabolic and pharmacological manipulations \cite[p.14]{Sherrington1906}}.

Indeed, the motor output produced by the massively simplified spinal circuits was remarkably organized and displayed clear ethological meaning: adaptive behaviours such as reflex stepping, standing \cite{Sherrington1910, Sherrington1915}, scratching \cite{Sherrington1903}, or shaking \cite{Goltz1896, Sherrington1903} were all available to be elicited from stimulation of the isolated spinal system. Strikingly, these reflexes were also shown to be deployed and modulated appropriately to specific stimuli. The scratch reflex, for example, carries the foot roughly to the place of stimulation \cite{Sherrington1904}, and in reflex stepping the animal can maintain a rhythmic march through all phases of locomotion over unobstructed surfaces. Integration of these reflexes with input from the telereceptors is obviously entirely absent, but these observations clarified, beyond any reasonable doubt, that spinal cord circuits alone are sufficient to produce and sustain entire behaviour sequences under the right conditions. Furthermore, deafferentation experiments showed that aspects of these rhythmic network motifs persist even in the absence of sensory input \cite{GrahamBrown1911}. Many of these reflex circuits were later termed \emph{central pattern generators}, or CPGs \cite{Grillner1975, Grillner1981}, and found to be present across both vertebrate and invertebrate species \cite{Orlovsky1999,Selverston2010}.

\subsection{The Coordinative Role of Inhibition}

One of the aspects of spinal reflexes that most deeply impressed Sherrington was the general capacity of reflex circuits to initiate and switch between concurrent responses despite the existence of a \emph{final common path} from the nervous system to muscles \cite{Sherrington1904}. Motor neurons in the spinal cord send their axons through the ventral roots of spinal segments to synapse directly on muscle fibres. From his experiments, Sherrington showed that it was common to find multiple motor neurons participating synergistically or antagonistically in a single coordinated reflex response. More importantly, he revealed that the same motor neurons were actually shared among multiple, potentially conflicting, reflex arcs. Sherrington was fascinated by the fact that these antagonistic reflexes, initiated simultaneously from distinct sensory receptors, were still found to be able to coordinate their influence despite sharing this final common path to muscles. That such coordination existed was made clear by stimulation experiments where two or more reflexes were elicited at the same time, generating muscle responses to the combined stimulation that were not a simple summation or linear combination of the responses obtained by stimulation delivered in isolation. Sherrington describes the conception clearly:

\blockquote[{\protect\cite[p.461]{Sherrington1904}}]{Take the primary retinal reflex, which moves the eye so as to bring the fovea to the situation of the stimulating image. From all the receptors in each lateral retinal half rise reflex arcs with a final common path in the nerve of the opposite \emph{rectus lateralis}. Suppose simultaneous stimulation of two of these retinal points, one nearer to, one farther from, the fovea. If the arcs of both points pour their impulses into the final common path together, the effect must be a resultant of the two discharges. If these sum, the shortening of the muscle will be too great and the fovea swing too far for either point. If the resultant be a compromise between the two individual points, the fovea will come to lie between the two points of stimulation. In both cases the result obtained would be useless for the purposes of either\ldots.

When two stimuli are applied simultaneously which would evoke reflex actions that employ the same final common path in different ways, in my experience one reflex appears without the other. The result is this reflex or that reflex, but not the two together.}

In Sherrington's time the existence of such common paths was a problem for the classic view of reflex control, where the function of the nervous system was conceived in terms of nerve conduction of excitatory impulses. The existence of the final common path mediating multiple reflexes made it necessary to speak openly of the problem of how to coordinate different circuit elements and to describe mechanisms that would allow the same neurons to take on context-dependent roles in generating motor responses. It was during the hunt for such a mechanism of reflex arc coordination that Sherrington hit upon the fundamental role of inhibition in the organization of neural function. Inhibition had always been a complicated topic for physiologists, but following the demonstration of cardiac muscle inhibition by the vagus nerve \cite{Weber1846}, and Sechenov's grand proposal of a central origin of reflex inhibition \cite{Sechenov1863}, Sherrington was able to articulate and experimentally validate simple mechanisms for spinal reflex coordination, not only at the level of reciprocal inhibition of antagonistic muscles \cite{Sherrington1893b}, but also at the level of coordination and maintenance of the so-called central state.

Specifically, Sherrington described the existence of a central inhibitory state that was at least as, if not more, important than central excitatory state. He emphasized the need for discrete regulatory mechanisms in the nervous system which were capable of deciding how much converging neural impulses would influence a target cell. From physiological experiments measuring the latency and amplitude of neural responses in very short reflex arcs, he surmised that the effect could not be a simple \emph{absence} of excitation or shutting down of function, but rather the presence of a fundamentally active force in the nervous system \cite{Sherrington1965}. He emphasized that the major difficulty in studies of inhibition was the fact that it can only be actively measured in comparison to a baseline of excitation. If this baseline is not set up accurately, inhibition and its effects can be easy to miss. He pinned down the effects of inhibitory state to a theoretical gap between two neural cells where the nature of conduction changed fundamentally. He named this junction the \emph{synapse}.

Today we know from several detailed descriptions of invertebrate CPGs that different spinal circuits can be delicately super-imposed on the same neuronal elements by using a number of strategies to regulate and coordinate function \cite{Orlovsky1999,Selverston2010}. These strategies span all levels of neural organization from networks to molecules, ranging from reciprocal inhibition network motifs gated by specific neuromodulators and cell membrane receptors to overlapping distributions of voltage and ligand gated ion channels with different kinetics. The existence of such finely tuned spinal networks brings significant constraints when thinking about cortical motor control. Ultimately, any descending signals from the brain to muscles are also sharing this final common path and thus should be expected to require some form of coordination with spinal circuits if behavioural output is to remain integrated.

Another important implication of the idea of central inhibition was how it could explain the physiological and behavioural changes during phenomena of ``shock'' and ``release'', a generalized depression of activity or over-action in a given area following injury or destruction to distant but related parts of the nervous system. Sherrington described how transient changes in excitatory and inhibitory state ultimately manifest themselves depending on the particular anatomical situation of the influencing and influenced centres. In particular, he emphasized how the presence of such transient changes can be enough to establish that two areas are connected, but how these transient effects can obscure the function of either area.

% Give a detailed example of a mechanism of reciprocal innervation explaining coordination
% Talk about the difficulty of measuring inhibition: you need to compare it to a baseline of excitation
% Talk about inhibition as an "active" force in the nervous system. A "pulse" of inhibition is as much an input as a pulse of "excitation".
% Talk about the synapse as the location for the mechanism of inhibition (include stuff on the discovery of the synapse)

% The second one, the final common path and the need of neural coordination at the output
% The third one, inhibition is as active a force in the nervous system as excitation
% The fourth one, response latencies and implications for motor control (and synapses)

However, reproduction of the stimulation experiments across many mammalian species \cite{Ferrier1873,Clark1937} revealed over time a number of subtle observations. First, rather than merely activating individual muscles, prolonged stimulation at individual cortical sites reliably evoked behaviourally relevant, purposeful actions \cite{Ferrier1873,Clark1937}. Second, if the stimulation electrodes were held in the same fixed cortical point, repeated pulse trains could be made to elicit different responses if the configuration of the limbs or head was changed \cite{Ward1938}. This variability in responses to cortical stimulation was much higher than in spinal or decerebrate preparations and was thus termed the ``instability of the cortical point'' \cite{GrahamBrown1912,Leyton1917}.

However, it is important to note that apart from CPG-like network patterns in the spinal cord, there is also a general capacity of spinal circuits as a whole to initiate and switch between the different responses, as well as modulate network activity with incoming sensory input in order to produce adaptive behaviour responses \cite{Forssberg1975}.

\subsection{The Instability of Cortical Points}

One of the most distinguishing physiological characteristics of mammalian motor cortex is the ability to evoke movements in different parts of the skeletal musculature by electrical stimulation of different motor cortical areas. Following the discovery of this rough somatotopy, several researchers developed microstimulation mapping protocols, where low-amplitude, short duration pulses of electrical current are delivered systematically across many different predetermined cortical sites arranged in a grid lattice spanning the entire motor cortex. In this way it was possible to establish the precise somatotopical organization of motor cortex across many different species, including humans.

However, difficulties in interpreting the function of these stimulation fields in the normal behaving animal have been pointed out since the early stimulation experiments. First, it was noted that many of the cortical stimulation points were ``unstable''. Specifically, if the stimulating electrode is kept in the same place and the stimulation protocol repeated, the variability in evoked movements was found to be much larger than what is normally encountered with stimulation of efferent nerve roots at the level of the spinal cord. Second, while microstimulation protocols are able to evoke single muscle twitches, the application of longer stimulation trains can trigger the release of complex, integrated movement sequences which are behaviourally meaningful. In the recent experiments by Graziano, it was demonstrated that not only the evoked movements were all part of the animal's natural behaviour repertoire, but they appeared to be goal-directed and to take into account the current state of the musculature. For example, a reaching action evoked by stimulation of a single cortical point would bring the hand to grasp a specific region in space. If the stimulating electrode was kept in the same place and the arm of the animal moved to a different configuration, the stimulation would now cause different effects in the animal musculature, but which nevertheless had the result of bringing the hand to grasp the same point in space.

Critical revision of any scientific theory requires pushing against the fringe of established fact, which derives often from a certain stubbornness to explain lingering contradictions in a satisfactory manner. It is precisely the appreciation of these contradictions that provides fertile ground for seeding new perspectives of the phenomenon under study. To this end, it is best to state explicitly and concisely the basic assumptions and facts of current theory, so that contrary evidence can be seen to stand out clearly.



Old observations can be reinterpreted in the light of a new perspective which will subsequently predict and inform the results of future experiments. The capacity of the nervous system to fully integrate behaviour acts in the absence of higher brain structures is an example of one of the largest such revisions in systems neuroscience. For over a century, physiologists had explored spinal reflexes using localized stimulation of the endings of nerve fibres.

% One of the major points of contention were the differences between nerve trunk conduction and reflex arc conduction. The speed of propagation of the action potential across a nerve fiber had been known since the experiments of Helmholtz \cite{Helmholtz1850,Schmidgen2002} so it was possible to estimate the expected latency for a short spinal reflex, assuming reasonable estimates of nerve length and mechanical latency. For example, in the flexion-reflex of the dog's hind limb this latency was estimated to be around \SI{27}{\milli\second} \cite[p.19]{Sherrington1906}. However, the observed latencies in the spinal animal were, under normal conditions, found to be at least double of this number. Furthermore, reflex arc conduction had many other characteristics that were distinctive from the well studied nerve trunk conduction. Besides longer latencies, reflex arc conduction exhibited irreversibility of direction of conduction; fatigability and refractory period; greater variability of threshold; temporal facilitation with successive stimuli; a weaker correspondence of end-effect with intensity and frequency of the stimulus; and a greater susceptibility to metabolic and pharmacological manipulations \cite[p.14]{Sherrington1906}.

% These differences made Sherrington realize the necessity for a change in the physical medium of conduction at some point along the arc. The existence of such gaps clashed with the dominant idea at the time which viewed the nervous system as a continuous conductive nerve net. However, with the development of Golgi staining, Ramon y Cajal had recently been able to finally introduce convincingly the idea that the nervous system was actually composed of discrete cellular units, \emph{neurons} \cite{RamonYCajal1894}. Sherrington became a supporter of the newly established neuron doctrine and posited that the required theoretical junction between two neurons, which he termed the ``\emph{synapse}'' \cite{Foster1897}, could explain the unique physiology of reflex arc conduction.

Today we know from several detailed descriptions of invertebrate CPGs that different spinal circuits can be delicately super-imposed on the same neuronal elements by using a number of strategies to regulate and coordinate function \cite{Orlovsky1999,Selverston2010}. These strategies span all levels of neural organization from networks to molecules, ranging from reciprocal inhibition network motifs gated by specific neuromodulators and cell membrane receptors to overlapping distributions of voltage and ligand gated ion channels with different kinetics. The existence of such finely tuned spinal networks brings significant constraints when thinking about cortical motor control. Ultimately, any descending signals from the brain to muscles are also sharing this final common path and thus should be expected to require some form of coordination with spinal circuits if behavioural output is to remain integrated.

\subsubsection*{A role in modulating the movements generated by lower motor centres}

A different perspective on motor cortex emerged from studying the neural control of locomotion, suggesting that the corticospinal tract plays a role in the \emph{adjustment} of ongoing movements that are generated by lower motor systems. In this view, rather than motor cortex assuming direct control over muscle movement, it instead modulates the activity and sensory feedback in spinal circuits in order to adapt a lower movement controller to challenging conditions. This idea that the descending cortical pathways superimpose speed and precision on an existing baseline of behaviour was also suggested by lesion work in primates \cite{Lawrence1968a}, but has been investigated most thoroughly in the context of cat locomotion.

It has been known for more than a century that completely decerebrate cats are capable of sustaining the locomotor rhythms necessary for walking on a flat treadmill utilizing only spinal circuits \cite{GrahamBrown1911}. Brainstem and midbrain circuits are sufficient to initiate the activity of these spinal central pattern generators \cite{Grillner1973}, so what exactly is the contribution of motor cortex to the control of locomotion? Single-unit recordings of pyramidal tract neurons (PTNs) from cats walking on a treadmill have shown that a large proportion of these neurons are locked to the step cycle \cite{Armstrong1984a}. However, we know from the decerebrate studies that this activity is not necessary for the basic locomotor pattern. What then is its role?

Lesions of the lateral descending pathways (containing corticospinal and rubrospinal projections) produce a long term impairment in the ability of cats to step over obstacles \cite{Drew2002}. Recordings of PTN neurons during locomotion show increased activity during these visually guided modifications to the basic step cycle \cite{Drew1996}. These observations suggest that motor cortex neurons are necessary for precise stepping and adjustment of ongoing locomotion to changing conditions. However, long-term effects seem to require complete lesion of \emph{both} the corticospinal and rubrospinal tracts \cite{Drew2002}. Even in these animals, the voluntary act of stepping over an obstacle does not disappear entirely, and moreover, they can adapt to changes in the height of the obstacles \cite{Drew2002}. Specifically, even though these animals never regain the ability to gracefully clear an obstacle, when faced with a higher obstacle, they are able to adjust their stepping height in such a way that would have allowed them to comfortably clear the lower obstacle \cite{Drew2002}. Furthermore, deficits caused by lesions restricted to the pyramidal tract seem to disappear over time \cite{Liddell1944}, and are most clearly visible only the first time an animal encounters a new obstacle \cite{Liddell1944}.

The view that motor cortex in non-primate mammals is principally responsible for adjusting ongoing movement patterns generated by lower brain structures is appealing. What is this modulation good for? What does it allow an animal to achieve? How can we assay its necessity?
