\section{An Integrative View of the Motor System}

A different approach to the problems of motor control developed initially from studies on the integration of spinal reflexes conducted by the Sherrington school. While many researchers continued to look for the integration of complex movements in higher brain structures like the motor cortex, Sherrington turned instead to systematically characterizing anatomically and physiologically the distribution of efferent \cite{Sherrington1892} and afferent \cite{Sherrington1893a} nerve roots in the spinal cord of multiple species. His goal was to shed light on the so-called \emph{reflex arc}, the nerve pathways involved in muscular reactions like the knee-jerk whereby simple sensory stimuli elicit an immediate, automatic response from the animal, even in the absence of higher brain input \cite{Sherrington1893b}.

Sherrington and his contemporaries studied in detail a number of long and short spinal reflexes\footnote{A reflex action in which a stimulus applied to one region elicits a response in another region is termed a \emph{long spinal} reflex, whereas a reflex reaction where the muscular response happens in the same region as the stimulus is termed a \emph{short spinal} reflex.} in a variety of model organisms under different levels of anesthesia, pharmacological manipulations and spinal transsection \cite{Sherrington1903}. This systematic approach made abundantly clear a number of facts about how the nervous system organizes motor behaviour.

The first one, and perhaps the most striking, is that complex motor responses can be integrated and coordinated even in the complete absence of the brain \cite{Sherrington1906}. While the existence of automatisms and fixed action patterns had been recognized since antiquity, systematic stimulation studies in decerebrate animals quickly revealed that the reflex was far from being a rigid and fixed entity, but was rather adaptive and dynamic. In particular, reflex circuits revealed a much wider range of response characteristics than nerve fibers, which were well known since the time of Galvani to exhibit complete stereotypy in their response to a stimulus under various conditions\footnote{Some unique response characteristics of reflex arc conduction include irreversibility of the direction of conduction; fatigability and refractory period; greater variability of threshold; temporal facilitation with successive stimuli; a weaker correspondence of end-effect with intensity and frequency of the stimulus; and a greater susceptibility to metabolic and pharmacological manipulations \cite[p.14]{Sherrington1906}}.

Indeed, the motor output produced by the massively simplified spinal circuits was remarkably organized and displayed clear ethological meaning: adaptive behaviours such as scratching \cite{Sherrington1903}, shaking \cite{Goltz1896, Sherrington1903} or reflex stepping and standing \cite{Sherrington1910, Sherrington1915} were all available to be elicited from stimulation of the isolated spinal system. Strikingly, these reflexes were also shown to be deployed and modulated appropriately to specific stimuli. The scratch reflex, for example, carries the foot roughly to the place of stimulation \cite{Sherrington1904}, and in reflex stepping the animal can maintain a rhythmic march through all phases of locomotion over unobstructed surfaces \cite{Sherrington1910}. Integration of these reflexes with input from the telereceptors is obviously entirely absent, but these observations clarified, beyond any reasonable doubt, that spinal cord circuits alone are sufficient to produce and sustain entire behaviour sequences under the right conditions. Furthermore, deafferentation experiments showed that aspects of these rhythmic network motifs persist even in the absence of sensory input \cite{GrahamBrown1911}. Many of these reflex circuits were later termed \emph{central pattern generators}, or CPGs \cite{Grillner1975, Grillner1981}, and found to be present across both vertebrate and invertebrate species \cite{Orlovsky1999,Selverston2010}.

\subsection{The Coordinative Role of Inhibition}

One of the aspects of spinal reflexes that most deeply impressed Sherrington was the general capacity of reflex circuits to initiate and switch between concurrent responses despite the existence of a \emph{final common path} from the nervous system to muscles \cite{Sherrington1904}. Motor neurons in the spinal cord send their axons through the ventral roots of spinal segments to synapse directly on muscle fibres. From his experiments, Sherrington showed that it was common to find multiple motor neurons participating synergistically or antagonistically in a single coordinated reflex response. More importantly, he revealed that the same motor neurons were actually shared among multiple, potentially conflicting, reflex arcs. Sherrington was fascinated by the fact that these antagonistic reflexes, initiated simultaneously from distinct sensory receptors, were still found to be able to coordinate their influence despite sharing this final common path to muscles. That such coordination existed was made clear by stimulation experiments where two or more reflexes were elicited at the same time, generating muscle responses to the combined stimulation that were not a simple summation or linear combination of the responses obtained by stimulation delivered in isolation. Sherrington describes the conception clearly:

\blockquote[{\protect\cite[p.461]{Sherrington1904}}]{Take the primary retinal reflex, which moves the eye so as to bring the fovea to the situation of the stimulating image. From all the receptors in each lateral retinal half rise reflex arcs with a final common path in the nerve of the opposite \emph{rectus lateralis}. Suppose simultaneous stimulation of two of these retinal points, one nearer to, one farther from, the fovea. If the arcs of both points pour their impulses into the final common path together, the effect must be a resultant of the two discharges. If these sum, the shortening of the muscle will be too great and the fovea swing too far for either point. If the resultant be a compromise between the two individual points, the fovea will come to lie between the two points of stimulation. In both cases the result obtained would be useless for the purposes of either\ldots.

When two stimuli are applied simultaneously which would evoke reflex actions that employ the same final common path in different ways, in my experience one reflex appears without the other. The result is this reflex or that reflex, but not the two together.}

In Sherrington's time the existence of such common paths was a problem for the classic view of reflex control, where the function of the nervous system was conceived in terms of nerve conduction of excitatory impulses. The existence of the final common path mediating multiple reflexes made it necessary to speak openly of the problem of how to coordinate different circuit elements and to describe mechanisms that would allow the same neurons to take on context-dependent roles in generating motor responses. It was during the hunt for such a mechanism of reflex arc coordination that Sherrington hit upon the fundamental role of inhibition in the organization of neural function. Inhibition had always been a complicated topic for physiologists, but following the demonstration of cardiac muscle inhibition by the vagus nerve \cite{Weber1846}, and Sechenov's grand proposal of a central origin of reflex inhibition \cite{Sechenov1863}, Sherrington was able to articulate and experimentally validate simple mechanisms for spinal reflex coordination, not only at the level of reciprocal inhibition of antagonistic muscles \cite{Sherrington1893b}, but also at the level of coordination and maintenance of the so-called central state.

Specifically, Sherrington described the existence of a central inhibitory state that was at least as, if not more, important than central excitatory state. He emphasized the need for discrete regulatory mechanisms in the nervous system which were capable of deciding how much converging neural impulses would influence a target cell. From physiological experiments measuring the latency and amplitude of neural responses in very short reflex arcs, he surmised that the effect could not be a simple \emph{absence} of excitation or shutting down of function, but rather the presence of a fundamentally active force in the nervous system \cite{Sherrington1965}. He emphasized that the major difficulty in studies of inhibition was the fact that it can only be actively measured in comparison to a baseline of excitation. If this baseline is not set up accurately, inhibition and its effects can be easy to miss. Sherrington was one of the first to suggest that the effects of central inhibitory state acted primarily on the theoretical gap between two neural cells where the nature of conduction changed fundamentally. He named this junction the \emph{synapse} \cite[p.929]{Foster1897}.

Today we know from several detailed descriptions of invertebrate CPGs that different spinal circuits can be delicately super-imposed on the same neuronal elements by using a number of strategies to regulate and coordinate function \cite{Orlovsky1999,Selverston2010}. These strategies span all levels of neural organization from networks to molecules, ranging from reciprocal inhibition network motifs gated by specific neuromodulators and cell membrane receptors to overlapping distributions of voltage and ligand gated ion channels with different kinetics. The existence of such finely tuned spinal networks brings significant constraints when thinking about cortical motor control. Ultimately, any descending signals from the brain to muscles are also sharing this final common path and thus should be expected to require some form of coordination with spinal circuits if behavioural output is to remain integrated.

One final implication of the idea of central inhibition was an explanation for the physiological and behavioural changes occurring during phenomena of ``shock'' and ``release'', states of generalized depression of activity or over-action in a given area following injury or destruction to distant but related parts of the nervous system. For example, it was well known at the time that circuits in the lower segments of the cat spinal cord underwent ``shock'' following transsection of fiber tracts projecting from higher segments, or extended lesions to the brain, such as decortication, or decerebration up to the level of the brainstem. The generalized inhibitory state that ensued masked the expression of reflexes organized by spinal centres. Following some period of time, however, the spinal preparation would overcome shock and reflex responses could be reliably and repeatedly elicited. Sherrington and his co-workers described systematically how transient changes in excitatory and inhibitory state ultimately manifest themselves depending on the particular anatomical situation of the influencing and influenced centres. From these results, they emphasized how the presence of such transient changes can be enough to establish that two areas are connected, but complicate inference of the functions located in either area.

\subsection{Hierarchical Organization of Motor Behaviour}

The study of the biological mechanisms underlying adaptive reflexes inspired multiple theories for the organization of behaviour, including Hebbian plasticity \cite{Hebb1949} and cybernetics \cite{Wiener1948}. In cybernetics, the term ``feedback'' replaced the concept of reflex in technical publications, but the essence of the abstract formulation---the coupling of sensory inputs to motor effectors as the basic unit of adaptive behaviour---remained largely the same. The school of thought initiated by Norbert Wiener, Arturo Rosenblueth, Ross Ashby and others, would come to formulate general properties of adaptive control using negative feedback \cite{Rosenblueth1943,Wiener1948,Ashby1960}. Specifically, they clarified formally how to design systems that can maintain stable relationships with the environment by comparing incoming sensory input to an internal reference and using the result of the comparison to drive motor responses (i.e. the intensity and direction of motor response should be such as to minimize the difference between the internal reference and the sensory input). Such systems were found to automatically generate adaptive motor outputs that stabilized sensory inputs even in the face of unexpected disturbances \cite{Wiener1948,Ashby1960} and have since become a staple of the automation industry and robotic control.

Inspired by the success of cybernetics, several writers proposed mechanisms for the adaptive hierarchical integration of multiple such feedback (or reflex) controllers \cite{Powers1973}. The basic principle of organization was to make the output of a higher level system effectively adjust the internal references of the feedback controllers at the lower level. In this way, the higher level system can act as a regulator without the need to control the details of the behaviour of the lower system. Two basic implications can be derived from this conception. First, the removal of the higher level system does not necessarily compromise the regulatory abilities of the lower systems, which can continue to operate independently. Second, fixing the output of the higher level system will not necessarily fix the output of the lower system, but only its internal reference. The final motor output will also depend on the state of the sensory input to the lower system, and will be regulated appropriately.

The first implication agrees well with the results of motor cortical lesions in non-human mammals, where the vast majority of the behavioural repertoire is preserved \cite{Goltz1888,Grunbaum1903,GrahamBrown1913,Leyton1917,Bjursten1976,Terry1989,Whishaw1991,Darling2011,Zaaimi2012,Kawai2015}. In this view, if subcortical systems are seen as independent controller mechanisms with cortex being their highest level regulator, then it would not be surprising if all of the subcortical capacities would be preserved following extirpation of cortical tissue. In addition, successive decerebration techniques would be in principle capable of revealing the regulatory limits of each stage in the hierarchy, assuming one starts at the lowest level.

As for the second implication, Sherrington himself provided very suggestive evidence in the form of the phenomenon of instability of stimulated cortical points \cite{GrahamBrown1912,Leyton1917}. When systematically mapping the somatotopic organization of the motor cortex by low-current electrical stimulation, it was noted that many of the cortical stimulation points were ``unstable''. Specifically, if the stimulating electrode was kept in the same place and the stimulation protocol repeated, there was a large variability in evoked movements, much larger than what was normally encountered in the study of purely spinal or decerebrate reflexes \cite{GrahamBrown1912}. For example, a cortical point stimulated in rapid succession could reliably produce a flexion response in one isolated muscle in the start of the series, only to completely reverse its response to extension of the same muscle towards the end of the series \cite{GrahamBrown1912}. In addition, even the boundaries of the excitable area itself, as mapped by stimulation, were subject to dramatic change in the same animal, depending on whether the stimulation protocol progressed in the anterior-posterior or posterior-anterior direction \cite{Leyton1917}. Studies by later authors would suggest that the instability of the point was related to changes in the configuration of the limbs or head, concluding that the stimulation enforced a postural target rather than generating a specific motor response \cite{Ward1938}.

This implication is also in agreement with studies using prolonged stimulation of individual cortical sites, which were found to reliably evoke entire sequences of behaviourally relevant, purposeful actions \cite{Ferrier1873,Clark1937,Graziano2002}. These actions were also found to depend on the current postural configuration of the animal, representing behaviour sequences such as hand-to-mouth or reaching a point in space \cite{Graziano2002}. If the stimulating electrode was kept in the same place and the arm of the animal moved to a new starting location, the stimulation would now cause different effects in the animal musculature, but which nevertheless had the result of bringing the hand to the same specific point in space \cite{Graziano2002}.

Although there are many limitations and even hugely missing gaps in this hierarchical view of motor system organization, the main insight that can be derived from such lines of thinking is that motor cortex does not need to be seen exclusively as a direct driver of motor behaviour, but rather its influences on the lower motor system may be much more subtle and modulatory than was previously appreciated. If this is true, then the question of what exactly is the teleology for cortical motor control becomes all the more difficult. Specifically, it may not be enough to ask which movements does motor cortex control. We may need to ask more generally what is motor cortex doing and what does it allow the organism to achieve over and above the regulatory behaviour patterns managed and adapted by subcortical systems such as cerebellum and brainstem. The capacity of the subcortical systems should not be underestimated, as they have undergone millions of years of evolution and are still used almost exclusively for the control of behaviour in many vertebrate species.

\subsubsection*{A role in modulating the movements generated by lower motor centres}

The idea that the descending cortical pathways superimpose speed and precision on an existing baseline of behaviour has been suggested even in some lines of lesion work in the primate \cite{Lawrence1968a}, but has been investigated much more thoroughly in the context of studies on the neural control of locomotion in cats. These studies have suggested that the corticospinal tract can play a role in the \emph{adjustment} of ongoing movements, modulating the activity and sensory feedback in spinal circuits in order to adapt a lower movement controller to challenging conditions.

It has been known for more than a century that completely decerebrate cats are capable of sustaining the locomotor rhythms necessary for walking on a flat treadmill utilizing only spinal circuits \cite{GrahamBrown1911}. In addition, there is a general capacity for spinal circuits to modulate network activity with incoming sensory input in order to coordinate and switch between different responses, even during specific phases of movement \cite{Forssberg1975}. Brainstem and midbrain circuits are sufficient to initiate the activity of these spinal central pattern generators \cite{Grillner1973}, so what exactly is the contribution of motor cortex to the control of locomotion? Single-unit recordings of pyramidal tract neurons (PTNs) from cats walking on a treadmill have shown that a large proportion of these neurons are locked to the step cycle \cite{Armstrong1984a}. However, we know from the decerebrate studies that this activity is not necessary for the basic locomotor pattern. What then is its role?

Lesions of the lateral descending pathways (containing corticospinal and rubrospinal projections) produce a long term impairment in the ability of cats to step over obstacles \cite{Drew2002}. Recordings of PTN neurons during locomotion show increased activity during these visually guided modifications to the basic step cycle \cite{Drew1996}. These observations suggest that motor cortex neurons are necessary for precise stepping and adjustment of ongoing locomotion to changing conditions. However, long-term effects seem to require complete lesion of \emph{both} the corticospinal and rubrospinal tracts \cite{Drew2002}. Even in these animals, the voluntary act of stepping over an obstacle does not disappear entirely, and moreover, they can adapt to changes in the height of the obstacles \cite{Drew2002}. Specifically, even though these animals never regain the ability to gracefully clear an obstacle, when faced with a higher obstacle, they are able to adjust their stepping height in such a way that would have allowed them to comfortably clear the lower obstacle \cite{Drew2002}. Furthermore, deficits caused by lesions restricted to the pyramidal tract seem to disappear over time \cite{Liddell1944}, and are most clearly visible only the first time an animal encounters a new obstacle \cite{Liddell1944}.

The view that motor cortex in non-primate mammals is principally responsible for adjusting ongoing movement patterns generated by lower brain structures is appealing. What is this modulation good for? What does it allow an animal to achieve? How can we assay its necessity?

\subsubsection*{Towards a new teleology; new experiments required}

It should now be clear that the involvement of motor cortex in the direct control of all ``voluntary movement'' is human-specific. There is a role for motor cortex across mammals in the control of precise movements of the extremities, especially those requiring individual movements of the fingers, but these effects are subtle in non-primate mammals. Furthermore, what would be a devastating impairment for humans may not be so severe for mammals that do not depend on precision finger movements for survival. Therefore, generalizing this specific role of motor cortex from humans to all other mammals would be misleading. We could be missing another, more primordial role for this structure that predominates in other mammals, and by doing so, we may also be missing an important role in humans.

The proposal that motor cortex induces modifications of ongoing movement synergies, prompted by the electrophysiological studies of cat locomotion, definitely points to a role consistent with the results of various lesion studies. However, in assays used, the ability to modify ongoing movement generally recovers after a motor cortical lesion. What are the environmental situations in which motor cortical modulation is most useful?

Cortex has long been proposed to be the structure responsible for integrating a representation of the world and improving the predictive power of this representation with experience \cite{Barlow1985,Doya1999}. If motor cortex is the means by which these representations can gain influence over the body, however subtle and ``modulatory'', can we find situations (i.e. tasks) in which this cortical control is required?

The necessity of cortex for various behavioural tasks has been actively investigated in experimental psychology for over a century, including the foundational work of Karl Lashley and his students \cite{Lashley1921a,Lashley1950a}. In the rat, large cortical lesions were found to produce little to no impairment in movement control, and even deficits in learning and decision making abilities were difficult to demonstrate consistently over repeated trials. However, Lashley did notice some evidence that cortical control may be involved in postural adaptations to unexpected perturbations \cite{Lashley1921a}. These studies once again seem to recapitulate the two most consistent observations found across the entire motor cortical lesion literature in non-primate mammals since Hitzig \cite{Fritsch1870}, Goltz \cite{Goltz1888}, Sherrington \cite{Sherrington1885} and others \cite{Oakley1979,Terry1989}. One, direct voluntary control over movement is most definitely not abolished through lesion; and two, certain aspects of some movements are definitely impaired, but only under certain challenging situations. The latter are often reported only anecdotally. It was this collection of intriguing observations in animals with motor cortical lesions that prompted us to expand the scope of standard laboratory tasks to include a broader range of motor control challenges that brains encounter in their natural environments.
