\section{Hierarchy of Feedback Controllers}

\subsection{The Reflex Arc and the Principle of the Final Common Path}

After his return to Cambridge, Sherrington mounted his attack on the problems of nerve conduction from the periphery. While many researchers continued to look for the integration of complex movements in higher brain structures like the motor cortex, Sherrington turned instead to systematically characterizing anatomically and physiologically the distribution of efferent \cite{Sherrington1892} and afferent \cite{Sherrington1893a} nerve roots in the spinal cord of multiple species. His goal was to shed light on the so-called \emph{reflex arc}, the nerve pathways involved in muscular reactions like the knee-jerk whereby simple sensory stimuli elicit an immediate, automatic response from the animal, even in the absence of higher brain input \cite{Sherrington1893b}.

Sherrington and his contemporaries studied in detail a number of long and short spinal reflexes\footnote{A reflex action in which a stimulus applied to one region elicits a response in another region is termed a \emph{long spinal} reflex, whereas a reflex reaction where the muscular response happens in the same region as the stimulus is termed a \emph{short spinal} reflex.} in a variety of model organisms under different levels of anesthesia, pharmacological manipulations and spinal transsection \cite{Sherrington1903}. This systematic approach made abundantly clear a number of facts about how the nervous system organizes motor control.

The first one, and perhaps the most striking, is that complex motor responses can be integrated and coordinated even in the complete absence of the brain \cite{Sherrington1906}. From early on it became clear that the most basic unit of integration, the reflex, was far from being a rigid and fixed entity, but was rather adaptive and dynamic. All reflex circuits revealed a much broader range of response characteristics than nerve fibers, which were well known since the time of Galvani to exhibit complete stereotypy in their response to a stimulus under various conditions\footnote{Some unique response characteristics of reflex arc conduction include irreversibility of the direction of conduction; fatigability and refractory period; greater variability of threshold; temporal facilitation with successive stimuli; a weaker correspondence of end-effect with intensity and frequency of the stimulus; and a greater susceptibility to metabolic and pharmacological manipulations \cite[p.14]{Sherrington1906}}. Furthermore, the displayed muscular responses could be interpreted as having ethological meaning: reflex stepping and standing \cite{Sherrington1910, Sherrington1915}, scratching \cite{Sherrington1903}, or shaking \cite{Goltz1896, Sherrington1903} are just some examples of the adaptive behaviour repertoire found in spinal animals. These reflexes were also shown to be deployed and modulated appropriately to specific stimuli. The execution of the scratch reflex, for example, depends on where in the skin the stimulus is delivered, and in reflex stepping the animal can maintain a rhythmic march through all phases of locomotion over unobstructed surfaces. Integration of these reflexes with input from the telereceptors is obviously entirely absent, but these observations even today should raise awareness to the fact that spinal cord circuits are sufficient to produce and sustain entire behaviour sequences under the right conditions.

Deafferentation experiments showed that aspects of these rhythmic network motifs persist even in the absence of sensory input \cite{GrahamBrown1911}. Many of these reflex circuits were later termed \emph{central pattern generators}, or CPGs \cite{Grillner1975, Grillner1981}, and found to be present across both vertebrate and invertebrate species \cite{Orlovsky1999,Selverston2010}. However, it is important to note that apart from CPG-like network patterns in the spinal cord, there is also a general capacity of spinal circuits as a whole to initiate and switch between the different responses, as well as modulate network activity with incoming sensory input in order to produce adaptive behaviour responses \cite{Forssberg1975}.

Another aspect of the neural control of behaviour emphasized by Sherrington was the existence of a \emph{final common path} from the nervous system to muscles \cite{Sherrington1904}. Motor neurons in the spinal cord send their axons through the ventral roots of spinal segments to synapse directly on muscle fibres. From his experiments, Sherrington emphasized that these motor neurons are actually shared by reflex arcs initiated by distinct sensory receptors. In Sherrington's time the existence of such common paths was a problem for reflex control, especially if the function of the nervous system was conceived in terms of nerve conduction. The idea of the final common path made it necessary to speak openly of coordination between different circuit elements and to describe mechanisms that would allow the same neurons to take on context-dependent roles in generating motor behaviour.

\subsection{The Coordinative Role of Inhibition}

In search of a mechanism for reflex arc coordination, Sherrington hit upon the fundamental role of inhibition in the organization of neural function. Inhibition had always been a complicated topic for physiologists, but following up on the demonstration of cardiac muscle inhibition by the vagus nerve \cite{Weber1846}, and Sechenov's grand proposal of a central origin of reflex inhibition \cite{Sechenov1863}, Sherrington was able to hypothesize and experimentally validate simple mechanisms for spinal reflex coordination, such as the law of reciprocal inhibition of antagonistic muscles \cite{Sherrington1893b}.

However, Sherrington's experiments went much beyond the simple antagonism of opponent muscles and also probed the central role of inhibition in the coordination of conflicting reflex responses initiated by distinct sensory receptors. These antagonistic reflexes were found to be able to coordinate their influence despite sharing a final common path to muscles. That such coordination existed was made clear by stimulation experiments where two or more reflexes were elicited at the same time. In these experiments, Sherrington showed that the muscle response to combined stimulation was not a simple summation or linear combination of the responses obtained by stimulation delivered in isolation. Rather, as he describes:

\blockquote[{\protect\cite[p.461]{Sherrington1904}}]{When two stimuli are applied simultaneously which would evoke reflex actions that employ the same final common path in different ways, in my experience one reflex appears without the other. The result is this reflex or that reflex, but not the two together.}

Sherrington's investigations of inhibition in the coordination of reflexes again produced a number of invaluable insights for thinking about cortical motor control. The first one is the realization of inhibition as a powerful active force in the nervous system \cite{Sherrington1965}. Rather than simply being the \emph{absence} of excitation or shutting down of function, inhibition was found to play a fundamental role in the coordination and maintenance of central state. A major difficulty that physiologists found with inhibition is that it can only be actively measured in comparison to a baseline of excitation. If this baseline is not set up accurately, inhibition and its effects can be easy to miss.

Another important realization is how transient changes in inhibitory and excitatory state can explain phenomena of ``shock'' and ``release'', a generalized depression of activity or over-action in a given area following injury or destruction to distant but related parts of the nervous system. How these transient changes ultimately manifest themselves is dependent on the particular anatomical situation of the influencing and influenced centres. The presence of such transient changes can be enough to establish that two areas are connected, in a way similar to a stimulation experiment, but the generalized transient change in central excitatory and inhibitory state can obscure the function of either area.

% Give a detailed example of a mechanism of reciprocal innervation explaining coordination
% Talk about the difficulty of measuring inhibition: you need to compare it to a baseline of excitation
% Talk about inhibition as an "active" force in the nervous system. A "pulse" of inhibition is as much an input as a pulse of "excitation".
% Talk about the synapse as the location for the mechanism of inhibition (include stuff on the discovery of the synapse)

% The second one, the final common path and the need of neural coordination at the output
% The third one, inhibition is as active a force in the nervous system as excitation
% The fourth one, response latencies and implications for motor control (and synapses)

% One of the major points of contention were the differences between nerve trunk conduction and reflex arc conduction. The speed of propagation of the action potential across a nerve fiber had been known since the experiments of Helmholtz \cite{Helmholtz1850,Schmidgen2002} so it was possible to estimate the expected latency for a short spinal reflex, assuming reasonable estimates of nerve length and mechanical latency. For example, in the flexion-reflex of the dog's hind limb this latency was estimated to be around \SI{27}{\milli\second} \cite[p.19]{Sherrington1906}. However, the observed latencies in the spinal animal were, under normal conditions, found to be at least double of this number. Furthermore, reflex arc conduction had many other characteristics that were distinctive from the well studied nerve trunk conduction. Besides longer latencies, reflex arc conduction exhibited irreversibility of direction of conduction; fatigability and refractory period; greater variability of threshold; temporal facilitation with successive stimuli; a weaker correspondence of end-effect with intensity and frequency of the stimulus; and a greater susceptibility to metabolic and pharmacological manipulations \cite[p.14]{Sherrington1906}.

% These differences made Sherrington realize the necessity for a change in the physical medium of conduction at some point along the arc. The existence of such gaps clashed with the dominant idea at the time which viewed the nervous system as a continuous conductive nerve net. However, with the development of Golgi staining, Ramon y Cajal had recently been able to finally introduce convincingly the idea that the nervous system was actually composed of discrete cellular units, \emph{neurons} \cite{RamonYCajal1894}. Sherrington became a supporter of the newly established neuron doctrine and posited that the required theoretical junction between two neurons, which he termed the ``\emph{synapse}'' \cite{Foster1897}, could explain the unique physiology of reflex arc conduction.

Today we know from several detailed descriptions of invertebrate CPGs that different spinal circuits can be delicately super-imposed on the same neuronal elements by using a number of strategies to regulate and coordinate function \cite{Orlovsky1999,Selverston2010}. These strategies span all levels of neural organization from networks to molecules, ranging from reciprocal inhibition network motifs gated by specific neuromodulators and cell membrane receptors to overlapping distributions of voltage and ligand gated ion channels with different kinetics. The existence of such finely tuned spinal networks brings significant constraints when thinking about cortical motor control. Ultimately, any descending signals from the brain to muscles are also sharing this final common path and thus should be expected to require some form of coordination with spinal circuits if behavioural output is to remain integrated.

\subsubsection*{A role in modulating the movements generated by lower motor centres}

A different perspective on motor cortex emerged from studying the neural control of locomotion, suggesting that the corticospinal tract plays a role in the \emph{adjustment} of ongoing movements that are generated by lower motor systems. In this view, rather than motor cortex assuming direct control over muscle movement, it instead modulates the activity and sensory feedback in spinal circuits in order to adapt a lower movement controller to challenging conditions. This idea that the descending cortical pathways superimpose speed and precision on an existing baseline of behaviour was also suggested by lesion work in primates \cite{Lawrence1968a}, but has been investigated most thoroughly in the context of cat locomotion.

It has been known for more than a century that completely decerebrate cats are capable of sustaining the locomotor rhythms necessary for walking on a flat treadmill utilizing only spinal circuits \cite{GrahamBrown1911}. Brainstem and midbrain circuits are sufficient to initiate the activity of these spinal central pattern generators \cite{Grillner1973}, so what exactly is the contribution of motor cortex to the control of locomotion? Single-unit recordings of pyramidal tract neurons (PTNs) from cats walking on a treadmill have shown that a large proportion of these neurons are locked to the step cycle \cite{Armstrong1984a}. However, we know from the decerebrate studies that this activity is not necessary for the basic locomotor pattern. What then is its role?

Lesions of the lateral descending pathways (containing corticospinal and rubrospinal projections) produce a long term impairment in the ability of cats to step over obstacles \cite{Drew2002}. Recordings of PTN neurons during locomotion show increased activity during these visually guided modifications to the basic step cycle \cite{Drew1996}. These observations suggest that motor cortex neurons are necessary for precise stepping and adjustment of ongoing locomotion to changing conditions. However, long-term effects seem to require complete lesion of \emph{both} the corticospinal and rubrospinal tracts \cite{Drew2002}. Even in these animals, the voluntary act of stepping over an obstacle does not disappear entirely, and moreover, they can adapt to changes in the height of the obstacles \cite{Drew2002}. Specifically, even though these animals never regain the ability to gracefully clear an obstacle, when faced with a higher obstacle, they are able to adjust their stepping height in such a way that would have allowed them to comfortably clear the lower obstacle \cite{Drew2002}. Furthermore, deficits caused by lesions restricted to the pyramidal tract seem to disappear over time \cite{Liddell1944}, and are most clearly visible only the first time an animal encounters a new obstacle \cite{Liddell1944}.

The view that motor cortex in non-primate mammals is principally responsible for adjusting ongoing movement patterns generated by lower brain structures is appealing. What is this modulation good for? What does it allow an animal to achieve? How can we assay its necessity?
