\section{Discovery of the Motor Cortex}

The involvement of the brain and spinal cord in motor control has been recognized since the earliest known clinical records on head and spinal injury \cite{Breasted1930,VanMiddendorp2010}. However, the role of the nervous system in generating behaviour was not fully appreciated until Galvani first reported his famous experiments on \textit{animal electricity} \cite{Galvani1791}. By isolating the sciatic nerve and gastrocnemius muscle in the frog, Galvani clearly demonstrated in a series of stimulation experiments that an electrical process, contained entirely within the biology of the frog's leg, was responsible for the spontaneous generation of muscle contractions. This would lead over the following century to the discovery and physiological characterization of the nerve impulse, the action potential, that travelled across the nerve to initiate muscle movement \cite{DuBois-Reymond1843,Bernstein1868}. These experiments only raised further questions of nerve conduction: if spontaneous muscle contraction is generated by nerve impulses transmitted throughout the nervous system, how is this transmission coordinated to generate the patterns of natural movement we see in the behaving animal?

In search of answers to this question, many researchers looked at the brain, the seat of anatomical convergence of the nervous system, for such an integrative role. Following Galvani's footsteps, several attempts were made to stimulate the cerebral cortex electrically, but with little success \cite{Gross2007}. It wasn't until the 1870s that the first indications of a direct involvement of the cortex in the production of movement came to light, around the time when Hughlings Jackson underwent his studies on epileptic convulsions \cite{Jackson1870}. He realized that in some patients the fits would start by a deliberate spasm on one side of the body, and that different body parts would become systematically affected one after the other. He connected the orderly march of these spasms to the existence of localized lesions in the brain of his patients and hypothesized that the origin of these fits was uncontrolled excitation caused by local changes in cortical \emph{grey matter} \cite{Jackson1870}. In that same year, Fritsch and Hitzig published their famous study demonstrating that it is possible to elicit movements by direct stimulation of the cortex in dogs \cite{Fritsch1870}. Furthermore, stimulation of different parts of the cortex produced movement in different parts of the body \cite{Fritsch1870,Ferrier1873}. It appeared that the causal mechanism for epileptic convulsions predicted by Hughlings Jackson had been found, and with it a possible explanation for how the normal brain might control movements. Specifically, it was thought that cortex could exert direct control over the whole musculature of the body. The greater the excitation in the cortical area dedicated to a specific body part, the greater the contraction of the muscles in the corresponding region.

There were many confusing elements in the story, however. While stimulation studies were showing how activity in the so called motor cortex was sufficient to produce movements, other researchers like Goltz were reporting how massive cortical lesions failed to demonstrate any visible long-term impairments in the motor behaviour of animals \cite{Goltz1888}. Moreover, reproduction of the stimulation experiments across many mammalian species \cite{Ferrier1873,Clark1937} revealed over time a number of subtle observations. First, rather than merely activating individual muscles, prolonged stimulation at individual cortical sites reliably evoked behaviourally relevant, purposeful actions \cite{Ferrier1873,Clark1937}. Second, if the stimulation electrodes were held in the same fixed cortical point, repeated pulse trains could be made to elicit different responses if the configuration of the limbs or head was changed \cite{Ward1938}. This variability in responses to cortical stimulation was much higher than in spinal or decerebrate preparations and was thus termed the ``instability of the cortical point'' \cite{Brown1912,Leyton1917}.

How are the impulses transmitted across the nervous system? A dominant idea at the time was that the nervous system was a continuous conductive nerve net, with no gaps between the different nerve segments, but the rising prominence of Schwann's cell theory led others to try and apply it to the nervous system as well. With the development of Golgi staining, Ramon y Cajal was able to finally introduce convincingly the idea that the nervous system is composed of discrete cellular units, neurons \cite{RamonYCajal1894}. Still, the question remained: if spontaneous muscle contraction is generated by nerve impulses transmitted throughout the nervous system, how is this transmission coordinated to generate the patterns of natural movement we see in the behaving animal?

was not a continuous conductive nerve net. Sherrington built upon the neuron doctrine to interpret spinal physiology and understood many of its properties as revealing a clear gap between nerve cells, the synapse. He went on to characterize the responses of the spinal network using in vivo preparations where he introduced the concept of the reflex arc as a basic unit of integration in the spinal cord, and the spinal motor neuron as the final common pathway from the nervous system to the muscles \cite{Sherrington1906}.

Suddenly, questions of teleology became much easier to address: activity in the motor neuron \emph{causes} contraction of the muscle fibers to which it is connected. Sensory neurons \emph{signal} stretch of the muscle spindles back to the spinal cord. The idea of the reflex arc transforming afferent sensory inputs to efferent motor commands was found to explain both discrete and rhythmic patterns of behaviour across a variety of species \cite{Sherrington1892}. Since then, many more details of these peripheral circuits have been understood by a combination of anatomical, stimulation, recording and ablation experiments. Despite all the intrinsic complexity that has been found underlying these circuits, the basic teleology of spinal motor control remained fundamental throughout to provide a solid foundation for experimentation and mechanistic explanation of neural function.

In stark contrast to the spinal cord, a teleology for the cerebral cortex is still deeply lacking. The delineation of a clear function or principle that would make all manner of hypotheses and observations fall in place, resonating towards understanding of an insight, is sorely missed. Instead, we see a multitude of ideas clashing with each other, achieving mostly dissipation of the energy of their supporters as they strive to keep afloat in the shifting sands of the latest fashion. The function of cortex has remained surprisingly elusive after more than one and a half centuries of research. Many of the same questions entertained by early researchers on the nature of the central influence of the cerebral cortex in behaviour remain unanswered to this day, repeated over and over again with little fundamental modification despite significant advances in genetic, molecular and recording technology.
