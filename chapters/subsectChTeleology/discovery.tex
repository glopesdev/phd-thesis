\section{Discovery of the Motor Cortex}

The involvement of the brain and spinal cord in motor control has been recognized since the earliest known clinical records on head and spinal injury \cite{Louis1994,VanMiddendorp2010}. However, the role of the nervous system in generating behaviour was not fully appreciated until Galvani first reported his famous experiments on \textit{animal electricity} \cite{Galvani1791}. By isolating the sciatic nerve and gastrocnemius muscle in the frog, Galvani clearly demonstrated in a series of stimulation experiments that an electrical process, contained entirely within the biology of the frog's leg, was responsible for the spontaneous generation of muscle contractions. This would lead over the following century to the discovery and physiological characterization of the nerve impulse, the action potential, that travelled across the nerve to initiate muscle movement \cite{DuBois-Reymond1843,Bernstein1868,Schuetze1983}. The success of these experiments immediately raised further questions of nerve conduction: if spontaneous muscle contraction is generated by nerve impulses transmitted throughout the nervous system, how is this transmission coordinated to generate the patterns of natural movement we see in the behaving animal?

In search of answers to this question, many researchers looked at the brain, the seat of anatomical convergence of the nervous system, for such an integrative role. Following Galvani's footsteps, several attempts were made to stimulate the cerebral cortex electrically, but with little success \cite{Gross2007}. It wasn't until the 1870s that the first indications of a direct involvement of the cortex in the production of movement came to light, around the time when Hughlings Jackson underwent his studies on epileptic convulsions \cite{Jackson1870}. He realized that in some patients the fits would start by a deliberate spasm on one side of the body, and that different body parts would become systematically affected one after the other. He connected the orderly march of these spasms to the existence of localized lesions in the brain of his patients and hypothesized that the origin of these fits was uncontrolled excitation caused by local changes in cortical \emph{grey matter} \cite{Jackson1870}. In that same year, Fritsch and Hitzig published their famous study demonstrating that it is possible to elicit movements by direct stimulation of the cortex in dogs \cite{Fritsch1870}. Furthermore, stimulation of different parts of the cortex produced movement in different parts of the body \cite{Fritsch1870}. It appeared that the causal mechanism for epileptic convulsions predicted by Hughlings Jackson had been found, and with it a possible explanation for how the normal brain might control movement. Specifically, it was thought that cortex could exert direct control over the whole musculature of the body. The greater the excitation in the cortical area dedicated to a specific body part, the greater the contraction of the muscles in the corresponding region.

\section{The Goltz-Ferrier Debates}

David Ferrier, a Scottish neurologist deeply impressed by Fritsch and Hitzig's experiments, proceeded to reproduce and expand on their results across a wide range of mammalian species \cite{Ferrier1873}. While these stimulation studies were showing how activity in the so called \emph{motor cortex} was sufficient to produce a large variety of movements, other researchers like Goltz were reporting how massive cortical lesions failed to demonstrate any visible long-term impairments in the motor behaviour of animals \cite{Goltz1888}. At the seventh International Medical Congress held in London in August 1881, both Goltz and Ferrier presented their results in a series of debates on the localization of function in the cerebral cortex \cite{Tyler2000}.

Goltz assumed a clear anti-localizationist position. He advanced that it was impossible to produce a complete paresis of any muscle, or complete dysfunction of any perception, by destruction of any part of the cerebral cortex, and that he found mostly deficits of general intelligence in his dogs \cite{Tyler2000}. Following Goltz's presentation, Ferrier emphasized the danger of generalizing from the dog to animals of other orders (e.g. man and monkey). He then proceeded to exhibit his own lesion results by means of antiseptic surgery in the monkey, describing how a circumscribed unilateral lesion of the motor cortex produced complete contralateral paralysis of the leg. He also produced a striking series of microscopic sections of Wallerian degeneration \cite{Waller1850} of the ``motor path'' from the cortex to the contralateral spinal cord, the crossed descending projections forming the pyramidal corticospinal tract \cite{Tyler2000}.

The debates concluded with the public demonstration of live specimens: a dog with large lesions to the parietal and posterior lobes from Goltz; and a hemiplegic monkey with a unilateral lesion to the motor cortex of the contralateral side from Ferrier. As predicted, Goltz's dog showed a clear ability to locomote and avoid obstacles and to make use of its other basic senses, while displaying peculiar deficits of intelligence such as failing to respond with fear to the cracking of a whip or ignoring tobacco smoke blown to its face. On the other hand, Ferrier's monkey showed up severely hemiplegic, in a condition similar to human stroke patients. After the demonstrations, the animals were killed and their brains removed. Preliminary observations revealed that the lesions in Goltz's dog were less extensive than expected, particularly on the left hemisphere. Ferrier's lesions on the other hand were precisely circumscribed to the contralateral motor cortex. The triumph of Ferrier firmly established the localizationist approach to neurology and secured the idea of a somatotopic arrangement in the motor cortex.

The Goltz-Ferrier debates had far-reaching implications throughout the entire research community of the time. Charles Sherrington himself started out his own work by tracing spinal cord degeneration over large periods of time (up to 11 months) following cortical lesions in Goltz's dogs \cite{Langley1884,Sherrington1885}. From these examinations, he observed for the first time in the dog the presence of a degenerated ``re-crossed'' pyramidal tract that travels down the cord ipsilateral to the side of the lesion \cite{Sherrington1885}. These fibers would later come to be called the ipsilateral, ventral corticospinal tract, and have since been found and described in most mammalian species \cite{Kuypers1981,Brosamle2000,Lacroix2004}.

He also had the chance during this time to observe first hand the negative effects of cortical lesions reported by Goltz in a variety of specimens. In his own words:

\blockquote[{\protect\cite[p.189]{Sherrington1885}}]{That the pyramidal tracts are in the dog requisite for volitional~impulses to reach limbs and body seems negatived by the fact that the animal can run, leap, turn to either side, use neck and jaws, \&c. with ease and success after nearly, if not wholly, complete degeneration of these tracts on both sides. Further, after complete degeneration of one pyramid, there is in the dog no obvious difference between the movements of the right and left sides.}

Interestingly, he does note that \enquote{defect of motion is observable only as a clumsiness in execution of fine movements} \cite{Sherrington1885}, hinting at ideas that are today still part of broadly accepted theories for the role of this pathway.

\section{The Reflex Arc and the Principle of the Final Common Path}

Sherrington would soon turn to attack the problem of nerve conduction from the opposite direction. After his return to Cambridge, he began to systematically characterize anatomically and physiologically the distribution of efferent \cite{Sherrington1892} and afferent \cite{Sherrington1893a} nerve roots in the spinal cord of multiple species. His goal was to shed light on the so-called \emph{reflex arc}, the nerve pathways involved in muscular reactions like the knee-jerk whereby simple sensory stimuli elicit an immediate, automatic response from the animal, even in the absence of higher brain input \cite{Sherrington1893b}.

Sherrington and his contemporaries studied in detail a number of long and short spinal reflexes\footnote{A reflex action in which a stimulus applied to one region elicits a response in another region is termed a \emph{long spinal} reflex, whereas a reflex reaction where the muscular response happens in the same region as the stimulus is termed a \emph{short spinal} reflex.} in a variety of model organisms under different levels of anesthesia, pharmacological manipulations and spinal transsection \cite{Sherrington1903}. This systematic approach soon revealed a number of difficulties with the conceptual arrangement of the reflex arc in terms of the simple nerve conduction ideas of the time.

One of the major points of contention were the differences between nerve trunk conduction and reflex arc conduction. The speed of propagation of the action potential across a nerve fiber had been known since the experiments of Helmholtz \cite{Helmholtz1850,Schmidgen2002} so it was possible to estimate the expected latency for a short spinal reflex, assuming reasonable estimates of nerve length and mechanical latency. For example, in the flexion-reflex of the dog's hind limb this expected latency was around \SI{27}{\milli\second} \cite[p.19]{Sherrington1906}. However, the observed latencies in the spinal animal were, under normal conditions, found to be at least double of this number. Furthermore, reflex arc conduction had many other characteristics that were distinctive from the well studied nerve trunk conduction. Apart from the longer latencies, reflex arc conduction exhibited irreversibility of direction of conduction; fatigability and refractory period; greater variability of threshold; temporal facilitation with successive stimuli; a weaker correspondence of end-effect with intensity and frequency of the stimulus; and a greater susceptibility to metabolic and pharmacological manipulations \cite[p.14]{Sherrington1906}.

It were these differences that made Sherrington realize the necessity for a change in the medium of conduction at some point along the arc. A dominant idea at the time was that the nervous system was a continuous conductive nerve net, but with the development of Golgi staining, Ramon y Cajal was able to finally introduce convincingly the idea that the nervous system is composed of discrete cellular units, \emph{neurons} \cite{RamonYCajal1894}. Sherrington soon became a supporter of the newly established neuron doctrine and posited that the required theoretical junction between two neurons, which he termed the ``\emph{synapse}'' \cite{Foster1897}, could explain the unique physiology of reflex arc conduction.

Apart from general properties of conduction, Sherrington was also quick to emphasize that the most striking feature of the nervous system, even at the level of the spinal cord, was its ability to integrate and coordinate the activity of multiple reflex arcs. Specifically, he showed how reflex responses initiated by distinct sensory receptors would often impinge on the same motor neurons. These shared neurons represented thus a \emph{final common path} \cite{Sherrington1904} throughout the nervous system. This of course posed a problem if multiple such reflex responses were to be initiated simultaneously. Sherrington went on to demonstrate that when two or more reflexes were elicited at the same time, the observed response was not a simple summation or linear combination of both responses. Rather, as he describes:

\blockquote[{\protect\cite[p.461]{Sherrington1904}}]{When two stimuli are applied simultaneously which would evoke reflex actions that employ the same final common path in different ways, in my experience one reflex appears without the other. The result is this reflex or that reflex, but not the two together.}

% Latency
% Unidirectional conduction
% Subthreshold summation
% Fatigue / Refractory phase
% Frequency of response independent from frequency of stimulation
% Nerve-trunk vs reflex arc conduction

% Coordination of simultaneous reflexes (allied and antagonistic)
% The role of Inhibition in shaping responses (agonist and antagonist muscles)
% Reciprocal inhibition
% Adaptive responses
% Propriospinal tracts and "successive degeneration"
% Decerebrate rigidity
% Facilitation of spinal reflexes after decerebration

Moreover, reproduction of the stimulation experiments across many mammalian species \cite{Ferrier1873,Clark1937} revealed over time a number of subtle observations. First, rather than merely activating individual muscles, prolonged stimulation at individual cortical sites reliably evoked behaviourally relevant, purposeful actions \cite{Ferrier1873,Clark1937}. Second, if the stimulation electrodes were held in the same fixed cortical point, repeated pulse trains could be made to elicit different responses if the configuration of the limbs or head was changed \cite{Ward1938}. This variability in responses to cortical stimulation was much higher than in spinal or decerebrate preparations and was thus termed the ``instability of the cortical point'' \cite{Brown1912,Leyton1917}.

Sherrington built upon the neuron doctrine to interpret spinal physiology and understood many of its properties as revealing a clear gap between nerve cells, the synapse. He went on to characterize the responses of the spinal network using in vivo preparations where he introduced the concept of the reflex arc as a basic unit of integration in the spinal cord, and the spinal motor neuron as the final common pathway from the nervous system to the muscles \cite{Sherrington1906}.

Suddenly, questions of teleology became much easier to address: activity in the motor neuron \emph{causes} contraction of the muscle fibers to which it is connected. Sensory neurons \emph{signal} stretch of the muscle spindles back to the spinal cord. The idea of the reflex arc transforming afferent sensory inputs to efferent motor commands was found to explain both discrete and rhythmic patterns of behaviour across a variety of species \cite{Sherrington1893b}. Since then, many more details of these peripheral circuits have been understood by a combination of anatomical, stimulation, recording and ablation experiments. Despite all the intrinsic complexity that has been found underlying these circuits, the basic teleology of spinal motor control remained fundamental throughout to provide a solid foundation for experimentation and mechanistic explanation of neural function.

In stark contrast to the spinal cord, a teleology for the cerebral cortex is still deeply lacking. The delineation of a clear function or principle that would make all manner of hypotheses and observations fall in place, resonating towards understanding of an insight, is sorely missed. Instead, we see a multitude of ideas clashing with each other, achieving mostly dissipation of the energy of their supporters as they strive to keep afloat in the shifting sands of the latest fashion. The function of cortex has remained surprisingly elusive after more than one and a half centuries of research. Many of the same questions entertained by early researchers on the nature of the central influence of the cerebral cortex in behaviour remain unanswered to this day, repeated over and over again with little fundamental modification despite significant advances in genetic, molecular and recording technology.
