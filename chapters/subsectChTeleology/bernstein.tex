\section{A Strategy for Probing Cortical Control}

It should now be clear that the involvement of motor cortex in the direct control of all ``voluntary movement'' is human-specific. There is a role for motor cortex across mammals in the control of precise movements of the extremities, especially those requiring individual movements of the fingers, but these effects are subtle in non-primate mammals. Furthermore, what would be a devastating impairment for humans may not be so severe for mammals that do not depend on precision finger movements for survival. Therefore, generalizing this specific role of motor cortex from humans to all other mammals would be misleading. We could be missing another, more primordial role for this structure that predominates in other mammals, and by doing so, we may also be missing an important role in humans.

The proposal that motor cortex induces modifications of ongoing movement synergies, prompted by the electrophysiological studies of cat locomotion, definitely points to a role consistent with the results of various lesion studies. However, in assays used, the ability to modify ongoing movement generally recovers after a motor cortical lesion. What are the environmental situations in which motor cortical modulation is most useful?

Cortex has long been proposed to be the structure responsible for integrating a representation of the world and improving the predictive power of this representation with experience \cite{Barlow1985,Doya1999}. If motor cortex is the means by which these representations can gain influence over the body, however subtle and ``modulatory'', can we find situations (i.e. tasks) in which this cortical control is required?

The necessity of cortex for various behavioural tasks has been actively investigated in experimental psychology for over a century, including the foundational work of Karl Lashley and his students \cite{Lashley1921a,Lashley1950a}. In the rat, large cortical lesions were found to produce little to no impairment in movement control, and even deficits in learning and decision making abilities were difficult to demonstrate consistently over repeated trials. However, Lashley did notice some evidence that cortical control may be involved in postural adaptations to unexpected perturbations \cite{Lashley1921a}. These studies once again seem to recapitulate the two most consistent observations found across the entire motor cortical lesion literature in non-primate mammals since Hitzig \cite{Fritsch1870}, Goltz \cite{Goltz1888}, Sherrington \cite{Sherrington1885} and others \cite{Oakley1979,Terry1989}. One, direct voluntary control over movement is most definitely not abolished through lesion; and two, certain aspects of some movements are definitely impaired, but only under certain challenging situations. The latter are often reported only anecdotally. It was this collection of intriguing observations in animals with motor cortical lesions that prompted us to expand the scope of standard laboratory tasks to include a broader range of motor control challenges that brains encounter in their natural environments.

\subsection{Thesis Outline}

In this work, an attempt to outline a new role for motor cortex is reported. As many previous efforts, it starts with behaviour, and the realization that controlled exposure of animals to a wider range of environments is of absolute necessity to gain insight into the teleology of the system. To this end, we have developed new tools to make it easier to survey a large range of environments while recording as many fine scale measures of behaviour and physiology as possible. These technological developments and methods are described in Chapter \ref{ch:tools}.

In Chapter \ref{ch:behaviour}, a set of behaviour and lesion studies is reported in the rat. These studies had the goal of probing the limits of recovery following extensive cortical lesions by exposing animals to more challenging and dynamic environments. Detailed analysis of the moment by moment behaviour of lesioned animals revealed a number of intriguing observations, the implications of which we discuss in Chapter \ref{ch:conclusions}.

