\section{A Teleology for Motor Cortex - Bernstein}

\subsection{The Search for a Teleology of Motor Cortex}

Despite his success in laying the foundations for the coordination of reflexes in the spinal cord, Sherrington continued to show interest in following up his original work on cortical localization and behaviour. In a series of seminal publications together with Gr\"unbaum \cite{Grunbaum1903,Leyton1917}, Sherrington presented a number of experiments attempting to clarify the role of excitable cortex.

In these experiments covering different primate species, Sherrington combined surface stimulation mapping with studies of behaviour after lesions to excitable areas in the same animal. Through these systematic studies he hoped to better understand two previously conflicting observations of motor cortical physiology: the instability and complexity of responses evoked by stimulating different cortical points \cite{GrahamBrown1912}, and the dramatic recovery of motor function which was often observed following circumscribed cortical lesions \cite{GrahamBrown1913}.

To do this, Sherrington and his co-workers would carefully map the forearm region of motor cortex in one hemisphere of individual animals, and then proceed to precisely excise the entire area, using the stimulation maps as a guide. In one of their reported cases, after the animal recovered from surgery, his use of the contralateral forearm was visibly impaired, showing pronounced weakness and inability to grasp with the right hand. However, one month following the surgery, the animal had recovered most of the functions in the affected arm, apart from deficits in individuated finger movements and slight weakening of thumb and index. In an attempt to trace out the parts of the brain responsible for the recovery, they re-exposed the lesioned hemisphere and again mapped out movement responses to stimulation. They observed that they were unable to evoke responses in the affected arm by stimulating the site of the lesion, as well as any adjacent cortex in front or behind \cite{Leyton1917}.

\subsubsection*{Towards a new teleology; new experiments required}

It should now be clear that the involvement of motor cortex in the direct control of all ``voluntary movement'' is human-specific. There is a role for motor cortex across mammals in the control of precise movements of the extremities, especially those requiring individual movements of the fingers, but these effects are subtle in non-primate mammals. Furthermore, what would be a devastating impairment for humans may not be so severe for mammals that do not depend on precision finger movements for survival. Therefore, generalizing this specific role of motor cortex from humans to all other mammals would be misleading. We could be missing another, more primordial role for this structure that predominates in other mammals, and by doing so, we may also be missing an important role in humans.

The proposal that motor cortex induces modifications of ongoing movement synergies, prompted by the electrophysiological studies of cat locomotion, definitely points to a role consistent with the results of various lesion studies. However, in assays used, the ability to modify ongoing movement generally recovers after a motor cortical lesion. What are the environmental siutations in which motor cortical modulation is most useful?

Cortex has long been proposed to be the structure responsible for integrating a representation of the world and improving the predictive power of this representation with experience \cite{Barlow1985,Doya1999}. If motor cortex is the means by which these representations can gain influence over the body, however subtle and ``modulatory'', can we find situations (i.e. tasks) in which this cortical control is required?

The necessity of cortex for various behavioural tasks has been actively investigated in experimental psychology for over a century, including the foundational work of Karl Lashley and his students \cite{Lashley1921a,Lashley1950a}. In the rat, large cortical lesions were found to produce little to no impairment in movement control, and even deficits in learning and decision making abilities were difficult to demonstrate consistently over repeated trials. However, Lashley did notice some evidence that cortical control may be involved in postural adaptations to unexpected perturbations \cite{Lashley1921a}. These studies once again seem to recapitulate the two most consistent observations found across the entire motor cortical lesion literature in non-primate mammals since Hitzig \cite{Fritsch1870}, Goltz \cite{Goltz1888}, Sherrington \cite{Sherrington1885} and others \cite{Oakley1979,Terry1989}. One, direct voluntary control over movement is most definitely not abolished through lesion; and two, certain aspects of some movements are definitely impaired, but only under certain challenging situations. The latter are often reported only anecdotally. It was this collection of intriguing observations in animals with motor cortical lesions that prompted us to expand the scope of standard laboratory tasks to include a broader range of motor control challenges that brains encounter in their natural environments.

In the following, we report an experiment that was designed to provide controlled exposure of animals to more naturally challenging environments. The results of this experiment have led us to formulate a new teleology for cortical motor control that we will present in the discussion.

