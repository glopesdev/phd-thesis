\section{Lesions of the Motor Cortex}

From the very beginning of brain research, the study of natural and artificial injury to the nervous system of animals has remained a critical means to infer neural function. The early studies of Broca localizing the function of articulate language to a specific region in the cerebral hemispheres \cite{Broca1861} established a long tradition of correlating the location of brain injury with a behavioural disorder. In this section we review the main findings derived from behavioural observation of animal and human subjects with different kinds of lesions to the motor cortex, and highlight the strengths and weaknesses of the various methods.

\subsection{Permanent Lesions}

The study of the effects of permanent lesions in motor behaviour started with the observation of contralateral hemiparesis or hemiplegia in human patients with naturally occurring injuries to the cerebral hemispheres \cite{Louis1994}. In humans, a permanent injury to the frontal lobes of the brain by stroke or mechanical means is often followed by weakening or paralysis of the limbs in the side of the body opposite to the lesion. This condition has a tendency to recover partially by itself, especially with training and rehabilitation, but permanent movement deficits in the affected limbs is the common prognosis. Following the discovery of the motor cortex and the Goltz-Ferrier debates, a vast number of studies have systematically explored the effects of different kinds of cortical injury in motor behaviour, not only on naturally occurring human lesions, but using artificially induced lesions in non-human mammals as a way to control for the type, location, and extent of the permanent lesion.

\subsubsection{Techniques}

The original methods used to induce a permanent lesion to the motor cortex were very crude, often involving gross mechanical aggression to the neural tissue by using surgical knife cuts or ablation by water-jet, aspiration, and thermo- or electrocoagulation. These methods are still widely used in lesion studies for their simplicity and bluntness, but have the disadvantage of making it hard to limit the lesion to a single area because of possible damage to subcortical areas or the destruction of fibers of passage. Fibers of passage are nerve fibers passing through the lesioned area which neither originate nor terminate in the region of interest. These limitations made it more difficult to interpret the effects of cortical lesions, and eventually led to the development of new techniques designed to work around such problems. Chemical injections of neurotoxic compounds such as ibotenic acid or kainic acid aim to increase selectivity of the lesion by limiting damage to neural cell bodies in the target area while leaving the fibers of passage intact. Photothrombosis or devascularization by pial stripping aim to reproduce the effects of clinical stroke and avoid extension of the lesion to subcortical areas.

\subsubsection{Type and Extent of Permanent Lesions}

\subsubsection{General Effects on Behaviour}

Widely diverging reports of the severity of effects following motor cortical lesions have been a staple in the field since the Goltz-Ferrier debates. Not only across species, but across individuals and specific lesion techniques, the general effects on behaviour following cortical injury have been a constant point of contention that remains unresolved to this day. Still, the sheer number of experiments available allow us to recognize a number of consistent observations.

\subsubsection{Effects on Fine Movements}

The most commonly reported long-term effects of motor cortical lesions are weakening (paresis) of the limbs and loss of the ability to execute precise fine movements requiring independent control of the hand and digits \cite{Ferrier1884}. Specifically in humans, actions requiring individual finger movements like the precision grip often fail to recover their baseline performance levels even in the face of extensive rehabilitation \cite{Xu2015}. In non-human primates, loss of prehensile strength has for a long time been reported as a lasting effect in behavioural assays such as the dexterity board, where monkeys are tasked to pick morsels of food from differently sized round holes placed in a grid \cite{Cole1952}. While normal monkeys routinely pick up the food by pinching individual bits with their fingers, monkeys with lesions to the hand and arm area of the primary motor cortex are unable to perform this movement, and instead employ coarser compensatory clasping strategies to retrieve the food \cite{Lawrence1968}.

This deficit in the execution of specific fine movements has been replicated in rats using the skilled reaching task, where animals have to retrieve food pellets placed on top of a pedestal by reaching their paw through a narrow slit. Normal rats will lift and advance the limb through the slit and pronate the paw to grasp the food pellet. After a successful grasp, they will withdraw the limb and supinate the paw for placing the food in the mouth. In rats with unilateral lesions to the forelimb area of the motor cortex this rotatory action of the paw is impaired. If unrestrained, animals will avoid using the limb contralateral to the lesion, even if it was their preferred limb before the intervention. If forced to use the affected limb, they adopt compensatory postures that allow them to retrieve the pellet without having to pronate or supinate the paw, although with a larger number of misses.

Complementary studies have since traced this deficit in both primates and rodents to the direct corticospinal tract that provides a monosynaptic connection from the motor cortex to the spinal cord. Sectioning of the tract is sufficient to reproduce these impairments in both species.

\subsubsection{Effects on Locomotion}

\subsection{Transient Inactivations}

\subsubsection{Techniques}

\subsection{Difficulties of Lesion Studies}

\subsubsection{Diaschisis}

\subsubsection{Recovery of Function}
