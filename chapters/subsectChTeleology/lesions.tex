\section{Lesions of the Motor Cortex}

From the very beginning of brain research, the study of natural and artificial injury to the nervous system of animals has remained a critical means to infer neural function. The early studies of Broca localizing the function of articulate language to a specific region in the cerebral hemispheres \cite{Broca1861} established a long tradition of correlating the location of brain injury with a behavioural disorder. In this section we review the main findings derived from behavioural observation of animal and human subjects with different kinds of lesions to the motor cortex, and highlight the strengths and weaknesses of the various methods.

\subsection{Permanent Lesions}

The study of the effects of permanent lesions in motor behaviour started with the observation of contralateral hemiparesis or hemiplegia in human patients with naturally occurring injuries to the cerebral hemispheres \cite{Louis1994}. In humans, a permanent injury to the frontal lobes of the brain by stroke or mechanical means is often followed by weakening or paralysis of the limbs in the side of the body opposite to the lesion. This condition has a tendency to recover partially by itself, especially with training and rehabilitation, but permanent movement deficits in the affected limbs is the common prognosis. Following the discovery of the motor cortex and the Goltz-Ferrier debates, a vast number of studies have systematically explored the effects of different kinds of cortical injury in motor behaviour, not only on naturally occurring human lesions, but using artificially induced lesions in non-human mammals as a way to control for the type, location, and extent of the permanent lesion.

\subsubsection{Techniques}

The original methods used to induce a permanent lesion to the motor cortex were very crude, often involving gross mechanical aggression to the neural tissue by using surgical knife cuts or ablation by water-jet, aspiration, and thermo- or electrocoagulation. These methods are still widely used in lesion studies for their simplicity and bluntness, but have the disadvantage of making it hard to limit the lesion to a single area because of possible damage to subcortical areas or the destruction of fibers of passage. Fibers of passage are nerve fibers passing through the lesioned area which neither originate nor terminate in the region of interest. These limitations made it more difficult to interpret the effects of cortical lesions, and eventually led to the development of new techniques designed to work around such problems. Chemical injections of neurotoxic compounds such as ibotenic acid or kainic acid aim to increase selectivity of the lesion by limiting damage to neural cell bodies in the target area while leaving the fibers of passage intact. Photothrombosis \cite{Watson1985} or devascularization by pial stripping aim to reproduce the effects of clinical stroke and avoid extension of the lesion to subcortical areas.

\subsubsection{Type and Extent of Permanent Lesions}

\subsubsection{General Effects on Behaviour}

Widely diverging reports of the severity of effects following motor cortical lesions have been a staple in the field since the Goltz-Ferrier debates. Not only across species, but across individuals and specific lesion techniques, the general effects on behaviour following cortical injury have been a constant point of contention that remains unresolved to this day. Still, the sheer number of experiments available allow us to recognize a number of consistent observations.

\subsubsection{Effects on Fine Movements}

In human and non-human primates, the most commonly reported effects of injury to motor cortical areas are weakening or paralysis of the limbs on the side of the body opposite to the lesion, followed by loss of the ability to execute fine movements requiring independent control of the hand and digits \cite{Ferrier1884,Lashley1924,Glees1950,Darling2011,Xu2015}. Although permanent movement deficits in the use of the affected limbs is the most frequent prognosis in humans \cite{Kwakkel2003}, it is also clear that these conditions have remarkable capacity for spontaneous recovery \cite{GrahamBrown1913,Leyton1917}, even in the absence of rehabilitation, in non-human primates. It has been proposed that spared ipsilesional or contralesional motor cortical areas can take over the functions of the damaged area, but there are conflicting reports regarding the degree to which cortical reorganization can provide an adequate account of recovery. Many of the observed deficits can vary not only with the extent of the lesion, but with the specific demands of the tasks used to assay motor function.

Anatomically, these motor deficits, as well as the evoked responses to electrical stimulation, are thought to depend on the direct descending projections from the cortex to the spinal cord, also called the corticospinal or pyramidal tract. Several studies have used unilateral and bilateral section of the corticospinal tract at the level of the \textit{medulla oblongata}\footnote{The area of the brainstem where corticospinal fibers cross over to the opposite side.} to isolate the effects of individual descending pathways in movement \cite{Tower1940,Lawrence1968,Lawrence1968a}. Direct corticospinal tract lesions are able to recapitulate in the non-human primate the paresis and loss of individuated finger movements following motor cortical lesion. On the other hand, lesions at the level 

From the early beginnings of motor cortex research, both movement impairments and evoked muscle responses were thought to rely on the direct descending projections from the cortex to the spinal cord, the corticospinal or pyramidal tract. Therefore, a more direct means to trace the role of this pathway in movement deficits was to section the pyramids at different levels and observe the effects. Sarah Tower was the first to describe in detail the results of unilateral and bilateral pyramidotomy in primates, with and without lesion of the motor cortex \cite{Tower1940}. She summarized the condition as ``hypotonic paresis'', characterized by a loss of skeletal muscle tone and depression of the vasomotor system, along with general weakening of the reflexes involving the affected limb segments. Although all discrete usage of the hand and digits was eliminated, she did emphasize the clear presence of voluntary movements in the various purposeful compensations produced by the animals to deal with the affliction. Tower attributed these compensations to the preserved capacities of brainstem circuits.

After the first results of cortical stimulation by Fritsch and Hitzig, controlled lesion studies in non-human primates have been trying for the past 100 years to pinpoint the function of this area and the degree to which it can recover \cite{Darling2011}.

By mapping out the parts of cortex where stimulation evokes movement of specific body parts in monkeys, Ferrier and others were able to direct their lesions to target only these areas and originally reported complete and persistent paralysis of the corresponding muscles \cite{Ferrier1884}. Soon after, however, Sherrington and others reported apparently contradictory observations of dramatic recovery following similar lesions to the arm area \cite{GrahamBrown1913,Leyton1917}. Furthermore, not only was near complete recovery reported, but the focus of such recovery could not be obviously traced to other brain areas either in the neighbourhood of the lesion or in the corresponding cortex of the opposite hemisphere. This was verified both by stimulation of the perilesional and contralesional cortex, which failed to evoke movements in the affected limb \cite{Leyton1917}, and by subsequent ablation of these areas. Although the lesion of the intact hemisphere did produce a temporary impairment in the use of the other limb, both of these new lesions failed to reinstate the original deficit \cite{GrahamBrown1913,Leyton1917}.

A more definite study to dissociate the effects of direct corticospinal and indirect brainstem descending pathways was conducted by Lawrence and Kuypers, and presented in their now classical publications \cite{Lawrence1968,Lawrence1968a}.

One of the limitations of early studies was the lack of quantitative assessment of motor function. Later studies introduced new behavioural assays such as the matchbox or dexterity board as a way to more formally test motor recovery \cite{Glees1950,Cole1952}. For example in the dexterity board, monkeys are tasked to pick morsels of food from differently sized round holes placed in a grid as a way to test for prehensile strength and fine digit control. Using such measures it was found that circumscribed lesions of the thumb and index areas produced permanent deficits of weakness and loss of fine movements despite an early recovery period \cite{Glees1950}. The stimulation experiments of Sherrington were also revisited and it was found that motor cortical areas adjacent to the lesion were now able to evoke movements of the impacted digits \cite{Glees1950}. Subsequent removal of these areas reinstated the original motor deficit \cite{Glees1950}.

This deficit in the execution of specific fine movements has been replicated in rats using the skilled reaching task, where animals have to retrieve food pellets placed on top of a pedestal by reaching their paw through a narrow slit. Normal rats will lift and advance the limb through the slit and pronate the paw to grasp the food pellet. After a successful grasp, they will withdraw the limb and supinate the paw for placing the food in the mouth. In rats with unilateral lesions to the forelimb area of the motor cortex this rotatory action of the paw is impaired. If unrestrained, animals will avoid using the limb contralateral to the lesion, even if it was their preferred limb before the intervention. If forced to use the affected limb, they adopt compensatory postures that allow them to retrieve the pellet without having to pronate or supinate the paw, although with a larger number of misses.

Complementary studies have since traced this deficit in both primates and rodents to the direct corticospinal tract that provides a monosynaptic connection from the motor cortex to the spinal cord. Sectioning of the tract is sufficient to reproduce these impairments in both species \cite{Lawrence1968}. While normal monkeys routinely pick up the food by pinching individual bits with their fingers, monkeys with lesions to the hand and arm area of the primary motor cortex are unable to perform this movement, and instead employ coarser compensatory clasping strategies to retrieve the food. Posture is affected \cite{Lashley1924}.

\subsubsection{Effects on Locomotion}

\subsection{Transient Inactivations}

\subsubsection{Techniques}

\subsection{Difficulties of Lesion Studies}

\subsubsection{Diaschisis}

\subsubsection{Recovery of Function}
