\section{Stimulation of the Motor Cortex}

Following the success of the early stimulation studies of Fritsch, Hitzig and Ferrier \cite{Fritsch1870,Ferrier1873}, the investigation of motor cortical function by artificial excitation of cortical tissue has become one of the hallmarks of the field. In fact, even today primary motor cortex is still defined by many authors as the area of cortex from which movements can be evoked with the least amount and duration of electrical current stimulation.

\subsection{Microstimulation Mapping}

One of the most distinguishing physiological characteristics of mammalian motor cortex is the ability to evoke movements in different parts of the skeletal musculature by electrical stimulation of different motor cortical areas. Following the discovery of this rough somatotopy, several researchers developed microstimulation mapping protocols, where low-amplitude, short duration pulses of electrical current are delivered systematically across many different predetermined cortical sites arranged in a grid lattice spanning the entire motor cortex. In this way it was possible to establish the precise somatotopical organization of motor cortex across many different species, including humans.

However, difficulties in interpreting the function of these stimulation fields in the normal behaving animal have been pointed out since the early stimulation experiments. First, it was noted that many of the cortical stimulation points were ``unstable''. Specifically, if the stimulating electrode is kept in the same place and the stimulation protocol repeated, there can be a large variability in evoked movements, much larger than what is normally encountered with stimulation of efferent nerve roots at the level of the spinal cord. Second, while microstimulation protocols are able to evoke single muscle twitches, the application of longer stimulation trains can trigger the release of complex, integrated movement sequences which are behaviourally meaningful. In the recent experiments by Graziano it was demonstrated that not only the evoked movements were all part of the animal's natural behaviour repertoire, but they appeared to be goal-directed and to take into account the current state of the musculature. For example, a reaching action evoked by stimulation of a single cortical point would bring the hand to grasp a specific region in space. If the stimulating electrode was kept in the same place and the arm of the animal moved to a different configuration, the stimulation would now cause different effects in the animal musculature, but which nevertheless had the result of bringing the hand to grasp the same point in space.

\subsection{Transcranial Magnetic Stimulation}

\subsection{Difficulties of Stimulation Studies}
