\section{Discussion}

In these experiments, we assessed the role of motor cortical structures by making targeted lesions to areas responsible for forelimb control \cite{Kawai2015,Otchy2015}. Consistent with previous studies, we did not observe any conspicuous deficits in movement execution for rats with bilateral motor cortex lesions when negotiating a stable environment. Even when exposed to a sequence of unstable obstacles, animals were able to learn an efficient strategy for crossing these more challenging environments, with or without motor cortex. These movement strategies also include a preparatory component that might reflect the state of the world an animal expected to encounter. Surprisingly, these preparatory responses also did not require the presence of motor cortex.

It was only when the environment did not conform to expectation, and demanded a rapid adjustment, that a difference between the lesion and control groups was obvious. Animals with extensive damage to the motor cortex did not deploy a change in strategy. Rather, they halted their progression for several seconds, unable to robustly respond to the new motor challenge. In an ecological setting, such hesitation could easily prove fatal. Control animals, on the other hand, were able to rapidly and flexibly reorganize their motor response to an entirely unexpected change in the environment.

Our preliminary investigations of the neurophysiological basis of these robust responses with ECoG have revealed the presence of large amplitude evoked potentials in the motor cortex arising specifically in response to an unexpected collapse of the steps during locomotion. Compared with evoked responses obtained from normal stepping under stable conditions (\SI{-100}{\micro\volt} peak at \SI{10}{\milli\second}), these potentials are both much larger (\SI{-300}{\micro\volt}) and delayed in time (peak at \SI{70}{\milli\second}). Still, they preceded any overt behaviour corrections from the animal following the perturbation, as observed in the high-speed video recordings. The onset of these evoked potentials is in the range of the long-latency stretch reflex, which has been suggested to involve a transcortical loop through the motor cortex \cite{Phillips1969,Matthews1990,Capaday1991}. However, the simultaneous complexity and rapidity of adaptive motor responses we observed in control animals is striking, as they appear to go beyond simple corrective responses to reach a predetermined goal and include a fast switch to entirely different investigatory or compensatory motor strategies adapted to the novel situation. What is the nature of these robust responses that animals without motor cortex seem unable to deploy? What do they allow an animal to achieve? Why are cortical structures necessary for their successful and rapid deployment?
