\section{Introduction}

In the natural world, an animal must be able to adapt locomotion to any surface, not only in anticipation of upcoming terrain, but also in response to the unexpected perturbations that often occur during movement. This allows animals to move robustly through the world, even when navigating a changing environment. Testing the ability of the motor system to generate a robust response to an unexpected change can be difficult as it requires introducing a perturbation without cueing the animal about the altered state of the world. Marple-Horvat and colleagues built a circular ladder assay for cats that was specifically designed to record from motor cortex during such conditions \cite{Marple-Horvat1993}. One of the modifications they introduced was to make one of the rungs of the ladder fall unexpectedly under the weight of the animal. When they recorded from motor cortical neurons during the rung drop, they noticed a marked increase in activity, well above the recorded baseline from normal stepping, as the animal recovered from the fall and resumed walking. However, whether this increased activity of motor cortex was necessary for the recovery response has never been assayed.
