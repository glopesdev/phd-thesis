\section{Introduction}

In the natural world, an animal must be able to adapt locomotion to any surface, not only in anticipation of upcoming terrain, but also in response to the unexpected perturbations that often occur during movement. This allows animals to move robustly through the world, even when navigating a changing environment. Testing the ability of the motor system to generate a robust response to an unexpected change can be difficult as it requires introducing a perturbation without cueing the animal about the altered state of the world. Marple-Horvat and colleagues built a circular ladder assay for cats that was specifically designed to record from motor cortex during such conditions \cite{Marple-Horvat1993}. One of the modifications they introduced was to make one of the rungs of the ladder fall unexpectedly under the weight of the animal. When they recorded from motor cortical neurons during the rung drop, they noticed a marked increase in activity, well above the recorded baseline from normal stepping, as the animal recovered from the fall and resumed walking. However, whether this increased activity of motor cortex was necessary for the recovery response has never been assayed.

\subsubsection*{Some remarks on lesion techniques}

The original methods used to induce a permanent lesion to the motor cortex were very crude, often involving gross mechanical aggression to the neural tissue by using surgical knife cuts or ablation by water-jet, aspiration, and thermo- or electrocoagulation. These methods are still widely used in lesion studies for their simplicity and bluntness, but have the disadvantage of making it hard to limit the lesion to a single area because of possible damage to subcortical areas or the destruction of fibers of passage. Fibers of passage are nerve fibers passing through the lesioned area which neither originate nor terminate in the region of interest. These limitations made it more difficult to interpret the effects of cortical lesions, and eventually led to the development of new techniques designed to work around such problems. Chemical injections of neurotoxic compounds such as ibotenic acid or kainic acid aim to increase selectivity of the lesion by limiting damage to neural cell bodies in the target area while leaving the fibers of passage intact \cite{Schwarcz1979}. Photothrombosis \cite{Watson1985} or devascularization by pial stripping \cite{Meyer1971} aim to reproduce the effects of clinical stroke while avoiding extension of the lesion to subcortical areas as much as possible.

The early studies of Broca localizing the function of articulate language to a specific region in the cerebral hemispheres \cite{Broca1861} established a long tradition of correlating the location of surgical brain injury with detailed analysis of any subsequent behavioural deficits. This method is not without its difficulties. The problems of plasticity and diaschisis will forever complicate conclusions based on injury and manipulation of nervous tissue \cite{Lashley1933}. Many recent methods for reversible chemical or optogenetic inactivation of the cortex have been proposed to improve statistical power of behavioural assessments \cite{DeFeudis1980,Dong2010,Guo2015}. Unfortunately, given that the cortex maintains a tight balance of excitation and inhibition during normal functioning and is also densely interconnected with the rest of the brain, the effects of such transient manipulations are prone to cause multiple downstream effects that can confound inferences about behavioural relevance \cite{Otchy2015}. In this respect, they are similar to stimulation experiments in that they are very useful in determining that two areas are connected in a circuit, but not necessarily what the connection means. Of course, permanent lesions themselves can induce plasticity changes in the function of downstream and upstream circuits. The expectation, however, is that such changes represent a homeostatically stable state of the system, allowing simultaneous investigation of the limits of recovery, as well as the kinds of problems for which a fully intact structure is definitely required.
