\section{Results}

To investigate whether the intact motor cortex is required for the robust control of movement in response to unexpected perturbations, we designed a reconfigurable dynamic obstacle course where individual steps can be made stable or unstable on a trial-by-trial basis (Figure \ref{fig:assay}, also see Methods). In this assay, rats shuttle back and forth across the obstacles, in the dark, in order to collect water rewards. We specifically designed the assay such that modifications to the physics of the obstacles could be made covertly. In this way, the animal has no explicit information about the state of the steps until it actually contacts them. Water deprived animals were trained daily for 4 weeks, throughout which they encountered increasingly challenging states of the obstacle course. Our goal was to characterize precisely the conditions under which motor cortex becomes necessary for the control of movement, and this motivated us to introduce an environment with graded levels of uncertainty.

We compared the performance of 22 animals: 11 with bilateral ibotenic acid lesions to the primary and secondary forelimb motor cortex, and 11 age and gender matched controls (5 sham surgery, 6 wild-types). Animals were given ample time to recover, 4 weeks post-surgery, in order to specifically isolate behaviours that are chronically impaired in animals lacking the functions enabled by motor cortical structures. Histological examination of serial coronal sections revealed significant variability in the extent of damaged areas (Figure \ref{fig:histology}), which was likely caused by mechanical blockage of the injection pipette during lesion induction at some sites. Nevertheless, volume reconstruction of the serial sections allowed us to accurately quantify the size of each lesion, identify each animal (from Lesion A to Lesion K; largest to smallest), and use these values to compare observed behavioural effects as a function of lesion size.

\begin{figure}
\centering
\includestandalone[scale=0.85]{chapters/figuresChBehaviour/histology}
\caption{Histological analysis of lesion size. (\textbf{A}) Representative example of Nissl-stained coronal section showing bilateral ibotenic acid lesion of primary and secondary forelimb motor cortex. (\textbf{B}) Distribution of lesion volumes in the left and right hemispheres for individual animals. A lesion was considered ``large'' if the total lesion volume was above \SI{15}{\milli\meter\cubed}. (\textbf{C}) Super-imposed reconstruction stacks for all the small lesions ($n = 6$). (\textbf{D}) Super-imposed reconstruction stacks for all the large lesions ($n = 5$).}
\label{fig:histology}
\end{figure}

During the first sessions in the ``stable'' environment, all animals, both lesions and controls, quickly learned to shuttle across the obstacles, achieving stable, skilled performance after a few days of training (Figure \ref{fig:learning}). Even though the distance between steps was fixed for all animals, the time taken to adapt the crossing strategy was similar irrespective of body size. When first encountering the obstacles, animals adopted a cautious gait, investigating the location of the subsequent obstacle with their whiskers, stepping with the leading forepaw followed by a step to the same position with the trailing paw (Video \ref{vid:learning}: ``First Leftwards Crossing''). However, over the course of only a few trials, all animals exhibited a new strategy of ``stepping over'' the planted forepaw to the next obstacle, suggesting an increased confidence in their movement strategy in this novel environment (Video \ref{vid:learning}: ``Second Leftwards Crossing''). This more confident gait developed into a coordinated locomotion sequence after a few additional training sessions (Video \ref{vid:learning}: ``Later Crossing''). The development of the ability to move confidently and quickly over the obstacle course was observed in both lesion and control animals (Video \ref{vid:learning-matrix}).

\begin{figure}
\centering
\includestandalone[scale=0.9]{chapters/figuresChBehaviour/learning}
\caption{Overall performance on the obstacle course is similar for both lesion ($n = 11$) and control animals ($n = 11$) across the different protocol stages. Each set of coloured bars represents the distribution of average time to cross the obstacles on a single session. Asterisks indicate sessions where there was a change in assay conditions during the session (see text). In these transition sessions, the average performance on the 20 trials immediately preceding the change is shown to the left of the solid vertical line whereas the performance on the remainder of that session (after the change) is shown to the right.}
\label{fig:learning}
\end{figure}

In addition to the excitotoxic lesions, in three animals we performed larger frontal cortex aspiration lesions in order to determine whether the remaining trunk and hindlimb representations were necessary to navigate the elevated obstacle course. Also, in order to exclude the involvement of other corticospinal projecting regions in the parietal and rostral visual areas \cite{Miller1987}, we included three additional animals which underwent even more extensive cortical lesion procedures (Figure \ref{fig:extended}A,B, see Methods). These \emph{extended} lesion animals were identified following chronological order (from Extended Lesion A to Extended Lesion F; where the first three animals correspond to frontal cortex aspiration lesions and the remaining animals to the more extensive frontoparietal lesions). In these extended cortical lesions, recovery was found to be overall slower than in lesions limited to the motor cortex, and animals required isolation and more extensive care during the recovery period.

Nevertheless, when tested in the shuttling assay, the basic performance of these extended lesion animals was similar to that of controls and animals with excitotoxic motor cortical lesions (Figure \ref{fig:extended}C). Animals with large frontoparietal lesions did exhibit a very noticeable deficit in paw placement throughout the early sessions (Figure \ref{fig:extended}D). Interestingly, detailed analysis of paw placement behaviour revealed that this deficit was almost entirely explained by impaired control of the hindlimbs. Paw slips were much more frequent when stepping with a hindlimb than with a forelimb (Figure \ref{fig:extended}E,F). In addition, when a slip did occur, these animals failed to adjust the affected paw to compensate for the fall (e.g. keeping their digits closed), which significantly impacted their overall posture recovery. These deficits in paw placement are consistent with results from sectioning the entire pyramidal tract in cats \cite{Liddell1944}, and reports in ladder walking following motor cortical lesion in rodents \cite{Metz2002}, but surprisingly we did not observe deficits in paw placement in animals with ibotenic acid lesions limited to forelimb motor cortex (Figure \ref{fig:extended}D). Furthermore, despite this initial impairment, animals with extended lesions were still able to improve their motor control strategy up to the point where they were moving across the obstacles as efficiently as controls and other lesioned animals (Figure \ref{fig:extended}C, Video \ref{vid:learning-matrix}). Indeed, in the largest frontoparietal lesion, which extended all the way to rostral visual cortex, recovery of a stable locomotion pattern was evident over the course of just ten repeated trials (Video \ref{vid:decorticate-habituation}). The ability of this animal to improve its motor control strategy in such a short period of time seems to indicate the presence of motor learning, not simply an increase in confidence with the new environment.

In subsequent training sessions we progressively increased the difficulty of the obstacle course, by making more steps unstable. The goal was to compare the performance of the two groups as a function of difficulty. Surprisingly, both lesion and control animals were able to improve their performance by the end of each training stage even for the most extreme condition where all steps were unstable (Figure \ref{fig:learning}, Video \ref{vid:conditions}). This seems to indicate that the ability of these animals to fine-tune their motor performance in a challenging environment remained intact.

One noticeable exception was the animal with the largest ibotenic acid lesion. This animal, following exposure to the first unstable protocol, was unable to bring itself to cross the obstacle course (Video \ref{vid:jpak20}). Some other control and lesioned animals also experienced a similar form of distress following exposure to the unstable obstacles, but eventually all these animals managed to start crossing over the course of a single session. In order to test whether this was due to some kind of motor disability, we lowered the difficulty of the protocol for this one animal until it was able to cross again. Following a random permutation protocol, where any two single steps were released randomly, this animal was then able to cross a single released obstacle placed in any location of the assay. After this success, it eventually learned to cross the highest difficulty level in the assay in about the same time as all the other animals, suggesting that there was indeed no lasting motor execution or learning deficit, and that the disability must have been due to some other unknown, yet intriguing, (cognitive) factor.

Having established that the overall motor performance of these animals was similar across all conditions, we next asked whether there was any difference in the strategy used by the two groups of animals to cross the unstable obstacles. We noticed that during the first week of training, the posture of the animals when stepping on the obstacles changed significantly over time (Figure \ref{fig:posture}B,C). Specifically, the centre of gravity of the body was shifted further forward and higher during later sessions, in a manner proportional to performance. However, after the obstacles changed to the unstable state, we observed an immediate and persistent adjustment of this crossing posture, with animals assuming a lower centre of gravity and reducing their speed as they approached the unstable obstacles (Figure \ref{fig:posture}C,D). Interestingly, we also noticed that a group of animals adopted a different strategy. Instead of lowering their centre of gravity, they either kept it unchanged or shifted it even more forward and performed a jump over the unstable obstacles (Figure \ref{fig:jumping}A,B). These two strategies were remarkably consistent across the two groups, but there was no correlation between the strategy used and the degree of motor cortical lesion (Figure \ref{fig:posture}E,F, \ref{fig:jumping}C). In fact, we found that the use of a jumping strategy was best predicted by the body weight of the animal (Figure \ref{fig:jumping}C).

During the two days where the stable state of the environment was reinstated, the posture of the animals was gradually restored to pre-manipulation levels (Figure \ref{fig:posture}B,C), although in many cases this adjustment happened at a slower rate than the transition from stable to unstable. Again, this postural adaptation was independent of the presence or absence of forepaw motor cortex.

We next looked in detail at the days where the state of the obstacle course was randomized on a trial-by-trial basis. This stage of the protocol is particularly interesting as it reflects a situation where the environment has a persistent degree of uncertainty. For this analysis, we were forced to exclude the animals that employed a jumping strategy, as their experience with the manipulated obstacles was the same irrespective of the state of the world. First, we repeated the same posture analysis comparing all the stable and unstable trials in the random protocol in order to control for whether there was any subtle cue in our motorized setup that the animals might be using to gain information about the current state of the world. There was no significant difference between randomly presented stable and unstable trials on the approach posture of the animal (Figure \ref{fig:random}A). However, classifying the trials on the basis of past trial history revealed a significant effect on posture (Figure \ref{fig:random}B). This suggested that the animals were adjusting their body posture when stepping on the affected obstacles on the basis of their current expectation about the state of the world, which is updated by the previously experienced state. Surprisingly, this effect again did not depend on the presence or absence of frontal motor cortical structures (Figure \ref{fig:random}C,D).

Finally, we decided to test whether general motor performance was affected by the randomized state of the obstacles. If the animals do not know what state the world will be in, then there will be an increased challenge to their stability when they cross over the unstable obstacles, possibly demanding a quick change in strategy when they learn whether the world is stable or unstable. In order to evaluate the dynamics of crossing, we compared the speed profile of each animal across these different conditions (Figure \ref{fig:speed}, see Methods). Interestingly, two of the animals with the largest lesions appeared to be significantly slowed down on unstable trials, while controls and the animals with the smallest lesions instead tended to accelerate after encountering an unstable obstacle. However, the overall effect for lesions versus controls was not statistically significant (Figure \ref{fig:speed}C).

Nevertheless, we were intrigued by this observation and decided to investigate, in detail, the first moment in the assay when a perturbation is encountered. In the random protocol, even though the state of the world is unpredictable, the animals know that the obstacles might become unstable. However, the very first time the environment becomes unstable, the collapse of the obstacles is completely unexpected and demands an entirely novel motor response.

A detailed analysis of the responses to the first collapse of the steps revealed a striking difference in the strategies deployed by the lesion and control animals. Upon the first encounter with the manipulated steps, we observed three types of behavioural responses from the animals (Video \ref{vid:manipulation-strategies}): investigation, in which the animals immediately stop their progression and orient towards, whisk, and physically manipulate the altered obstacle; compensation, in which the animals rapidly adjust their behaviour to negotiate the unexpected instability; and halting, in which the ongoing motor program ceases and the animals' behaviour simply comes to a stop for several seconds. Remarkably, these responses depended on the presence or absence of motor cortex (Figure \ref{fig:ethogram}). Animals with the largest motor cortical lesions, upon their first encounter with the novel environmental obstacle, halted for several seconds, whereas animals with an intact motor cortex, and those with the smallest lesions, were able to rapidly react with either an investigatory or compensatory response (Video \ref{vid:manipulation-small},\ref{vid:manipulation-large}).

The response of animals with extended lesions was even more striking. In two of these animals, there was a failure to recognize that a change had occurred at all (Video \ref{vid:manipulation-decorticate-oblivious}). Instead, they kept walking across the now unstable steps for several trials, never stopping to assess the new situation. One of them gradually noticed the manipulation and stopped his progression, while the other one only fully realized the change after inadvertently hitting the steps with its snout (Video \ref{vid:manipulation-decorticate-oblivious}: Extended Lesion A). This was the first time we ever observed this behaviour, as all animals with or without cortical lesions always displayed a clear switch in behavioural state following the first encounter with the manipulation. In the remaining animals with extended lesions, two of them clearly halted their progression following the collapse of the obstacles, in a way similar to the large motor cortex ibotenic lesions (Video \ref{vid:manipulation-decorticate-halting}). The third animal (Extended Lesion B) actually collapsed upon contact with the manipulated step, falling over its paw and digits awkwardly and hitting the obstacles with its snout. Shortly after this there was a switch to an exploratory behaviour state, in a way similar to Extended Lesion A.
