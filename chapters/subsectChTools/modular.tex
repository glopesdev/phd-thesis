				
\section{Introduction}

The formal study of animal behaviour has a long history spanning hundreds of years across the fields of ethology, experimental psychology and neuroscience. While the ethologists mainly endeavoured to study behaviour in its natural environment, the psychologists and neurophysiologists have classically resorted, of necessity, to more controlled laboratory settings. The reason is mainly one of complexity. Behaviour is a highly multi-dimensional, multi-scale phenomenon that often allows no clear separation between relevant and irrelevant variables \cite{Gomez-Marin2014}. It is in general impossible to predict what an animal is going to do simply because some of the crucial information is not even accessible to measurement. In order to mitigate this problem, neuroscientists resort to making impoverished preparations where the number of variables that are changing at any given moment is low and very carefully controlled. The hope is that in this way the interpretation of brain signals recorded simultaneously with animal behaviour will be facilitated.

Depending on the kind of question a neuroscientist is after, an appropriate behaviour paradigm is set up. Anaesthetized and head-fixed preparations, as well as classical or operant conditioning boxes are regularly employed to drive the behaviour of the animal to oscillate between a set of repeatedly reproducible states more amenable to statistical analysis. Building such behaviour assays often requires very specialized engineering skills and long development cycles of trial and error in order to ensure all the relevant variables are controlled accordingly. Because of this, the tendency of the field has been to concentrate on a small set of ``standardized'' assays which have been shown to work for one area of research or other. Small variations to the standard tasks are gradually introduced in order to probe different aspects of the system. The complexity of behaviour studies in neuroscience has thus traditionally progressed by attrition and painstaking accumulation of small perturbations to overall design patterns.

Interestingly, however, many of the most significant conceptual advances in our understanding of brain function have in fact developed \emph{pari passu} with forays into entirely new behaviour spaces. Moving from anaesthetized to awake physiology completely changed the way we understand the neural processing of sensory stimuli \cite{Sellers2015}. Similarly, moving from head-fixed to freely moving behaviour led to the discovery of place fields in hippocampus \cite{OKeefe1971}. Single trial analysis of simultaneously recorded responses have revealed patterns of neural activity such as hippocampal ripples that are simply impossible to recover from statistical averages of repetitive behaviour episodes \cite{Foster2006,Davidson2009}. Each of these developments has required significant advances in tools used to record and control behavioural data at a fine scale. Unfortunately, the technical cost and scientific risk of trying something novel means that such advances are still much fewer and far between than would be desirable.

From the beginning of this work it was understood that revealing the teleology of cortical control over behaviour would require just this kind of foray into diverse and potentially unknown behaviour spaces. We agreed that it might be worth to try and develop a toolkit for the behavioural neuroscientist that would accelerate the exploration of this vast space. One of the first obvious targets for improvement was the behaviour box. Traditionally, when a given behaviour assay is found to produce interesting results, its design is progressively tweaked so as to exacerbate the features of the original effect. In this work, we started by breaking apart this concept of the polished behaviour box, and wondered what would happen if instead of a standard box, we could have a box of standards.

\section{The Modular Behaviour Box}

At the outset it was decided that the scale of the modular architecture would probably have to match a given animal model, given the vastly different size scales between rodents, cats and primates. Our animal model of choice is the rodent \emph{rattus norvegicus}, and all of our proposed design choices target its size scale. Small adjustments could, however, be reasonably made up to a point for other mammals of similar stature, such as mice.

The main component and interface of the modular box is the individual $1\times 1$ module (Figure \ref{fig:modules}A). This module defines a standardized footprint ($\SI{12}{\centi\meter}\times \SI{12}{\centi\meter}$), against which all other modules are measured. Every newly fabricated module is built to specification to match a multiple of this standardized footprint (e.g. it is possible to have $2\times 1$, $2\times 2$, $4\times 1$ or any other multiple combination of the standard size). Inside the module footprint the module designer places a single logical component of a behaviour box and ensures that it can operate in isolation. Figure \ref{fig:modules} shows some examples of reusable modules developed throughout the project.

\begin{figure}
\centering
\includestandalone[scale=0.90]{chapters/figuresChTools/modules}
\caption{Some examples of standardized behaviour modules. (\textbf{A}) Detail of a $1\times 1$ module mounted in support frame. Fixation is achieved by driving a screw through post-insertion nuts placed in the structural framing (see text). (\textbf{B}) Example reward port module which can be floor- or wall-mounted. All relevant electronics and water distribution circuits are assembled on the back of the module (not shown). (\textbf{C}) Wall-mounted reconfigurable obstacle course stepper module. Stepper motors mounted on the back of the module allow for dynamic reconfiguration of the orientation of each step. (\textbf{D}) Floor-mounted obstacle course step pair. Multiple of these modules can be tiled together to assemble obstacle courses of arbitrary length.}
\label{fig:modules}
\end{figure}

\begin{figure}
\centering
\includestandalone[scale=1.00]{chapters/figuresChTools/box}
\caption{Example of a linear shuttling box assembled from a $\SI{1}{\meter}\times \SI{1}{\meter}$ modular structure using reward port and obstacle course step modules.}
\label{fig:box}
\end{figure}

\begin{figure}
\centering
\includestandalone[scale=0.90]{chapters/figuresChTools/box3d}
\caption{Side view of the linear shuttling box.}
\label{fig:box3d}
\end{figure}

\begin{figure}
\centering
\includestandalone[scale=1.00]{chapters/figuresChTools/verticalBox}
\caption{Example of vertical assembly. (\textbf{A}) Detail of a $1\times 1$ wall-mounted platform module. (\textbf{B}) Example of a vertical maze configuration.}
\label{fig:verticalBox}
\end{figure}

One of the principal requirements for assembling a box is fastening all its components together. By having a standard footprint, it is possible to design a set of regularly spaced mounting points that allows the experimentalist to quickly generate an entirely new configuration by simply swapping modular components inside the box (Figure \ref{fig:box}, \ref{fig:box3d}). For this work, we took advantage of an existing aluminium structural framing system (Bosch Rexroth, DE) to build the common mounting points (Figure \ref{fig:modules}A). Modules are fastened against post-insertion nuts which are able to slide across the whole length of the aluminium rail. Each of the modules is fastened by four screws, one in each corner. In order to ensure modules can be tightly and securely fixed one next to the other, we used a system of regularly spaced double rails (Figure \ref{fig:box}). This gives the frame the flexibility to easily reposition and rearrange individual modules tiling the entire footprint of any arbitrarily large box.

If the support frame is laid out vertically, it is possible to create modular walls of arbitrary dimensions. Some of the modules can be mounted equally well on a vertical or horizontal configuration, such as reward ports (Figure \ref{fig:modules}B). The three-dimensionality of the design has even been exploited to create vertical mazes (Figure \ref{fig:verticalBox}) to great success.

Throughout the project we made the base of every module from \SI{5}{\milli\meter} acrylic pieces. While not an absolute requirement for the design, this choice of plastic material has the advantage that a laser cutter can be used to very quickly produce a large collection of custom-built modules. In addition, patterns can be engraved or cut on the base to provide additional mounting points for hardware embedded in the module. The use of such rapid prototyping fabrication tools alongside with off the shelf available electronic sensors and actuators meant we were able to completely redesign the entire behaviour box, sometimes in a matter of days.
