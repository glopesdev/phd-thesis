				
\section{Introduction}

The formal study of animal behaviour has a long history spanning hundreds of years across the fields of ethology, experimental psychology and neuroscience. While the ethologists mainly endeavoured to study behaviour in its natural environment, the psychologists and neurophysiologists have classically resorted, of necessity, to more controlled laboratory settings. The reason is mainly one of complexity. Behaviour is a highly multi-dimensional, multi-scale phenomenon that often allows no clear separation between relevant and irrelevant variables \cite{Gomez-Marin2014}. It is in general impossible to predict what an animal is going to do simply because some of the crucial information is not even accessible to measurement. In order to mitigate this problem, neuroscientists resort to making impoverished preparations where the number of variables that are changing at any given moment is low and very carefully controlled. The hope is that in this way the interpretation of brain signals recorded simultaneously with animal behaviour will be facilitated.

Depending on the kind of question a neuroscientist is after, an appropriate behaviour paradigm is set up. Anaesthetized and head-fixed preparations, as well as classical or operant conditioning boxes are regularly employed to drive the behaviour of the animal to oscillate between a set of repeatedly reproducible states more amenable to statistical analysis. Building such behaviour assays often requires very specialized engineering skills and long development cycles of trial and error in order to ensure all the relevant variables are controlled accordingly. Because of this, the tendency of the field has been to concentrate on a small set of ``standardized'' assays which have been shown to work for one area of research or other. Small variations to the standard tasks are gradually introduced in order to probe different aspects of the system. The complexity of behaviour studies in neuroscience has thus traditionally progressed by attrition and painstaking accumulation of small perturbations to overall design patterns.

Interestingly, however, many of the most significant conceptual advances in our understanding of brain function have in fact developed \emph{pari passu} with forays into entirely new behaviour spaces. Moving from anaesthetized to awake physiology completely changed the way we understand the neural processing of sensory stimuli \cite{Sellers2015}. Similarly, moving from head-fixed to freely moving behaviour led to the discovery of place fields in hippocampus \cite{OKeefe1971}. Single trial analysis of simultaneously recorded responses have revealed patterns of neural activity such as hippocampal ripples that are simply impossible to recover from statistical averages of repetitive behaviour episodes \cite{Foster2006,Davidson2009}. Each of these developments has required significant advances in tools used to record and control behavioural data at a fine scale. Unfortunately, the technical cost and scientific risk of trying something novel means that such advances are still much fewer and far between than would be desirable.

From the beginning of this work it was understood that revealing the teleology of cortical control over behaviour would require just this kind of foray into diverse and potentially unknown behaviour spaces. We agreed that it might be worth to try and develop a toolkit for the behavioural neuroscientist that would accelerate the exploration of this vast space. One of the first obvious targets for improvement was the behaviour box. Traditionally, when a given behaviour assay is found to produce interesting results, its design is progressively tweaked so as to exacerbate the features of the original effect. In this work, we started by breaking apart this concept of the polished behaviour box, and wondered what would happen if instead of a standard box, we could have a box of standards.
