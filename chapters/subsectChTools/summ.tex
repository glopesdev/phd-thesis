				
\section{Chapter Summary}

The study of animal behaviour has provided tremendous insight into the functions of the brain. However, in the laboratory, behaviour is often studied in impoverished and repetitive regimes in order to control its complexity. In this chapter, we introduce a set of hardware and software tools that make it easier to rapidly survey a larger range of environments without losing control over observable behavioural data.

In the first part of the chapter, we introduce an architecture for a multi-purpose modular behaviour box. This architecture makes it possible to use simple fabrication and rapid prototyping tools to quickly reconfigure a physical environment for different assays requiring complex combinations of sensors and actuators.

In the second part, we present Bonsai, a high-performance visual programming language for controlling and monitoring real-time data streams on a digital computer. We describe Bonsai's core principles and architecture and demonstrate how it allows for the rapid and flexible prototyping of integrated experimental designs in neuroscience. We specifically highlight some applications that require the combination of many different hardware and software components, including video tracking of behavior, electrophysiology and closed-loop control of stimulation.

\pagebreak


