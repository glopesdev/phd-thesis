\section{Visit to the Academy of Lagado}

This academy is not an entire single building, but a continuation of several houses on both sides of a street, which growing waste, was purchased and applied to that use.


I was received very kindly by the warden, and went for many days to the academy.  Every room has in it one or more projectors; and I believe I could not be in fewer than five hundred rooms.

The first man I saw was of a meagre aspect, with sooty hands and face, his hair and beard long, ragged, and singed in several places.  His clothes, shirt, and skin, were all of the same colour.  He has been eight years upon a project for extracting sunbeams out of cucumbers, which were to be put in phials hermetically sealed, and let out to warm the air in raw inclement summers.  He told me, he did not doubt, that, in eight years more, he should be able to supply the governor’s gardens with sunshine, at a reasonable rate: but he complained that his stock was low, and entreated me “to give him something as an encouragement to ingenuity, especially since this had been a very dear season for cucumbers.”  I made him a small present, for my lord had furnished me with money on purpose, because he knew their practice of begging from all who go to see them.

I went into another chamber, but was ready to hasten back, being almost overcome with a horrible stink.  My conductor pressed me forward, conjuring me in a whisper “to give no offence, which would be highly resented;” and therefore I durst not so much as stop my nose.  The projector of this cell was the most ancient student of the academy; his face and beard were of a pale yellow; his hands and clothes daubed over with filth.  When I was presented to him, he gave me a close embrace, a compliment I could well have excused.  His employment, from his first coming into the academy, was an operation to reduce human excrement to its original food, by separating the several parts, removing the tincture which it receives from the gall, making the odour exhale, and scumming off the saliva.  He had a weekly allowance, from the society, of a vessel filled with human ordure, about the bigness of a Bristol barrel.

I saw another at work to calcine ice into gunpowder; who likewise showed me a treatise he had written concerning the malleability of fire, which he intended to publish.

There was a most ingenious architect, who had contrived a new method for building houses, by beginning at the roof, and working downward to the foundation; which he justified to me, by the like practice of those two prudent insects, the bee and the spider.

There was a man born blind, who had several apprentices in his own condition: their employment was to mix colours for painters, which their master taught them to distinguish by feeling and smelling.  It was indeed my misfortune to find them at that time not very perfect in their lessons, and the professor himself happened to be generally mistaken.  This artist is much encouraged and esteemed by the whole fraternity.

In another apartment I was highly pleased with a projector who had found a device of ploughing the ground with hogs, to save the charges of ploughs, cattle, and labour.  The method is this: in an acre of ground you bury, at six inches distance and eight deep, a quantity of acorns, dates, chestnuts, and other mast or vegetables, whereof these animals are fondest; then you drive six hundred or more of them into the field, where, in a few days, they will root up the whole ground in search of their food, and make it fit for sowing, at the same time manuring it with their dung: it is true, upon experiment, they found the charge and trouble very great, and they had little or no crop.  However it is not doubted, that this invention may be capable of great improvement.

I went into another room, where the walls and ceiling were all hung round with cobwebs, except a narrow passage for the artist to go in and out.  At my entrance, he called aloud to me, “not to disturb his webs.”  He lamented “the fatal mistake the world had been so long in, of using silkworms, while we had such plenty of domestic insects who infinitely excelled the former, because they understood how to weave, as well as spin.”  And he proposed further, “that by employing spiders, the charge of dyeing silks should be wholly saved;” whereof I was fully convinced, when he showed me a vast number of flies most beautifully coloured, wherewith he fed his spiders, assuring us “that the webs would take a tincture from them; and as he had them of all hues, he hoped to fit everybody’s fancy, as soon as he could find proper food for the flies, of certain gums, oils, and other glutinous matter, to give a strength and consistence to the threads.”

There was an astronomer, who had undertaken to place a sun-dial upon the great weathercock on the town-house, by adjusting the annual and diurnal motions of the earth and sun, so as to answer and coincide with all accidental turnings of the wind.

I was complaining of a small fit of the colic, upon which my conductor led me into a room where a great physician resided, who was famous for curing that disease, by contrary operations from the same instrument.  He had a large pair of bellows, with a long slender muzzle of ivory: this he conveyed eight inches up the anus, and drawing in the wind, he affirmed he could make the guts as lank as a dried bladder.  But when the disease was more stubborn and violent, he let in the muzzle while the bellows were full of wind, which he discharged into the body of the patient; then withdrew the instrument to replenish it, clapping his thumb strongly against the orifice of then fundament; and this being repeated three or four times, the adventitious wind would rush out, bringing the noxious along with it, (like water put into a pump), and the patient recovered.  I saw him try both experiments upon a dog, but could not discern any effect from the former.  After the latter the animal was ready to burst, and made so violent a discharge as was very offensive to me and my companion.  The dog died on the spot, and we left the doctor endeavouring to recover him, by the same operation.

I visited many other apartments, but shall not trouble my reader with all the curiosities I observed, being studious of brevity.

I had hitherto seen only one side of the academy, the other being appropriated to the advancers of speculative learning, of whom I shall say something, when I have mentioned one illustrious person more, who is called among them “the universal artist.”  He told us “he had been thirty years employing his thoughts for the improvement of human life.”  He had two large rooms full of wonderful curiosities, and fifty men at work.  Some were condensing air into a dry tangible substance, by extracting the nitre, and letting the aqueous or fluid particles percolate; others softening marble, for pillows and pin-cushions; others petrifying the hoofs of a living horse, to preserve them from foundering.  The artist himself was at that time busy upon two great designs; the first, to sow land with chaff, wherein he affirmed the true seminal virtue to be contained, as he demonstrated by several experiments, which I was not skilful enough to comprehend.  The other was, by a certain composition of gums, minerals, and vegetables, outwardly applied, to prevent the growth of wool upon two young lambs; and he hoped, in a reasonable time to propagate the breed of naked sheep, all over the kingdom.

We crossed a walk to the other part of the academy, where, as I have already said, the projectors in speculative learning resided.

The first professor I saw, was in a very large room, with forty pupils about him.  After salutation, observing me to look earnestly upon a frame, which took up the greatest part of both the length and breadth of the room, he said, “Perhaps I might wonder to see him employed in a project for improving speculative knowledge, by practical and mechanical operations.  But the world would soon be sensible of its usefulness; and he flattered himself, that a more noble, exalted thought never sprang in any other man’s head.  Every one knew how laborious the usual method is of attaining to arts and sciences; whereas, by his contrivance, the most ignorant person, at a reasonable charge, and with a little bodily labour, might write books in philosophy, poetry, politics, laws, mathematics, and theology, without the least assistance from genius or study.”  He then led me to the frame, about the sides, whereof all his pupils stood in ranks.  It was twenty feet square, placed in the middle of the room.  The superfices was composed of several bits of wood, about the bigness of a die, but some larger than others.  They were all linked together by slender wires.  These bits of wood were covered, on every square, with paper pasted on them; and on these papers were written all the words of their language, in their several moods, tenses, and declensions; but without any order.  The professor then desired me “to observe; for he was going to set his engine at work.”  The pupils, at his command, took each of them hold of an iron handle, whereof there were forty fixed round the edges of the frame; and giving them a sudden turn, the whole disposition of the words was entirely changed.  He then commanded six-and-thirty of the lads, to read the several lines softly, as they appeared upon the frame; and where they found three or four words together that might make part of a sentence, they dictated to the four remaining boys, who were scribes.  This work was repeated three or four times, and at every turn, the engine was so contrived, that the words shifted into new places, as the square bits of wood moved upside down.


Six hours a day the young students were employed in this labour; and the professor showed me several volumes in large folio, already collected, of broken sentences, which he intended to piece together, and out of those rich materials, to give the world a complete body of all arts and sciences; which, however, might be still improved, and much expedited, if the public would raise a fund for making and employing five hundred such frames in Lagado, and oblige the managers to contribute in common their several collections.

He assured me “that this invention had employed all his thoughts from his youth; that he had emptied the whole vocabulary into his frame, and made the strictest computation of the general proportion there is in books between the numbers of particles, nouns, and verbs, and other parts of speech.”

I made my humblest acknowledgment to this illustrious person, for his great communicativeness; and promised, “if ever I had the good fortune to return to my native country, that I would do him justice, as the sole inventor of this wonderful machine;” the form and contrivance of which I desired leave to delineate on paper, as in the figure here annexed.  I told him, “although it were the custom of our learned in Europe to steal inventions from each other, who had thereby at least this advantage, that it became a controversy which was the right owner; yet I would take such caution, that he should have the honour entire, without a rival.”

We next went to the school of languages, where three professors sat in consultation upon improving that of their own country.

The first project was, to shorten discourse, by cutting polysyllables into one, and leaving out verbs and participles, because, in reality, all things imaginable are but norms.

The other project was, a scheme for entirely abolishing all words whatsoever; and this was urged as a great advantage in point of health, as well as brevity.  For it is plain, that every word we speak is, in some degree, a diminution of our lunge by corrosion, and, consequently, contributes to the shortening of our lives.  An expedient was therefore offered, “that since words are only names for things, it would be more convenient for all men to carry about them such things as were necessary to express a particular business they are to discourse on.”  And this invention would certainly have taken place, to the great ease as well as health of the subject, if the women, in conjunction with the vulgar and illiterate, had not threatened to raise a rebellion unless they might be allowed the liberty to speak with their tongues, after the manner of their forefathers; such constant irreconcilable enemies to science are the common people.  However, many of the most learned and wise adhere to the new scheme of expressing themselves by things; which has only this inconvenience attending it, that if a man’s business be very great, and of various kinds, he must be obliged, in proportion, to carry a greater bundle of things upon his back, unless he can afford one or two strong servants to attend him.  I have often beheld two of those sages almost sinking under the weight of their packs, like pedlars among us, who, when they met in the street, would lay down their loads, open their sacks, and hold conversation for an hour together; then put up their implements, help each other to resume their burdens, and take their leave.

But for short conversations, a man may carry implements in his pockets, and under his arms, enough to supply him; and in his house, he cannot be at a loss.  Therefore the room where company meet who practise this art, is full of all things, ready at hand, requisite to furnish matter for this kind of artificial converse.

Another great advantage proposed by this invention was, that it would serve as a universal language, to be understood in all civilised nations, whose goods and utensils are generally of the same kind, or nearly resembling, so that their uses might easily be comprehended.  And thus ambassadors would be qualified to treat with foreign princes, or ministers of state, to whose tongues they were utter strangers.

I was at the mathematical school, where the master taught his pupils after a method scarce imaginable to us in Europe.  The proposition, and demonstration, were fairly written on a thin wafer, with ink composed of a cephalic tincture.  This, the student was to swallow upon a fasting stomach, and for three days following, eat nothing but bread and water.  As the wafer digested, the tincture mounted to his brain, bearing the proposition along with it.  But the success has not hitherto been answerable, partly by some error in the quantum or composition, and partly by the perverseness of lads, to whom this bolus is so nauseous, that they generally steal aside, and discharge it upwards, before it can operate; neither have they been yet persuaded to use so long an abstinence, as the prescription requires.
