% !TEX TS-program = pdflatex
% !TEX encoding = UTF-8 Unicode

% This is a simple template for a LaTeX document using the "article" class.
% See "book", "report", "letter" for other types of document.

\documentclass[11pt]{article} % use larger type; default would be 10pt

\usepackage[utf8]{inputenc} % set input encoding (not needed with XeLaTeX)

%%% Examples of Article customizations
% These packages are optional, depending whether you want the features they provide.
% See the LaTeX Companion or other references for full information.

%%% PAGE DIMENSIONS
\usepackage{geometry} % to change the page dimensions
\geometry{a4paper} % or letterpaper (US) or a5paper or....
% \geometry{margin=2in} % for example, change the margins to 2 inches all round
% \geometry{landscape} % set up the page for landscape
%   read geometry.pdf for detailed page layout information

\usepackage{graphicx} % support the \includegraphics command and options

% \usepackage[parfill]{parskip} % Activate to begin paragraphs with an empty line rather than an indent

%%% PACKAGES
\usepackage{booktabs} % for much better looking tables
\usepackage{array} % for better arrays (eg matrices) in maths
\usepackage{paralist} % very flexible & customisable lists (eg. enumerate/itemize, etc.)
\usepackage{verbatim} % adds environment for commenting out blocks of text & for better verbatim
\usepackage{subfig} % make it possible to include more than one captioned figure/table in a single float
% These packages are all incorporated in the memoir class to one degree or another...
\usepackage{hyperref}

%%% HEADERS & FOOTERS
\usepackage{fancyhdr} % This should be set AFTER setting up the page geometry
\pagestyle{fancy} % options: empty , plain , fancy
\renewcommand{\headrulewidth}{0pt} % customise the layout...
\lhead{}\chead{}\rhead{}
\lfoot{}\cfoot{\thepage}\rfoot{}

%%% SECTION TITLE APPEARANCE
\usepackage{sectsty}
\allsectionsfont{\sffamily\mdseries\upshape} % (See the fntguide.pdf for font help)
% (This matches ConTeXt defaults)

%%% ToC (table of contents) APPEARANCE
\usepackage[nottoc,notlof,notlot]{tocbibind} % Put the bibliography in the ToC
\usepackage[titles,subfigure]{tocloft} % Alter the style of the Table of Contents
\renewcommand{\cftsecfont}{\rmfamily\mdseries\upshape}
\renewcommand{\cftsecpagefont}{\rmfamily\mdseries\upshape} % No bold!

%%% END Article customizations

%%% The "real" document content comes below...

\title{Instructions for CNP thesis template}
\author{Jos\'{e} R. Fernandes}

%\date{} % Activate to display a given date or no date (if empty),
         % otherwise the current date is printed 

\begin{document}
\maketitle

\section{About the template}

This is Princeton University's template, written by Jeffrey Scott Dwoskin,\\ 
\url{http://www.math.princeton.edu/graduate/tex/puthesis.html}\\
with minor modifications to B5 format, from Universitat Pompeu Fabra\\
 \url{http://www.upf.edu/bibtic/en/guiesiajudes/eines/tesis/quart.html\#template}\\

\section{Getting started with LaTeX}

Seeing your beautiful thesis seamlessly compiled is a joy of modern life. However, errors can be opaque to understand and very frustrating to debug if this is your first time using LaTeX.

Get started with one of the many of the tutorials available online and play with a smaller document. For specific topics, e.g., how to write equations, LaTeX's wikibook is great, \url{http://en.wikibooks.org/wiki/LaTeX}. Otherwise, there is no shortage of good answers online to the most obscure problems you might run into. Stackoverflow is particularly good, \url{http://stackoverflow.com/}. If you feel despair then you are not asking the right question. 

\section{Running the template}

\begin{enumerate}
\item{Download a TeX distribution, e.g., \url{https://tug.org/mactex/}}
\item{Download a typesetting program. I used TeXworks, \url{http://www.tug.org/texworks/}. MacTeX already includes one, but at least I know that TeXworks runs.}
\item{Open ThesisTemplateCNP.tex}
\item{Run in pdfLaTeX twice. Then run the bibliography twice (I used BibDesk, which is included in MacTeX). Finally, typeset in pdfLaTeX again. It's four button clicks.}
\end{enumerate}

The template runs in Mac as described above. The template hasn't been tested with other operating systems, so it might take some debugging to adapt it. 

\section{Editing the template}

\subsection{General organization, chapters and frontmatter}

The scripts are organized hierarchically. All global definitions, such as paper size, margins, etc., go into the file ThesisTemplateCNP.tex. Except for thesis and chapter titles, and names, no content should be in this main file.

Frontmatter files, e.g., abstract.tex, are located inside the ./chapters folder. Frontmatter definitions are located in the folder ./SupportFilesTemplates.

The main file calls chapter files, in the ./chapters folder, and frontmatter files (dedication, acknowledgments, abstract, t\'{i}tulo and resumo, and others you might include, for instance, a table of figures or copyright page). Each of the chapter files calls subsection files, located in respective ./chapters/subsectCh* folders. Figures for each chapter are located in the ./chapters/figuresCh* folders.

\subsection{UNL covers}

The covers for ITQB theses are located in the ./chapters folder. They are editable in Adobe Illustrator. Alternatively you can create your own cover, convert the files to pdf and call them change the includepdf lines in the main file. 

\subsection{Bibliography}

The information necessary for references is in bibliographyFile.bib. This file should not be edited directly. Instead, use a LaTeX bibliography editor such as BibDesk (included in the MacTeX package). Remember that references will not appear without typesetting the bibliography twice. 

\section{Questions}

Feel free to email me if you have any questions: fer.jose.nandes@gmail.com. And to edit and improve the template.


\end{document}
